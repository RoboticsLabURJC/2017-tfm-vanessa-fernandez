\documentclass[a4paper, 12pt, oneside]{book}

\usepackage[utf8]{inputenc}
\usepackage{lmodern}
\usepackage{layout}
\usepackage{emptypage}
\usepackage{fancyhdr}
\usepackage[spanish,es-tabla]{babel}
%\usepackage[Conny]{fncychap}
%\usepackage{graphicx}
\usepackage{subfigure} % subfiguras
\usepackage{caption}
\usepackage{mathtools}
\usepackage{hyperref}
\usepackage[a4paper,top=3cm, bottom=3cm, inner=2.5cm, outer=2.5cm]{geometry}
\usepackage{listings}
\usepackage[spanish]{babel}
\usepackage{url}
\usepackage{float}
\usepackage{multirow}
\usepackage{rotating} 
\usepackage{color}
\usepackage{colortbl}
\usepackage[table]{xcolor}
\usepackage[spanish]{babel}
\usepackage{enumerate}
\usepackage{natbib}
\usepackage[acronym, nonumberlist]{glossaries}
\makeglossaries
\newacronym{api}{API}{Application Programming Interface}
\newacronym{cnn}{CNN}{Redes Neuronales Convolucionales}
\newacronym{csail}{CSAIL}{MIT Computer Science \& Artificial Intelligence Lab}
\newacronym{gui}{GUI}{Graphical User Interface}
\newacronym{hdf5}{HDF5}{Hierarchichal Data Format version 5}
\newacronym{ia}{IA}{Inteligencia Artificial}
\newacronym{ice}{ICE}{Internet Communications Engine}
\newacronym{lstm}{LSTM}{Long Short-Term Memory}
\newacronym{rna}{RNA}{Redes Neuronales Artificiales}
\newacronym{rnn}{RNN}{Redes Neuronales Recurrentes}
\newacronym{ros}{ROS}{Robot Operating System}
\newacronym{sdf}{SDF}{Simulation Description Format}
\newacronym{svg}{SVG}{Scalable Vector Graphics}
\newacronym{va}{VA}{Visión Artificial}
\newacronym{xml}{XML}{Extensible Markup Language} % Archivo que contiene los acrónimos

\makeatletter
\renewcommand{\@makeschapterhead}[1]{%
%  \vspace*{50\p@}%
  \vspace*{0\p@}%
  {\parindent \z@ \raggedright
    \normalfont
    \interlinepenalty\@M
    \Huge \bfseries  #1\par \nobreak
%    \vskip 40\p@
    \vskip 15\p@
  }}
\makeatother

\renewcommand{\baselinestretch}{1.4}
\setlength{\headheight}{16pt} 
\captionsetup{justification=justified}
\pretolerance=1000

\chead[]{}
\rhead[]{}
\renewcommand{\headrulewidth}{0.5pt}

\pagestyle{empty}

\title{Conducción autónoma de un vehículo en un simulador mediante aprendizaje extremo a extremo basado en visión}
\author{Vanessa Fernández Martínez}

\lstset{
	float=hbp,
	basicstyle=\ttfamily\small,
	columns=flexible,
	tabsize=4,
	frame=single,
	extendedchars=true,
	showspaces=false,
	showstringspaces=false,
	numbers=none,
	numberstyle=\tiny,
	breaklines=false,
	breakautoindent=true,
	captionpos=b
}
\setcounter{tocdepth}{4}
\setcounter{secnumdepth}{4}

\definecolor{lightgray}{gray}{0.9}

\begin{document}
%%%%%%%%%%%%%%% Portada %%%%%%%%%%%%%%%%%%%%
\begin{titlepage}
	\begin{center}
		\vspace*{3mm}
		\begin{center}
			\includegraphics[width=0.4\linewidth]{figures/logo.jpg}
		\end{center}
		\vspace{6.5mm}
		
		\fontsize{15.5}{14}\selectfont ESCUELA TÉCNICA SUPERIOR DE INGENIERÍA DE INFORMÁTICA
		\vspace{13mm}
		
		\fontsize{14}{14}\selectfont MÁSTER UNIVERSITARIO EN VISIÓN ARTIFICIAL 
		
		\vspace{55pt}
		
		\fontfamily{lmss}\fontsize{15.7}{14}\selectfont \textbf{TRABAJO FIN DE MÁSTER} 
		
		\vspace{20mm}
		\begin{huge}
			Conducción autónoma de un vehículo en \vspace{0.4cm} simulador mediante aprendizaje extremo a \\ \vspace{0.4cm} extremo basado en visión
		\end{huge}
		
		\vspace{20mm}
		
		\begin{large}
			Autor: Vanessa Fernández Martínez
			
			Tutor: José María Cañas Plaza
			
			Cotutor: Francisco Miguel Rivas Montero
			
			\vspace{10mm}
		\end{large}
		\begin{normalsize}
			Curso académico 2018/2019		
		\end{normalsize}
		\vspace{10mm}
		
	\end{center}
	
\end{titlepage}

\pagenumbering{Roman}

%%%%%%%%%%%%%%% Agradecimientos %%%%%%%%%%%%
\chapter*{Agradecimientos}

En primer lugar, me gustaría dar las gracias a mi familia por el apoyo que me ha dado a lo largo de este tiempo. Todos ellos han mostrado interés por el trabajo y lo han hecho más llevadero. Sobre todo quiero mencionar a mis padres y a mi hermana, los cuales me han animado, me han dado todo su cariño, y me han ayudado en los momentos más difíciles, pues sin ellos no hubiese sido posible conseguirlo.\\

Gracias a José María y a Fran, por su dedicación y apoyo durante todos estos meses de desarollo del trabajo, por los ánimos, por su paciencia, por sus ideas y su ayuda. Y por supuesto, por haberme transmitido su pasión por la visión artificial, la robótica y la investigación.\\

Por otro lado, quiero dar las gracias a mis compañeros de la universidad que me han animado durante todos estos años y han compartido conmigo el estrés de los exámenes: Nuria, Irene, Carolina, Fran, Aitor, Carlos, Nacho, Miguel Ángel, Celia, etc. Además, gracias a mis amigos que me han apoyado durante tantos años en los buenos y los malos momentos: Marcos, Nerea, Lizaveta, Chaimae, Laura, Manal, Marian,  Isabella, etc.\\

Por último, no puedo olvidarme de dar las gracias al CNAH, no solamente por la natación, sino sobre todo por la gente que lo forma, especialmente el equipo femenino. Gracias a las más jóvenes del equipo, por sacarme una sonrisa en todo momento. A las mayores: Jessi, Concha, María, Elena y Laura, que me han hecho desconectar cada día del estrés y pasármelo como nunca. No puedo olvidarme, de Manolo Revilla, que desde que llegué al club me ha tratado como una amiga y se ha procupado por mí.\\


\begin{flushright}
	\emph{¡Muchas gracias a todos!}
\end{flushright}


%%%%%%%%%%%%%%% Resumen %%%%%%%%%%%%%%%%%%%%
\chapter*{Resumen}

En la última década, en \acrfull{va} se está estudiando ampliamente la conducción autónoma. Los humanos somos capaces de mirar a la carretera y saber al instante que acción llevar a cabo. En función a la situación en la que nos encontramos sabemos qué acciones llevar a cabo para lograr una buena conducción.  Sin embargo, este procedimiento es más complicado para los ordenadores. En la actualidad se está investigando ampliamente cómo emplear las \acrfull{rna} para materializar comportamiento autónomo en vehículos. En este proyecto se estudia la conducción autónoma en simuación mediante redes neuronales basadas en información visual.\\

Las decisiones tomadas por la red neuronal están determinadas por los datos empleados durante el entrenamiento de la red. Por lo tanto, cuanto más representativo sea el conjunto de datos, mejor rendimiento se espera que tenga la red. Además, el coche deberá ser capaz de conducir en diferentes entornos. Por ello, se ha creado un conjunto de datos propio a partir de un piloto manual que conduce de forma autónoma usando visión.\\


El objetivo principal de este trabajo es comparar diferentes modelos de redes neuronales que se pueden emplear para la conducción autónoma. Por este motivo, se han estudiado y probado diferentes arquitecturas de redes. Primero, se ha estudiado el empleo de redes neuronales convolucionales \textit{de clasificación} en conducción autónoma, realizando múltiples pruebas para tratar de conseguir la red más robusta posible y emplearla en el pilotaje del vehículo. Segundo, se ha estudiado el empleo de redes neuronales convolucionales y \textit{recurrentes de regresión} en conducción autónoma. Se han llevado a cabo diversas pruebas también para intentar conseguir la red más robusta posible y emplearla en el pilotaje del coche.
\cleardoublepage

%%%%%%%%%%%%%%% Índices %%%%%%%%%%%%%%%%%%%%
%\cleardoublepage
\renewcommand{\tablename}{Tabla}
%\renewcommand{\listtablename}{Índice de tablas}
%\tableofcontents

%\cleardoublepage % Í­ndice de figuras
%\addcontentsline{toc}{chapter}{\listfigurename}
%\listoffigures

%\cleardoublepage % Í­ndice de tablas
%\addcontentsline{toc}{chapter}{Índice de tablas}
%\listoftables 
\cleardoublepage
\tableofcontents % indice de contenidos

\cleardoublepage
\listoffigures % indice de figuras
\addcontentsline{toc}{chapter}{Índice de figuras} % para que aparezca en el indice de contenidos

\cleardoublepage
\listoftables % indice de tablas
\addcontentsline{toc}{chapter}{Índice de tablas} % para que aparezca en el indice de contenidos


%%%%%%%%%%%%%%% Acronimos %%%%%%%%%%%%%%%%%%%%

% Incluye el listado de acrónimos utilizados en el trabajo. 
\printglossary[type=\acronymtype,title={Acrónimos}]
% Añade el resto de acrónimos si así se desea. Si no elimina el comando siguiente
\glsaddallunused  
 
 
%%%%%%%%%%%%%%% Capí­tulos %%%%%%%%%%%%%%%%%%
\pagestyle{fancy}
\pagenumbering{arabic}
\setlength{\parindent}{6mm}

\lhead[]{CAPÍTULO \thechapter. INTRODUCCIÓN}
\chapter{Introducción}\label{cap.introduccion}

En este capítulo se definirá el contexto en el cual se sitúa este proyecto, y la motivación principal que ha llevado a su desarrollo. Se explicará de forma general qué es la visión artificial, así como el uso de redes neuronales en la misma. Además, se expondrá qué es la conducción autónoma.

\section{Contexto y motivación}

Desde la antigüedad el ser humano ha soñado con crear máquinas capaces de pensar. Cuando surgieron los primeros ordenadores programables, las personas se plantearon la idea de lograr que estos computadores adquirieran inteligencia, adquiriendo capacidades empleadas para realizar tareas propias de los humanos. Algunos ejemplos de estas tareas son entender el habla o las imágenes, y automatizar tareas rutinarias. El campo que desarrolla estas tareas se denomina \acrfull{ia}~\cite{Goodfellow} y cada vez tiene más presencia en temas de investigación.\\

En \acrshort{ia} existen diversos desafíos muy interesantes; sin embargo, en la mayoría de ellos es extremadamente difícil alcanzar el rendimiento y la eficiencia del cerebro humano. Las máquinas nos superan en tareas como procesamiento de gran cantidad de datos, almacenamiento de información o tareas de razonamiento como el juego de ajedrez. Sin embargo, algunas habilidades que el ser humano realiza inconscientemente, como caminar o ver, son muy complejas para las máquinas; y por ello el cerebro humano supera a la máquina en este tipo de tareas.\\

La \acrshort{ia} comprende diferentes campos (Figura \ref{fig.ia}): \textit{Machine Learning}, \textit{Knowledge Engineering}, Lingüística computacional, \acrfull{rna}~\cite{rna}, Procesamiento del lenguaje natural, Minería de datos, \acrfull{va}, etc. Este proyecto se enfoca en la \acrshort{va}, que trata de analizar y procesar imágenes de tal forma que un ordenador sea capaz de interpretar dichas imágenes. La \acrshort{ia} intenta conseguir que una máquina realice el mismo proceso que el Sistema Visual Humano de tal forma que sea capaz de tomar decisiones y actuar en función de la situación en que se encuentre.\\

\begin{figure}[H]
  \begin{center}
    \includegraphics[width=0.8\textwidth]{figures/Introduccion/venn.png}
		\caption{Diagrama de Venn que muestra los campos que engloba la \acrshort{ia}}
		\label{fig.ia}
		\end{center}
\end{figure}

El aprendizaje de las máquinas es un punto de encuentro de diferentes disciplinas que engloba a la estadística, la geometría, la programación y la optimización, entre otras. La \acrshort{va} intenta simular las capacidades del ojo y el cerebro humano, empleando los conceptos de estas disciplinas.\\

Uno de los problemas que se está estudiando ampliamente en \acrshort{va} en la última década es la conducción autónoma. Los humanos somos capaces de mirar a la carretera y saber al instante si el coche que conducimos está en una curva o una recta, si hay coches alrededor y cómo interactúan entre ellos. En función a la situación en la que nos encontramos sabemos qué acciones llevar a cabo para lograr una buena conducción.  Sin embargo, este procedimiento es más complicado para los ordenadores. En la actualidad se está investigando ampliamente cómo emplear las \acrfull{rna} para predecir comportamiento autónomo en vehículos.\\

El objetivo principal de este proyecto es el estudio de conducción autónoma en simulación mediante \acrfull{rna}. En el estudio se incluyen diferentes arquitecturas de redes neuronales empleadas para imitar el comportamiento humano en un vehículo. Además, se presentarán los resultados de las distintas redes neuronales.


\section{Visión artificial}

Las primeras aplicaciones de la Visión Artificial datan de los años 60, donde destaca la creación del perceptrón, desarrollado por Frank Rosenblatt. Se considera la primera neurona artificial con capacidades de aprendizaje. Era capaz de distinguir figuras simples, como triángulos y cuadrados, a base de ensayo y error.\\

En la década de los 70 surgen las primeras aplicaciones comerciales de la \acrshort{va}, como por ejemplo el reconocimiento óptico de caracteres (OCR). Sin embargo, no es hasta los años 80 y 90 cuando la \acrshort{va} toma mayor peso. En la actualidad es una parte muy importante que contribuye a la transformación digital de diversos sectores. Por este motivo sólo es posible dar una pequeña pincelada sobre las múltiples aplicaciones en las que se ha aplicado hasta el momento.\\

Un claro ejemplo, es la navegación en robótica (Figura \ref{fig.robot}), donde la visión constituye una capacidad sensorial más para la percepción del entorno que rodea al robot. Generalmente se recurre a técnicas de visión estereoscópica con el fin de reconstruir la escena 3D. En algunas ocasiones se añade algún módulo de reconocimiento con el fin de identificar la presencia de determinados objetos, hacia los que debe dirigirse o evitar. Cualquier información que pueda extraerse mediante \acrshort{va} supone una gran ayuda para el movimiento del robot. \\

\begin{figure}[H]
  \begin{center}
    \includegraphics[width=0.3\textwidth]{figures/Introduccion/robot.jpg}
		\caption{Navegación en robótica mediante \acrshort{va}}
		\label{fig.robot}
		\end{center}
\end{figure}


Otro ejemplo donde la \acrshort{va} supone un gran avance es en la comunidad médica, donde permite diagnosticar con mayor rapidez y detalle enfermedades y lesiones. De esta forma es posible aplicar tratamientos personalizados y eficaces en menor tiempo. Un claro ejemplo de investigadores que emplean \acrshort{va} es el \acrfull{csail}~\cite{cancer}, donde el desarrollo de algoritmos que analizan mamografías de una forma novedosa permite ayudar a detectar el cáncer de mama (Figura \ref{fig.cancer}) con hasta cinco años de anticipación.\\

\begin{figure}[H]
  \begin{center}
    \includegraphics[width=0.4\textwidth]{figures/Introduccion/cancer.png}
		\caption{Detección de cáncer de mama}
		\label{fig.cancer}
		\end{center}
\end{figure}


Una posible aplicación es el mantenimiento e inventariado urbano. Es posible identificar problemas en intalaciones y mobiliario urbano (averías, mal estado de contenedores (Figura \ref{fig.contenedor}), socavones en la vía pública, etc) mediante cámaras ubicadas por ejemplo en autobuses. Los mantenimientos de infraestructuras de transporte, como vías y cables ferroviarios, pueden programarse automáticamente implantando sistemas de \acrshort{va} en los propios trenes. 


\begin{figure}[H]
  \begin{center}
    \includegraphics[width=0.5\textwidth]{figures/Introduccion/contenedor.png}
		\caption{Detección de contenedores}
		\label{fig.contenedor}
		\end{center}
\end{figure}

La reducción de accidentes gracias a vehículos autónomos es una realidad gracias a la \acrshort{va}, ya que los sistemas de guiado que poseen estos vehículos están basados en esta visión. Algunos ejemplos de estos sistemas (Figura \ref{fig.car}) son: los sistemas de aviso de cambio de carril, o de control de velocidad de crucero. 

\begin{figure}[H]
  \begin{center}
    \includegraphics[width=0.5\textwidth]{figures/Introduccion/car.jpg}
		\caption{Conducción autónoma}
		\label{fig.car}
		\end{center}
\end{figure}


\section{Conducción autónoma}

La conducción autónoma pretende que un vehículo sea capaz de conducir sólo en base a los datos proporcionados por determinados sensores (cámaras, LIDAR, etc), es decir, es capaz de aprender las normas de circulación. La posibilidad de crear un sistema capaz de conducir un vehículo ya se había contemplado en el siglo pasado. Sin embargo, la tecnología disponible en ese momento no permitía resolver una tarea tan compleja. Después de los avances tecnológicos cambió este hecho.\\

A finales del siglo pasado, algunos investigadores \cite{Dickmanns} \cite{alvinn} experimentaton con la creación de las primeras arquitecturas de conducción autónoma, desarrollando y probando algunos prototipos que podían conducir en calles reales. Estas pruebas se realizaron en áreas controladas y protegidas, y la conducción no fue lo suficientemente buena como para crear un producto de uso seguro. Estos experimentos dejaron claro que aún quedaba mucho para obtener una solución, pero al mismo tiempo, demostraron que la conducción autónoma podría convertirse en una perspectiva real.\\

En los últimos años se ha hecho mayor incapié en la investigación de la conducción autónoma con el fin de solventar el incremento de la tasa de muerte por accidentes de tráfico. Aunque algunos de estos accidentes se producen por fallos mecánicos del vehículo, la mayoría de dichos accidentes se debe a imprudencias y distracciones humanas. La conducción autónoma eliminaría estas distracciones haciendo posible la disminución de accidentes.\\

Hoy en día, cada vez existen más fabricantes de vehículos que incorporan tecnologías de conducción autónoma. Existe un estándar elaborado por la Sociedad de Ingenieros Automotrices (SAE), conocido como J3016 \cite{sae}, que establece los niveles de conducción autónoma según la capacidad del vehículo.\\

\begin{itemize}
    \item Nivel 0: No hay automatización de la conducción. Las tareas de conducción son realizadas en su totalidad por el conductor.
    
    \item Nivel 1: Asistencia al conductor. El vehículo posee algún sistema de automatización de la conducción (control de crucero, autoaparcamiento), ya sea para el control de movimiento longitudinal o el movimiento lateral, aunque no ambas cosas al mismo tiempo. El conductor realiza el resto de tareas de conducción, por lo que debe estar siempre atento.
    
    \item Nivel 2: Automatización parcial. Considera que el conductor ya no tiene que conducir en todo momento y que el coche empieza a ser realmente autónomo, aunque con ciertos matices. El vehículo es capaz de actuar de manera independiente dentro de escenarios controlados y en situaciones específicas de conducción. El conductor debe seguir prestado atención a lo que ocurre a su alrededor para evitar posibles riesgos. Un buen ejemplo de Nivel 2 de conducción autónoma pueden ser los modelos BMW Serie 7 o el Mercedes Clase E, capaces de moverse solos durante un tiempo o con el sistema de asistente de atascos.
    
    \item Nivel 3: Automatización condicional. En este nivel, el coche comienza a interactuar con el entorno que le rodea y es capaz de analizar posibles riesgos externos con el fin de evitarlos. Ya no se habla de conductor sino que hablamos de un usuario preparado para intervenir, es decir, el coche ya conduce completamente solo y el conductor es un simple vigilante de que todo funcione correctamente. El coche está preparado para ser conducido de manera habitual en cualquier momento.
    
    \item Nivel 4: Alta autonomía. En este nivel el sistema cuenta tanto con los sistemas de automatización presentes en el anterior nivel, como con sistemas de detección de objetos y eventos. Además, es capaz de responder ante ellos. El sistema de automatización de la conducción tiene un sistema de respaldo para actuar en caso de fallo del sistema principal y poder conducir hasta una situación de riesgo mínimo. En algunas situaciones es posible que el vehículo no siga conduciendo.
    
    \item Nivel 5: Autonomía total. Este nivel cuenta con todos los beneficios del sistema de automatización del nivel 4. Sin embargo, la diferencia es que en este caso el vehículo podría seguir conduciendo en todo momento o circunstancia.
\end{itemize}

Ejemplos importantes de conducción autónoma son: el DARPA Grand Challenge y el Urban Challenge. El DARPA
Grand Challenge, organizado en 2004 y 2005 en Estados Unidos, fue una carrera de vehículos autónomos que debían recorrer 120 kms por el desierto de Nevada sin intervención humana y disponiendo únicamente de un listado de puntos intermedios entre el principio del circuito y el final.  El Urban Challenge, organizado en 2007, fue una carrera de vehículos autónomos por zona urbana en la que debían recorrer 96 km en menos de 6 horas.\\

Como resultado de estos desafíos, destaca el proyecto ganador de 2005 de la Universidad de Stanford, cuyos miembros liderados por Sebastian Thrun acabaron desarrollando el vehículo autónomo de Google. En 2014, Google reveló un nuevo prototipo de su automóvil sin conductor (Firefly), que no tenía volante, pedal de acelerador o freno, siendo 100\% autónomo, aunque era un prototipo empleado exclusivamente para pruebas. En la actualidad este vehículo se conoce como Waymo.\\

Este año, BMW y Mercedes-Benz han decidido unir fuerzas para desarrollar coches autónomos. Estas dos grandes compañías desarrollarán tecnologías para la creación de los próximos vehículos autónomos. Pretenden desarrollar ayudas a la conducción avanzadas y sistemas que automaticen la conducción en autopista y en el aparcamiento. Su objetivo es crear sistemas de conducción autónoma de nivel 4.\\

Tesla incluye el sistema inteligente Autopilot que alcanza el nivel de conducción 3. Sin embargo, Elon Musk ha anunciado este año que en 2020 será posible hablar de coches completamente autónomos (niveles 4 y 5), ya que su sistema Autopilot ofrecerá una conducción 100\% autónoma, donde el conductor pasaría a ser un mero espectador durante la conducción.\\

En la actualidad se están desarrollando sistemas que toman decisiones empleando una red neuronal profunda, la cual recibe información del entorno mediante diferentes sensores (LIDAR, radar, cámaras, etc.).  A partir de los datos recogidos por los sensores la red predice unos valores de salida que serán los empleados para la conducción.  Las decisiones tomadas por la red neuronal están determinadas por los datos empleados durante el entrenamiento de la red. Por lo tanto, cuanto más representativo sea el conjunto de datos, mejor rendimiento se espera que tenga la red, ya que conocerá todas las situaciones posibles en las que puede estar el vehículo.\\

Hoy en día el principal obstáculo para la conducción autónoma no se deriva de las limitaciones de la tecnología, sino de factores políticos, jurídicos, de regulación, de infraestructura y de responsabilidad que se deben abordar. A pesar de estas dificultades la investigación ha hecho muchos avances.\\


\section{Redes neuronales artificiales}

Una Red Neuronal Artificial \acrshort{rna} es un modelo matemático inspirado en el comportamiento biológico de las neuronas y en cómo se organizan dichas neuronas en el cerebro. Estas redes intentan imitar ciertas  características propias de los seres humanos, como pueden ser la capacidad de memorizar y de asociar hechos. Estas neuronas siguen la misma estructura jerárquica que el cerebro humano, es decir, las diversas neuronas se organizan por capas.\\

Una comparativa entre las neuronas biológicas y las neuronas artificiales se puede observar en la Figura \ref{fig.neurona}. Las neuronas biológicas constan de un cuerpo celular o soma que contiene un núcleo y ramas denominadas dendritas. Las dendritas transfieren la información de las células próximas mediante sinapsis al soma. Además, poseen un axón que lleva el impulso nervioso del soma a otras neuronas. Sin embargo, la neurona artificial es un modelo simplificado de las neuronas biológicas. Las sinapsis y las dendritas de  la  neurona  artificial  son las  entradas  al  elemento procesador (soma). Cada una de estas entradas posee un peso asociado de conexión. Cada una de estas entradas es multiplicada por su peso y se suman generalmente estos productos, que pasan entonces a la función de la transferencia para generar un resultado que se transmita por la salida (axón).\\

\begin{figure}[H]
  \begin{center}
    \includegraphics[width=0.6\textwidth]{figures/Introduccion/neurona.png}
		\caption{Comparación de neurona biológica (izquierda) y neurona artificial (derecha).}
		\label{fig.neurona}
		\end{center}
\end{figure}

En este proyecto emplearemos la red multicapa, que consta de dos o más capas de neuronas interconectdas. Cada una de las capas puede hacer un tipo de transformación en su entrada, donde las señales atraviesan todas las capas. Cuando existen más de dos capas, hablamos de que la red posee capas ocultas. En esta red normalmente las capas iniciales realizan generalizaciones simples, y en capas más profundas se hacen las generalizaciones más complejas.\\

La característica más especial del aprendizaje con redes neuronales es la capacidad de aprender y generalizar gracias a una base de datos específica para el problema que se debe tratar. Una vez esta red es entrenada, es capaz de estimar resultados para ejemplos que no ha visto anteriormente.\\

En este proyecto se emplearán dos tipos de redes neuronales para resolver el mismo problema. Por un lado se utilizan redes neuronales convolucionales y redes neuronales recurrentes.


\subsection{Redes neuronales convolucionales}

Las \acrfull{cnn} son una clase de red neuronal artificial profunda que se emplean principalmente para clasificar imágenes, agrupar estas imágenes por similitud y realizar el reconocimiento de objetos dentro de las escenas. Este tipo de redes pueden identificar rostros, individuos, letreros de calles, tumores y muchos otros aspectos de los datos visuales.\\

Las \acrshort{cnn} se basan en la arquitectura de la corteza visual del cerebro humano. Las \acrshort{cnn} aplican una serie de filtros a los datos para extraer y aprender características de nivel superior, que el modelo puede usar para la clasificación, el reconocimiento u otro tipo de tarea. Las \acrshort{cnn} están formadas por diferentes tipos de capas que veremos en las próximas subsecciones: capas convolucionales, capas de agrupación o \textit{pooling}, y capas completamente conectadas o \textit{fully connected}. \\

Las \acrshort{cnn} siguen el esquema de la Figura \ref{fig.arquitectura_cnn} \cite{rna8}. Normalmente este tipo de redes está formado por un conjunto de módulos convolucionales, que consisten en una capa convolucional seguida de una capa \textit{pooling}. La capa convolución realiza una operación de convolución, mientras que la capa de agrupación o \textit{pooling} genera características invariantes calculando estadísticas de las activaciones de convolución a partir de un campo receptivo (un pequeño campo de la capa anterior). En este tipo de redes, cada neurona de una capa oculta se conecta al campo receptivo local. En la capa convolucional, las neuronas se distribuyen en diversas capas paralelas, denominadas mapas de características. En un mapa de características cada neurona está conectada a un campo receptivo local. Además, para cada mapa de características todas las neuronas comparten el mismo parámetro de peso conocido como \textit{kernel} o filtro.\\


\begin{figure}[H]
  \begin{center}
    \includegraphics[width=0.5\textwidth]{figures/Introduccion/arquitectura_cnn.png}
		\caption{Estructura de \acrshort{cnn}}
		\label{fig.arquitectura_cnn}
		\end{center}
\end{figure}

Además, en este tipo de redes es importante tener en cuenta tanto las dimensiones de entrada como las dimensiones de las distintas capas. Las imágenes de entrada normalmente tienen dimensiones W x H x C, donde W es el ancho de la imagen, H es la altura, y C es el número de canales de la imagen. Cuando la información va atravesando las capas, normalmente se disminuye los valores W x H, mientras que la profundidad de la red aumenta. Las primeras capas proporcionan una información localizada, es decir, el donde; mientras que las capas finales proporcionan información acerca del contenido de la imagen, es decir, el qué.


\subsection{Redes neuronales recurrentes}

Las \acrfull{rnn} son un tipo de red neuronal artificial, donde la idea es usar información secuencial en vez de información independiente como en las redes tradicionales. En algunos casos emplear información independiente es mala idea, como puede ser en la predicción de la siguiente palabra en una cadena de texto, ya que sin información previa la red no es capaz de predecir la palabra.\\

Las \acrshort{rnn} permiten que la información previa al instante actual persista. Una red neuronal recurrente se puede considerar como copias múltiples de la misma red, cada una de las cuales pasa un mensaje a su sucesor. En la Figura \ref{fig.rnn} se puede ver un esquema de \acrshort{rnn}.\\

\begin{figure}[H]
  \begin{center}
    \includegraphics[width=0.5\textwidth]{figures/Introduccion/esquema_rnn.png}
		\caption{Esquema de \acrfull{rnn}}
		\label{fig.rnn}
		\end{center}
\end{figure}

Las \acrfull{rnn} aprenden a emplear la información pasada en los casos donde la brecha entre la información relevante y la información actual es pequeña. Pero habrá casos donde necesitemos más contexto, como por ejemplo si queremos predecir la última palabra del texto ``Crecí en Francia... Hablo francés con fluidez". La información reciente sugiere que la siguiente palabra es un idioma, pero si queremos concretar qué idioma es, necesitamos el contexto desde más atrás. Sin embargo, a medida que aumenta la brecha, las \acrshort{rnn} no son capaces de aprender a conectar la información. En cambio las \acrshort{lstm} no tienen ese problema.\\

Las redes \acrfull{lstm} \cite{lstm} son un tipo especial de \acrshort{rnn} capaz de aprender dependencias a largo plazo. Esta clase de redes fueron diseñadas para recordar información de periodos de tiempo largo. Se explicará más acerca de este tipo de red en las siguientes subsecciones.\\


\subsection{Tipos de capas}

En las siguientes susbsecciones se explican los diferentes tipos de capas empleadas en las redes \acrshort{cnn} y \acrshort{lstm}.


\subsubsection{Capa Convolucional}

Las capas convolucionales son las más importantes de una \acrshort{cnn}. La operación de convolución (Figura \ref{fig.convolucion}) recibe como entrada una imagen y luego aplica sobre ella un filtro o \textit{kernel} que devuelve un mapa de características. Con esta operación se reduce el tamaño de los parámetros. En las capas convolucionales existen diferentes parámetros a tener en cuenta:\\

\begin{itemize}
    \item Dimensiones de los filtros de convolución. Suelen ser una matriz cuadrada (tamaño M x M). Cada píxel de cada mapa de características solamente tendrá en cuenta los píxeles que estén dentro del filtro.
    
    \item Número de filtros de convolución. Determina la profundidad del volumen de salida. Cada filtro genera un mapa decaracterísticas.
    
    \item \textit{Stride}. Determina cuánto vamos a deslizar el filtro sobre la matriz de entrada. Por  ejemplo, cuando el stride es 1 se mueven los filtros 1 píxel a la vez. 
    
    \item \textit{Padding}. Añade alrededor de la matriz de entrada ceros para evitar perder dimensiones tras la convolución.
\end{itemize}

\begin{figure}[H]
  \begin{center}
    \includegraphics[width=0.6\textwidth]{figures/Introduccion/convolucion.png}
		\caption{Ejemplo de operación de convolución}
		\label{fig.convolucion}
		\end{center}
\end{figure}

Se puede calcular el tamaño del volumen de salida en función al volumen de entrada (W), el tamaño del filtro de convolución (M), el stride aplicado (S) y la cantidad de zero-padding que se aplica (P). El tamaño del volumen de salida se calcula como: (W - M + 2P) / (S+1).\\


Tras aplicar la convolución, se aplica una función de activación a los mapas de características. Esta función de activación es no lineal para conseguir modelos no lineales. La función de activación más usada es la función ReLU.


\subsubsection{Capa de \textit{Pooling}}

La capa de \textit{pooling} o de agrupación se coloca normalmente detrás de la capa convolucional. Se emplea para reducir las dimensiones espaciales (ancho x alto) del volumen de entrada, pero no afecta a la dimensión de profundidad del volumen.\\

En ocasiones la operación que realiza la capa de \textit{pooling} se denomina reducción de muestreo debido a que la reducción de tamaño lleva a pérdidas de información. Aunque esta pérdida puede ser buena para la red por dos motivos: (1) trabaja en reducir el sobreajuste, (2) la disminución del tamaño produce un menor consumo de memoria durante el entrenamiento de las redes.\\

El funcionamiento de esta capa se basa en una ventana deslizante que actua sobre el volumen de entrada. La operación realizad por esta ventana deslizante depende del tipo de \textit{pooling} elegido. Las clases de submuestreo más empleadas son:

\begin{itemize}
    \item \textit{Max pooling}: Se queda con el valor máximo de los valores de la ventana deslizante. Se puede ver un ejemplo en la Figura \ref{fig.pooling}
    
    \item \textit{Average pooling}: Calcula cada píxel del volumen de salida realizando el promedio de los píxeles que se encuentran dentro de la ventana deslizante del volumen de entrada. Esta operación se hace canal por canal.
\end{itemize}

\begin{figure}[H]
  \begin{center}
    \includegraphics[width=0.5\textwidth]{figures/Introduccion/pooling.png}
		\caption{Ejemplo de capa \textit{max pooling}}
		\label{fig.pooling}
		\end{center}
\end{figure}


\subsubsection{Capa \textit{Fully connected}}

Las capas completamente conectadas o \textit{fully connected} conectan cada neurona de la capa de entrada con cada neurona de la capa de salida. Además, asignan un determinado peso a cada conexión. La gran cantidad de conexiones produce que exista un gran número de parámetros configurables en esta capa.


\subsubsection{Capa LSTM}

La unidad LSTM puede añadir o quitar información, lo cual lo hace mediante estructuras denominadas puertas. Estas puertas son como una especiede camino para dejar pasar información. Una unidada LSTM tiene tres puertas.

\begin{itemize}
    \item \textit{Forget gate}. Decide que información debe desechar. 
    \item \textit{Input gate}. Esta capa decide que valores se deben actualizar.
    \item La unidad produce un \textit{output} o valor de salida.
\end{itemize}

\begin{figure}[H]
  \begin{center}
    \includegraphics[width=0.5\textwidth]{figures/Introduccion/unidad_lstm.png}
		\caption{Unidad \acrshort{lstm}}
		\label{fig.unidad_lstm}
		\end{center}
\end{figure}

\lhead[]{CAPÍTULO \thechapter. OBJETIVOS}
\chapter{Objetivos}\label{cap.objetivos}

Una vez explicado el contexto de este proyecto, se describirán en este capítulo los objetivos, los requisitos y la metodología empleados.

\section{Objetivos}

El propósito principal de este proyecto es el estudio de diferentes redes neuronales basadas en información visual que permitan a un vehículo ser capaz de conducir de forma autónoma. El coche deberá ser capaz de conducir en diferentes circuitos en el simulador Gazebo. Los entornos de entrenamiento y de prueba serán diferentes para que el vehículo sea capaz de aprender diferentes estímulos que le permitan conducir en situaciones variadas.\\

Este objetivo genérico se ha articulado en cuatro subobjetivos concretos:

\begin{enumerate}
    \item Se creará una aplicación de control visual con la infraestructura necesaria que se comunica con el simulador Gazebo, donde se podrá ver el resultado de la predicción de las redes neuronales. Esta aplicación tendrá diferentes ingredientes: 

    \begin{itemize}
        \item Infraestructura de conexión en el simulador, tanto para recoger imágenes de la cámara a bordo del coche como para enviar órdenes al acelerador y al volante.
        \item Permitirá cargar y emplear diferentes redes neuronales, además de incluir una GUI.
        \item En un fichero específico se guarda el código que genera las órdenes de velocidad predichas por la red neuronal y se envían al coche simulado.
    \end{itemize}
    
    \item Se crearán varias bases de datos propias de conducción autónoma con un algoritmo de pilotaje basado en visión sencillo, que permitan el entrenamiento supervisado de diferentes redes neuronales.
    
    \item Se realizará un estudio y se llevarán a cabo pruebas de diferentes modelos de redes neuronales convolucionales de clasificación que se pueden emplear para la conducción autónoma. 
    
    \item Se realizará un estudio y se llevarán a cabo pruebas de diferentes modelos de redes neuronales convolucionales y redes neuronales recurrentes de regresión que se pueden emplear para la conducción autónoma. 

\end{enumerate}


\section{Requisitos}

El proyecto se desarrollará basándose en los subobjetivos mencionados anteriormente y tendrá que ajustarse a los requisitos de partida del proyecto: 

\begin{enumerate}
    \item La simulación se realizará en el simulador Gazebo, en concreto en la versión 7.15.0. El modelo de coche empleado es el modelo f1ROS (posee una cámara como sensor) creado por la organización JdeRobot \footnote{\url{https://jderobot.org/Main_Page}}. Este modelo se encuentra disponible en el repositorio de Github JdeRobot-assets \footnote{\url{https://github.com/JdeRobot/assets}}.

    \item Se empleará el \textit{middleware} robótico \acrshort{ros}, en concreto en la versión \textit{ROS Kinetic}. Este \textit{middleware} que simplifica el desarrollo de software robótico se explicará en mayor detalle en el Capítulo~\ref{cap.estado}. 

    \item El sistema operativo que se empleará en este proyecto será Ubuntu 16.04.
    
    \item El lenguaje de desarollo empleado será Python. Debido a la compatilibidad con el \textit{middleware} ROS Kinetic no se ha empleado Python-3.X, sino que se utiliza Python-2.7.
    
    \item Se hará uso de la API de redes neuronales Keras, escrita en Python y capaz de ejecutarse sobre TensorFlow, CNTK o Theano. En este proyecto se ejecutará sobre TensorFlow y se empleará la versión 2.2.4.
    
    \item Las soluciones deben ser ágiles. Los algoritmos propuestos no pueden detenerse demasiado tiempo a pensar cuál será el próximo movimiento del vehículo, porque debe reaccionar rápido, en tiempo real y con movimientos suaves.
    
\end{enumerate}


\section{Metodología}

El desarrollo del proyecto se ha realizado siguiendo una metodología iterativa, donde cada iteración está compuesta por varias fases: determinar objetivos, planificación, diseño e implementación, análisis de riesgos, además de reuniones periódicas con los tutores.\\

Se ha decidido seguir el modelo de desarrollo en espiral, creado por Barry Boehm~\cite{modelo_espiral}~\cite{modelo_espiral1}~\cite{modelo_espiral2}. Este modelo se adapta perfectamente a este tipo de proyectos, ya que permite separar el comportamiento final en varias subtareas más sencillas y después juntarlas. Además, el modelo permite una gran flexibilidad ante cambios en los requisitos, algo muy común en estos proyectos.\\

Este modelo de ciclo de vida permite obtener prototipos funcionales poco a poco, a la vez que se realiza el desarrollo del producto de forma incremental. El modelo consta de diferentes iteraciones, también conocidas como ciclos. En cada ciclo existen cuatro fases bien diferenciadas: (1) Se concretan los objetivos específicos que deben cumplirse para que el ciclo actual se considere terminado en función de los objetivos finales; (2) se realiza un análisis detallado de cada posible riesgo que pueda tener el objetivo definido y se planean estrategias alternativas; (3) se desarrolla el producto y se realizan las pruebas necesarias; (4) se analizan los resultados obtenidos a través de las pruebas, y se planifica la siguiente iteración.\\


\begin{figure}[H]
  \begin{center}
    \includegraphics[width=0.4\textwidth]{figures/Objetivos/espiral.png}
		\caption{Modelo en espiral}
		\label{fig.espiral}
		\end{center}
\end{figure}

Esta metodología se ha llevado a cabo mediante reuniones semanales con los tutores en las que se analizaban los resultados de cada iteración, y en función de los resultados se fijaban nuevos objetivos. Además, en estas reuniones se analizaban los posibles fallos y se resolvían las dudas que iban surgiendo.\\


El código desarrollado semanalmente se ha subido al repositorio propio público de Github \footnote{\url{https://github.com/RoboticsURJC-students/2017-tfm-vanessa-fernandez}}, que emplea el sistema de control de versiones. Además, se ha desarrollado una bitácora en la página de JdeRobot \footnote{\url{https://jderobot.org/Vmartinezf-tfm}}, donde semanalmente se han explicado los avances y se han mostrado los resultados mediante imágenes y vídeos.\\

El resultado del TFM, las diferentes redes neuronales desarrolladas, se encuentran disponibles en el repositorio Github como software libre.



\section{Plan de trabajo}

Las etapas en las que se divide el proyecto, que se corresponden con el modelo en espiral, son:

\begin{itemize}
    \item Familiarización con la API de redes neuronales Keras y estudio de diferentes soluciones de aprendizaje extremo a extremo para conducción autónoma. En esta etapa se ha descargado e instalado Keras, así como todo el software necesario para desarrollar el proyecto. Además, se ha estudiado la creación de redes neuronales convolucionales en Keras, y su uso en algunos proyectos de la organización JdeRobot.
    
    \item Desarrollo de la infraestructura necesaria en Gazebo, y de una aplicación de control visual que permita la conducción del coche integrando una red entrenada en Keras.
    
    \item Creación de una base de datos que permita entrenar una red neuronal con la información visual del coche y los datos de velocidad.
    
    \item Estudio y mejora de redes neuronales convolucionales de clasificación aplicadas a la conducción autónoma. Se realizarán múltiples pruebas para tratar de conseguir la red más robusta posible y emplearla en la aplicación desarrollada.
    
    \item Estudio y mejora de redes neuronales convolucionales y recurrentes de regresión aplicadas a la conducción autónoma. Se realizarán diversas pruebas para intentar conseguir la red más robusta posible y emplearla en la aplicación desarrollada.
\end{itemize}


\lhead[]{CAPÍTULO \thechapter. ESTADO DEL ARTE}
\chapter{Estado del arte}\label{cap.estado}

En este capítulo se presenta el estado del arte sobre conducción autónoma mediante \acrfull{cnn} y \acrfull{rnn}. Se describirán diferentes bases de datos para conducción autónoma como Comma.ai ~\cite{comma} o Udacity ~\cite{udacity-data}, así como diferentes arquitecturas de redes neuronales empleadas para el mismo problema como pueden ser \textit{PilotNet} ~\cite{explaining-end2end} o \textit{ControlNet} ~\cite{reactive-ground}. Además, se describirán diferentes simuladores para conducción autónoma.\\

Se presentan también los diferentes ingredientes software en los que nos hemos apoyado para desarrollar el trabajo. Tales como el simulador Gazebo, el entorno JdeRobot, la librería OpenCV, PyQt, Python como lenguaje de programación, Keras como entorno para el desarrollo de redes neuronales y HDF5 como formato de archivo para guardar los modelos de redes neuronales.\\


\section{Bases de datos para conducción autónoma}

La conducción autónoma pretende que un vehículo sea capaz de conducir solo en base a los datos proporcionados por determinados sensores. En concreto, la cámara es el sensor más empleado en las diferentes redes neuronales que se mencionarán en el estado del arte. Dado que queremos que el vehículo sea capaz de conducir bajo diferentes circunstancias, en distintos entornos y diferentes iluminaciones, necesitaremos entrenar el modelo con un conjunto de imágenes representativo. Por ello, a lo largo de los últimos años han surgido diferentes \textit{datasets} con el fin de solucionar este problema. A continuación, se exponen algunos ejemplos de bases de datos empleadas para este propósito.

\subsection{Comma.ai}

La \textit{startup} de conducción autónoma Comma.ai \footnote{https://comma.ai/} creó en 2016 un conjunto de datos ~\cite{comma} que permite probar modelos para controlar un vehículo autónomo. Este conjunto de datos consta de 11 videoclips grabados a 20 Hz por una cámara \textit{Point Grey} colocada en el parabrisas de un \textit{Acura ILX} 2016. El conjunto de datos es un archivo zip comprimido que ocupa un total de 45 GB.\\

Este conjunto de datos consta de un total de 7.25 horas de datos de conducción, donde los fotogramas de vídeo tienen un tamaño de 160 x 320 píxeles. Junto a los archivos de vídeo se proporciona un conjunto de medidas de sensores donde se registran medidas como la velocidad, la aceleración, el ángulo de giro, la ubicación del GPS y los ángulos del giroscopio.\\

Además registran los sellos temporales en los que se midieron estas medidas de los sensores y los \textit{time  stamps} en que se capturaron los fotogramas de la cámara. Los datos de los sensores se capturan en bruto y los fotogramas de la cámara se almacenan en archivos HDF5 para que sean fáciles de usar en el aprendizaje automático y el \textit{software} de control.

\subsection{Udacity}

Udacity posee un proyecto de código libre \footnote{https://github.com/udacity/self-driving-car/tree/master/datasets} para conducción autónoma. El proyecto ofrece ejemplos de grabaciones de datos de más de diez horas de conducción y conjuntos de datos anotados de conducción, donde los objetos en el vídeo han sido marcados con cuadros circundantes. Además de las herramientas de código abierto, Udacity publica desafíos de programación para promover el desarrollo del proyecto. \\

Inicialmente, Udacity \cite{udacity-data} poseía 40 GB de datos públicos con el fin de facilitar a las personas la construcción de modelos competitivos sin acceso al tipo de datos de conducción que Tesla o Google poseen. Sin embargo, debido a que los modelos de aprendizaje profundo necesitan muchos datos, la compañía publicó 183 GB adicionales de datos de conducción.\\

Actualmente el conjunto de datos de Udacity \cite{udacity-dataset} consta de 223 GB de datos. Estos datos fueron grabados durante más de 70 minutos de conducción en días soleados y nublados, repartidos en dos días en \textit{Mountain View}. Las imágenes fueron grabadas por tres cámaras frontales: izquierda, derecha y central. La variedad de imágenes aumentará la calidad de los resultados y proporcionará a los participantes datos más realistas para poder trabajar, ya que este conjunto de datos representa mejor los desafíos de la conducción en el mundo real y las condiciones variables de la carretera. Los datos almacenados constan de latitud, longitud, marcha, freno, aceleración, ángulos de dirección y velocidad.


\subsection{SAIC Dataset}

En el artículo \textit{End-to-end Multi-Modal Multi-Task Vehicle Control for Self-Driving Cars with Visual Perceptions} \cite{multi-modal} se creó un nuevo conjunto de datos, llamado SAIC, con el fin de obtener un conjunto de datos para pruebas reales de conducción.\\

El conjunto de datos incluye cinco horas de datos de conducción en el área norte de San José, principalmente en carreteras urbanas. Este conjunto contiene datos de conducción tanto de día como de noche. \\

El vehículo es conducido entre varios puntos y cada viaje entre los puntos tiene una duración de aproximadamente diez minutos. El estacionamiento, la espera en el semáforo y otras condiciones se consideran partes ruidosas y se filtran. Después de filtrar los vídeos ruidosos, los datos de dos horas se dividen en entrenamiento, validación y conjunto de test.\\

En la grabación del conjunto de datos se incluyen tres conductores para evitar sesgos hacia un comportamiento de conducción específico. De manera similar, se graban flujos de vídeo, valores de velocidad y direcciones. Las secuencias de vídeo contienen vídeos de una cámara frontal central y dos laterales con un \textit{frame rate} de 30 fotogramas por segundo.



\section{Simuladores para conducción autónoma}

Un vehículo es caro, lo que implica que muchas investigaciones sobre conducción autónoma solamente estén disponibles para centros de investigacón y corporaciones. Cuando se emplea un vehículo puede que algo falle al probarlo, pudiendo incluso romperse el vehículo. Hoy en día existen numerosos simuladores, lo que permite a cualquier persona crear, programar y probar infinidad de vehículos y escenarios de forma segura y económica. Algunos de los simuladores más empleados se explican a continuación.

\subsection{CARLA}

CARLA ~\cite{carla} ~\cite{carla-org} es un simulador de código abierto para la investigación de conducción autónoma. Se ha desarrollado desde cero para respaldar el desarrollo, el entrenamiento y la validación de sistemas de conducción autónomos. Además, admite diferentes conjuntos de sensores y condiciones ambientales.\\

CARLA (Figura \ref{fig.carla}) simula un mundo dinámico y proporciona una interfaz simple entre el mundo y un agente que interactúa con el mundo. Para llevar a cabo esta funcionalidad, CARLA está diseñado como un sistema cliente-servidor, donde el servidor ejecuta la simulación y renderiza la escena. La API del cliente se implementa en Python y es responsable de la interacción entre el agente autónomo y el servidor a través de \textit{sockets}. El cliente envía comandos y metacomandos al servidor y recibe las lecturas del sensor. Los comandos (dirección, aceleración y frenado) controlan el vehículo. Los metamandatos controlan el comportamiento del servidor y se utilizan para restablecer la simulación, cambiar las propiedades del entorno (condiciones climáticas, iluminación y densidad de automóviles y peatones) y modificar el conjunto de sensores. \\

CARLA presenta las siguientes características:

\begin{itemize}
\item Escalabilidad a través de una arquitectura multi-cliente servidor: varios clientes en el mismo nodo o en diferentes nodos pueden controlar diferentes actores.
\item Permite a los usuarios controlar todos los aspectos relacionados con la simulación (generación de tráfico, comportamientos de peatones, climas, sensores, etc).
\item  Los usuarios pueden configurar diversos conjuntos de sensores (LIDAR, cámaras, sensores de profundidad, \acrshort{gps}, etc).
\item Permite deshabilitar la representación para ofrecer una ejecución rápida de la simulación del tráfico y los comportamientos de la carretera para los que no se requieren gráficos.
\item Se pueden crear mapas siguiendo el estándar \textit{OpenDrive} a través de herramientas como \textit{RoadRunner}.
\item Los usuarios pueden definir diferentes situaciones de tráfico.
\item Integra el \textit{middleware} robótico \acrshort{ros}.
\end{itemize}

\begin{figure}
\begin{center}
	\includegraphics[width=0.5\textwidth]{figures/Estado_arte/carla.jpeg}
   \caption{Simulador CARLA.}
	\label{fig.carla}
\end{center}
\end{figure}


\subsection{Gazebo}

Gazebo \cite{gazebo1} (Figura \ref{fig.gazebo}) es un simulador 3D de código abierto distribuido bajo licencia Apache 2.0. Este simulador se ha utilizado en ámbitos de investigación en robótica e Inteligencia Artificial. Es capaz de simular robots, objetos y sensores en entornos complejos de interior y exterior. Posee gráficos de gran calidad y un robusto motor de físicas (masa del robot, rozamiento, inercia, amortiguamiento, etc.). Fue elegido para realizar el DARPA Robotics Challenge (2012-2015) y está mantenido por la Open Source Robotics Foundation (OSRF).\\

Los modelos de robots que se emplean en la simulación son creados mediante algún programa de modelado 3D (Blender, Sketchup, etc). Estos robots simulados necesitan ser dotados de inteligencia para lo cual se emplean \textit{plugins}. Estos \textit{plugins} pueden dotar al robot de inteligencia u ofrecer la información de sus sensores a aplicaciones externas y recibir de éstas comandos para los actuadores de los robots.

\begin{figure}
\begin{center}
	\includegraphics[width=0.5\textwidth]{figures/Estado_arte/gazebo.png}
   \caption{Simulador Gazebo.}
	\label{fig.gazebo}
\end{center}
\end{figure}


\subsection{Udacity's Self-Driving Car Simulator}

Udacity's Self-Driving Car Simulator \cite{udacity-data} \cite{udacity-sim} fue construido para Udacity's Self-Driving Car Nanodegree con el objetivo de que los estudiantes pudieran aprender cómo entrenar modelos de aprendizaje profundo que permitieran a los vehículos conducir de forma autónoma. Este simulador es de código abierto y requiere Unity.\\

El simulador de Udacity (Figura \ref{fig.udacity-sim}) permite al usuario seleccionar la escena deseada así como el modo de conducción en la pantalla principal. Existen dos modos de conducción: \textit{Training Mode} y \textit{Autonomus Mode}. En el modo \textit{Training Mode} el coche se conduce manualmente mediante el teclado o el ratón y se almacenan los datos de conducción y las imágenes de las cámaras que posee el vehículo. Los datos grabados con este modo se pueden emplear para entrenar un modelo de aprendizaje automático. En el modo \textit{Autonomous Mode} se puede probar el modelo de aprendizaje automático creado y comprobar su rendimiento en ejecución.\\

Técnicamente, el simulador actúa como un servidor al cual el programa puede conectarse y recibir un flujo de imágenes. Se puede crear un programa de Python que emplea un modelo de aprendizaje automático para procesar las imágenes de la carretera para predecir las mejores instrucciones de conducción y enviarlas de vuelta al servidor. Cada instrucción de conducción contiene un ángulo de dirección y un dato de aceleración que cambia la dirección y la velocidad del automóvil.

\begin{figure}
\begin{center}
	\includegraphics[width=0.5\textwidth]{figures/Estado_arte/udacity.png}
   \caption{Simulador Udacity's Self-Driving Car Simulator.}
	\label{fig.udacity-sim}
\end{center}
\end{figure}


\subsection{Deepdrive 2.0}

Deepdrive 2.0 \cite{deepdrive} es un simulador de código abierto para Linux y Windows. Los simuladores actuales parecen vincularse a un \textit{hardware} específico o no tienen forma de vincularse a vehículos físicos. Para conseguir este propóstio Deepdrive (Figura \ref{fig.deepdrive}) incluye una amplia gama de sensores, automóviles y entornos, y facilita la transferencia a vehículos reales. Esto permitirá que un mayor número de personas utilice únicamente el simulador para hacer pruebas constantes.\\

Presenta algunas características únicas respecto a otros simuladores de código abierto:

\begin{itemize}
\item El \textit{frame rate} es más elevado al emplear varias cámaras, ya que emplea memoria compartida en lugar de \textit{sockets} y transferencia asíncrona.
\item La superficie de la carretera no es plana, sino que incluye colinas, curvas y la anchura de la carretera varía.
\item El mapa, los automóviles, la iluminación, etc. son gratuitos y son modificables en Unreal.
\end{itemize}

\begin{figure}
\begin{center}
	\includegraphics[width=0.5\textwidth]{figures/Estado_arte/deepdrive.png}
   \caption{Simulador Deepdrive.}
	\label{fig.deepdrive}
\end{center}
\end{figure}


\section{Redes neuronales}

La conducción autónoma no es posible sin un algoritmo que tome decisiones. En algunos casos estos algoritmos pueden ser redes neuronales. En esta sección se describirán diferentes arquitecturas de redes neuronales empleadas en la conducción autónoma.


\subsection{Redes neuronales convolucionales}

El aprendizaje de extremo a extremo para conducción autónoma se ha explorado desde finales de los años ochenta. The Autonomous Land Vehicle in a Neural Network (ALVINN) \cite{alvinn} se desarrolló para aprender ángulos de dirección a partir de una cámara y las medidas proporcionadas por un láser mediante una red neuronal con una sola capa oculta. Basados en esta idea de redes de extremo a extremo (dada una imagen o imágenes se preciden ángulos de dirección), exiten mútiples aproximaciones \cite{road} \cite{end2end} \cite{interpretable} de las cuales veremos algunas a continuación.\\

Un buen ejemplo de red de extremo a extremo es la red PilotNet \cite{end2end} \cite{explaining-end2end} creada por Nvidia. En ``End to end learning for self-driving cars'' \cite{end2end} se describe dicha red con detalle. Es una red neuronal convolucional (CNN) que mapea píxeles en crudo de una sola cámara frontal a comandos de dirección directamente. El comando propuesto por la CNN se compara con el comando deseado para la imagen en concreto y los pesos de la red se van ajustando para aproximar la salida de la red a la salida deseada. El ajuste de los pesos se realiza empleando \textit{back propagation}.\\

La red PilotNet (Figura \ref{fig.pilotnet}) consta de 9 capas, que incluyen una capa de normalización, 5 capas convolucionales y 3 capas \textit{fully-connected}. La imagen de entrada se divide en planos YUV y se pasa a la red. Las capas convolucionales las diseñaron para realizar la extracción de características y las eligieron a través de experimentos que variaban las configuraciones de capas. Las dos primeras capas convolucionales usaban un \textit{stride} de 2x2 y un kernel 5x5, mientras que las 3 últimas capas usaban un \textit{non-stride} y un kernel 3x3. Las 3 capas \textit{fully-connected} fueron diseñadas para funcionar como un controlador de la dirección, pero no es posible saber exactamente qué partes de la red funcionan principalmente como extractor de características y cuáles sirven como controlador. El sistema aprende automáticamente las representaciones internas, como la detección de características útiles de la carretera.\\

\begin{figure}
\begin{center}
	\includegraphics[width=0.5\textwidth]{figures/Estado_arte/pilotnet.png}
   \caption{Arquitectura Pilotnet.}
	\label{fig.pilotnet}
\end{center}
\end{figure}

El objetivo de \cite{explaining-end2end} es explicar lo que PilotNet aprende y cómo toma sus decisiones. Con este fin, se desarrolló un método para determinar qué elementos en la imagen de la carretera influyen más en la decisión de la dirección de PilotNet. Llaman a estas secciones de imagen objetos salientes. Se puede encontrar un informe detallado del método de detección de saliencia en ``VisualBackProp: Efficient Visualization of CNNs for Autonomous Driving'' \cite{visual}.\\


La idea central de ``Explaining  how  a  deep  neural network trained with end-to-end learning steers a car'' \cite{explaining-end2end} para discernir los objetos salientes es encontrar partes de la imagen que corresponden a ubicaciones donde los mapas de características tienen las mejores activaciones. Las activaciones de los mapas de nivel superior se convierten en máscaras para las activaciones de niveles inferiores utilizando el siguiente algoritmo:

\begin{enumerate}
    \item En cada capa, las activaciones de los mapas de características se promedian.
    \item El mapa con el promedio más alto se escala según el tamaño del mapa de la capa de abajo. El aumento de escala se realiza mediante una deconvolución. Los parámetros (\textit{filter size} y \textit{stride}) utilizados para la deconvolución son los mismos que se emplearon en la capa convolucional utilizada para generar el mapa. Los pesos de la deconvolución se establecen en 1.0 y los sesgos en 0.0.
    \item El mapa promediado aumentado de un nivel superior se multiplica después con el mapa promediado de la capa de abajo (ahora son del mismo tamaño). El resultado es una máscara de tamaño intermedio.
    \item La máscara intermedia se escala al tamaño de los mapas de la capa inferior de la misma manera que en el paso 2.
    \item El mapa intermedio mejorado se multiplica de nuevo con el mapa promediado de la capa de abajo. Se obtiene una nueva máscara intermedia.
    \item Los pasos 4 y 5 se repiten hasta que se alcanza la entrada. La última máscara que es del tamaño de la imagen de entrada se normaliza al rango 0-1 y se convierte en la máscara de visualización final.
\end{enumerate}

Esta máscara de visualización muestra qué regiones de la imagen de entrada contribuyen más a la salida de la red. Estas regiones identifican los objetos salientes. En la Figura \ref{fig.salient} se pueden ver ejemplos de objetos salientes para varias imágenes de entrada.\\

\begin{figure}
\begin{center}
	\includegraphics[width=0.4\textwidth]{figures/Estado_arte/saliencia.png}
   \caption{Ejemplos de objetos salientes para varias imágenes de entrada.}
	\label{fig.salient}
\end{center}
\end{figure}

Los resultados muestran que PilotNet aprende a reconocer objetos relevantes en la carretera y que es capaz de mantener el vehículo en el carril con éxito en una amplia variedad de condiciones, independientemente de si las marcas del carril están presentes en la carretera o no.\\


En ``Self-driving a Car in Simulation Through a CNN'' \cite{self-driving} se propone una nueva arquitectura de red, llamada TinyPilotnet, que se deriva de la red Pilotnet \cite{end2end} \cite{explaining-end2end}. La red TinyPilotnet (Figura \ref{fig.tinypilotnet}) está compuesta por una capa de entrada, en la que se introducirán imágenes de resolución 16x32 y un único canal, seguida por dos capas convolucionales de kernel 3x3, y una capa \textit{dropout} configurada al 50\% de probabilidad para agilizar el entrenamiento. Finalmente, el tensor de información se convierte en un vector que es conectado a dos capas \textit{fully-connected} que conducen a un par de neuronas, cada una de ellas dedicada a predecir los valores de dirección y aceleración respectivamente. La imagen de entrada tiene un solo canal formado por el canal de saturación del espacio de color HSV.\\

\begin{figure}
\begin{center}
	\includegraphics[width=0.5\textwidth]{figures/Estado_arte/tinypilotnet.png}
   \caption{Arquitectura TinyPilotnet.}
	\label{fig.tinypilotnet}
\end{center}
\end{figure}

En ``Event-based  vision  meets  deep  learning  on  steering  prediction  forself-driving  cars'' \cite{event} se presenta un enfoque de red neuronal profunda que emplea cámaras de eventos (sensores de inspiración biológica que no adquieren imágenes completas a una velocidad de \textit{frames} fija, sino que tienen píxeles independientes que sólo producen cambios de intensidad de forma asíncrona en el momento en el que ocurren) para predecir el ángulo de giro de un vehículo. Los eventos se convierten en fotogramas de eventos por acumulación de píxeles en un intervalo de tiempo constante. Posteriormente, una red neuronal profunda los asigna a los ángulos de dirección.\\

En este artículo inicialmente apilan los fotogramas de eventos de diferente polaridad, creando una imagen de eventos 2D. Después, implementan una serie de arquitecturas ResNet, es decir, ResNet18 y ResNet50. Estas redes son utilizadas como extractores de características para el problema de regresión, considerando solo las capas convolucionales. Para codificar las características de la imagen extraídas de la última capa convolucional en un descriptor vectorizado, se emplea una capa \textit{global average pooling} que devuelve la media del canal de las características. Después se agrega una capa \textit{fully-connected} (con dimensionalidad 256 para ResNet18 y 1024 para ResNet50), seguida de una ReLU no lineal y una capa \textit{fully-connected} unidimensional para generar el ángulo.\\

En este artículo \cite{event} se preciden ángulos empleando 3 tipos de entradas: 1. imágenes en escala de grises, 2. diferencia de imágenes en escala de grises, 3. imágenes creadas por la acumulación de eventos. Analizan el rendimiento de la red en función del tiempo de integración utilizado para generar las imágenes de eventos (10, 25, 50, 100 y 200 ms). Cuanto mayor es el tiempo de integración, mayor es la traza de eventos que aparecen en los contornos de los objetos. La red funciona mejor cuando se entrena con imágenes de eventos correspondientes a 50 ms, y el rendimiento se degrada para tiempos de integración cada vez más grandes. Uno de los problemas que presentan las entradas que emplean imágenes en escala de grises es que a altas velocidades las imágenes se difuminan y la diferencia de imágenes se vuelve muy ruidosa.\\

En el artículo ``From Pixels to Actions: Learning to Drive a Car with Deep Neural Networks'' \cite{pixels} se realiza un amplio estudio donde se analiza una red neuronal de extremo a extremo para predecir las acciones de dirección de un vehículo en base a las imágenes de una cámara, así como las dependencias temporales de entradas consecutivas y la diferencia entre redes de clasificación y redes de regresión.\\

La arquitectura principal que emplean es una variación de la arquitectura PilotNet, AlexNet o VGG19. Para AlexNet se elimina el \textit{dropout} de las 2 capas densas finales y se reduce el tamaño de 500 y 200 neuronas. La capa de salida de la red depende de su tipo (regresión o clasificación) y para una red de clasificación del número de clases. Para el caso de clasificación, cuantifican las medidas del ángulo de dirección en valores discretos, que representan las etiquetas de la clase. Esta cuantificación es necesaria como entrada cuando se tiene una red de clasificación y permite equilibrar los datos a través de los pesos de la muestra. Esta ponderación actúa como un coeficiente para la tasa de aprendizaje de la red para cada muestra. El peso de una muestra está directamente relacionado con la clase a la que pertenece cuando se cuantifica. La ponderación de muestra se realiza para regresión y clasificación.\\

Se estudia la influencia de las especificaciones de cuantización de clase en el rendimiento del sistema. Estas especificaciones consisten en la cantidad de clases y la asignación del rango de entrada de estas clases. Se comparan redes con diferentes grados de granularidad, lo que influye en el rendimiento. Se compara un esquema de cuantificación de grano grueso de 7 clases con uno de grano fino de 17 clases, obteniendo mejores resultados con el de grano grueso.\\

Además, en este artículo se evalúan métodos que permiten que el sistema aproveche la información de entradas consecutivas: un método que sigue una arquitectura de extremo a extremo y un método que emplea capas recurrentes (lo veremos en la siguiente subsección). \\

El método que emplea una CNN para la predicción, que llaman \textit{stacked frames}, concatena varias imágenes de entrada consecutivas para crear una imagen apilada. La entrada a la red es esta imagen apilada (para la imagen t se concatenan las imágenes t-1, t-2, etc). El tamaño de entrada será la única variable que se modique, es decir, no se modifica la red. Por esta razón, las imágenes se concatenan en la dimensión de profundidad (canal) y no en una nueva dimensión. Por ejemplo, apilar 2 imágenes anteriores a la imagen RGB actual de 160 x 320 x 3 cambaría su tamaño a 160 x 320 x 9. Los resultados muestran un aumento en el rendimiento de las métricas con este método. Se cree que es debido a que la red puede hacer una predicción basada en la información promedio de múltiples imágenes. Para una sola imagen, el valor predicho puede ser o muy alto o muy bajo. En cambio, para imágenes concatenadas, la información combinada podría cancelarse entre sí, dando una mejor predicción promedio. Suponiendo que la red promedie la información, aumentar el número de imágenes podría hacer que la red perdiera la capacidad de respuesta. Por ello emplean 3 fotogramas concatenados.\\

Además, en este artículo se demuestra cualitativamente que las métricas estándar que se emplean para evaluar redes no necesariamente reflejan con precisión el comportamiento de conducción de un sistema. Una matriz de confusión prometedora puede dar como resultado un comportamiento de conducción deficiente, mientras que una matriz con mal aspecto puede dar como resultado un buen comportamiento de conducción.



\subsection{Redes neuronales recurrentes}


Las redes neuronales recurrentes (RNNs) representan una clase de redes neuronales artificiales que utilizan células de memoria para modelar la relación temporal entre los datos de entrada y, por lo tanto, aprender la dinámica subyacente. Con la introducción de las Long Short-Term Memory (LSTM), el modelado de relaciones a largo plazo se hizo posible dentro de RNN.\\

En múltiples investigaciones sobre conducción autónoma se ha aprovechado la capacidad de estas redes para aprovechar la información de imágenes consecutivas. Algunas de estas investigaciones las veremos a continuación.\\

Un ejemplo de investigación donde se emplean capas LSTM es la propuesta por ``Reactive  ground  vehicle  control  via  deep networks'' \cite{reactive-ground}. En esta investigación se presenta un controlador reactivo basado en aprendizaje profundo que emplea una arquitectura de red simple que requiere pocas imágenes de entrenamiento. A pesar de esta estructura simple, su arquitectura de red, llamada ControlNet, supera a otras redes más complejas en múltiples entornos (entornos interiores estructurados y entornos exteriores no estructurados) utilizando diferentes plataformas robóticas. Es decir, el artículo se centra en el control reactivo, donde el robot debe evitar obstáculos que no están presentes durante la construcción del mapa.\\

ControlNet extrae imágenes RGB para generar comandos de control: gira a la derecha, gira a la izquierda y recto. La arquitectura de ControlNet consiste en alternar capas convolucionales con capas de \textit{maxpooling} seguidas de capas \textit{fully-connected}. Las capas convolucionales y la de \textit{pooling} extraen información geométrica sobre el medio ambiente, mientras que las capas \textit{fully-connected} actúan como un clasificador general. La capa LSTM permite al robot incorporar información temporal permitiéndole continuar moviéndose en la misma dirección sobre varios fotogramas. La estructura de ControlNet (Figura \ref{fig.controlnet}) es:

\begin{itemize}
    \item 2D Convolution, 16 filtros de tamaño 10x10
    \item Max Pooling, filtro de 3x3, \textit{stride} de 2
    \item 2D Convolution, 16 filtros de tamaño 5x5
    \item Max Pooling, filtro de 3x3, \textit{stride} de 2
    \item 2D Convolution, 16 filtros de tamaño 5x5
    \item Max Pooling, filtro de 3x3, \textit{stride} de 2
    \item 2D Convolution, 16 filtros de tamaño 5x5
    \item Max Pooling, filtro de 3x3, \textit{stride} de 2
    \item 2D Convolution, 16 filtros de tamaño 5x5
    \item Max Pooling, filtro de 3x3, \textit{stride} de 2
    \item Fully connected, 50 neuronas
    \item ReLu
    \item Fully connected, 50 neuronas
    \item LSTM (5 frames)
    \item Softmax con 3 salidas
\end{itemize}

\begin{figure}
\begin{center}
	\includegraphics[width=1\textwidth]{figures/Estado_arte/controlnet.png}
   \caption{Estructura de red ControlNet.}
	\label{fig.controlnet}
\end{center}
\end{figure}

En ``End-to-end deep learning for steering autonomous vehicles considering temporal dependencies'' \cite{temporal-dependencies} se propone una Convolutional Long Short-Term Memory Recurrent Neural Networks, conocida como C-LSTM (Figura \ref{fig.clstm}), que es entrenable de extremo a extremo, para aprender las dependencias visual y temporal dinámica de la conducción. El sistema investigado está compuesto por una cámara RGB frontal y una red neuronal que consta de una CNN y LSTM que estiman el ángulo del volante en función de la entrada de la cámara. Las imágenes de la cámara se procesan fotograma a fotogama por la CNN. Las características resultantes luego se procesan dentro de la red LSTM para aprender las dependencias temporales. La predicción del ángulo de dirección se calcula a través de la capa de clasificación de salida después de las capas LSTM.\\

Aplican el concepto de \textit{transfer learning}. La CNN está pre-entrenada en el conjunto de datos Imagenet. Luego transfieren la red neuronal entrenada a otra específica enfocada en imágenes de conducción. Posteriormente, en la LSTM se procesa una secuencia de vectores de características de longitud fija w de la CNN. A su vez, las capas LSTM aprenden a reconocer las dependencias temporales que conducen a una decisión de dirección Yt basada en las entradas de Xt-w a Xt. Los valores pequeños de t conducen a reacciones más rápidas, pero la red aprende solo las dependencias a corto plazo y la susceptibilidad a los aumentos de fotogramas mal clasificados individualmente. Mientras que los valores elevados de t conducen a un comportamiento más suave y, por tanto, predicciones de dirección más estables, pero aumenta las posibilidades de aprender dependencias erróneas a largo plazo.\\

El concepto de ventana deslizante permite a la red aprender a reconocer diferentes ángulos de dirección desde el mismo fotograma Xi pero en diferentes estados temporales de las capas LSTM. Tanto los pesos de la LSTM como de la CNN se comparten en diferentes pasos dentro de la ventana deslizante y, esto permite un tamaño de ventana arbitrariamente largo.\\

Plantean la regresión del ángulo de dirección como un problema de clasificación. Esta es la razón por la que el único número que representa el ángulo de dirección Yt está codificado como un vector de activaciones de las neuronas de la capa de clasificación. Utilizan una capa totalmente conectada con activaciones \textit{tanh} para la capa de clasificación.\\

En esta propuesta para el entrenamiento de dominio ``específico'', la capa de clasificación de la CNN se reinicializa y se entrena con los datos de carretera de la cámara. El entrenamiento de la capa LSTM se lleva a cabo de manera múltiple, la red aprende las decisiones de dirección que están asociadas con los intervalos de conducción. La capa de clasificación y las capas LSTM emplean una mayor velocidad de aprendizaje porque se inicializan con valores aleatorios. La CNN y la LSTM se entrenan conjuntamente al mismo tiempo.\\

\begin{figure}
\begin{center}
	\includegraphics[width=0.5\textwidth]{figures/Estado_arte/clstm.png}
   \caption{Arquitectura C-LSTM.}
	\label{fig.clstm}
\end{center}
\end{figure}

En ``Deep  steering:  Learning  end-to-end  driving  model  from spatial and temporal visual cues'' \cite{deep-steering} se propone un modelo basado en visión que mapea imágenes de entrada en ángulos de dirección usando redes profundas. Se segmenta la red en subredes. Es decir, los fotogramas se introducen primero en una red de extracción de características, generando una representación de características de longitud fija que modela el entorno visual y el estado interno de un vehículo. Las características extraídas se envían a una red de predicción de dirección. En la subred de extracción de características emplea una Spatio-Temporal Convolution (ST-Conv) que cambia las dimensiones temporales y espaciales. Se emplea una capa \textit{fully-connected} tras la ST-Conv para obtener un vector de características de dimensión 128. Además, en la subred de extracción de características se introducen capas LSTM, para lo cual se emplea ConvLSTM. La subred de predicción de dirección propone concatenar acciones de dirección y de estado del vehículo con el vector de características de 128 dimensiones. Para ello se añade un paso de recurrencia entre la salida final y las dos capas \textit{concat} justo antes/después de la LSTM. La capa \textit{concat} antes de la LSTM agrega la velocidad, y el par de torsión y ángulo de rueda al vector de 128 dimensiones, formando un vector de 131 dimensiones. La capa \textit{concat} después de LSTM está compuesta por un vector de características 128-d + salida de LSTM 64-d + salida final previa 3d.\\

En ``Interpretable  learning  for  self-driving  cars  by visualizing  causal  attention'' \cite{interpretable} se propone un modelo de atención visual para entrenar una red convolucional de extremo a extremo desde las imágenes hasta el ángulo de giro. El modelo de atención resalta las regiones de imagen que potencialmente influyen en la salida de la red, de las cuales algunas son influencias reales y otras espúreas. Su modelo predice comandos de ángulo de dirección continuos a partir de píxeles en bruto. El modelo predice el radio de giro inverso ût, pero se relaciona con el comando de ángulo de dirección mediante geometría de Ackermann. \\

En este método emplean una red neuronal convolucional para extraer un conjunto de vectores de características visuales codificadas, a las que se refieren como una característica convolucional cubo xt. Cada vector de características puede contener descripciones de objetos de alto nivel que permiten que el modelo de atención preste atención selectiva a ciertas partes de una imagen de entrada al elegir un subconjunto de vectores de características. Utilizan la red PilotNet \cite{end2end} para aprender un modelo de conducción, pero omiten las capas de \textit{maxpooling} para evitar la pérdida de información de ubicación espacial. Recopilan un cubo xt de características convolucionales tridimensionales de la última capa empujando la imagen preprocesada a través del modelo, y el cubo de características de salida se emplea como entrada de las capas LSTM. Utilizan una red LSTM que predice el radio de giro inverso y genera ponderaciones de atención en cada paso de tiempo t condicionado al estado oculto anterior y una característica convolucional actual xt. Asumen una capa oculta condicionada al estado oculto anterior y los vectores de características actuales. El peso de atención para cada ubicación espacial se calcula luego mediante una función de regresión logística multinomial.\\

El último paso de este método es un decodificador de grano fino en el que refinan un mapa de atención visual y detectan saliencias visuales locales. Aunque un mapa de atención del decodificador de grano grueso proporciona una probabilidad de importancia sobre un espacio de imagen 2D, el modelo debe determinar regiones específicas que causan un efecto casual en el rendimiento de la predicción. Obtienen una disminución en el rendimiento cuando se oculta una prominencia visual local en una imagen de entrada en bruto. En primer lugar, recopilan un conjunto consecutivo de pesos de atención e ingresan imágenes en bruto para los T pasos de  tiempo especificados por el usuario. Luego, crean un mapa de atención, Mt. La red neuronal de 5 capas (basada en PilotNet) emplea una pila de filtros 5x5 y 3x3 sin ninguna capa \textit{pooling}, y por tanto la imagen de dimensiones 80x160 se procesa para producir un cubo de características 10x20x64, conservando su relación de aspecto. Para extraer una prominencia visual local, primero muestrean aleatoriamente partículas de 2D con reemplazo sobre una imagen de entrada condicionada en el mapa de atención Mt. También emplean el eje de tiempo como la tercera dimensión para considerar las características temporales de las saliencias visuales, almacenando partículas espacio temporales 3D. Posteriormente, aplican un algoritmo de \textit{clustering} (DBSCAN) para encontrar una prominencia visual local agrupando las partículas 3D en \textit{clusters}. Para los puntos de cada grupo y cada fotograma de tiempo t, calculan el algoritmo \textit{convex hull} para encontrar una región local de cada prominencia visual destacada.\\

En ``From pixels to actions: Learning to drive a car with deep neural networks'' \cite{pixels} además de aprovechar la información temporal concatenando fotogramas se estudia la inclusión de capas recurrentes. Es decir, modifican su arquitectura para incluir capas LSTM, que permiten capturar información temporal entre entradas consecutivas. Las redes se entrenan con un vector de entrada que consiste en la imagen de entrada y una serie de imágenes anteriores, lo que se traduce en una ventana de tiempo. Comparan muchas variaciones de la arquitectura PilotNet \cite{end2end}: (1) se cambia una o dos capas densas a capas LSTM, (2) se agrega una capa LSTM después de las capas densas, y (3) se cambia la capa de salida a LSTM. Todos los experimentos que realizan con redes LSTM demostraron que la incorporación de capas LSTM no aumentó ni redujo el rendimiento de la red.\\

\begin{figure}
\begin{center}
	\includegraphics[width=0.5\textwidth]{figures/Estado_arte/deepestlstm.png}
   \caption{Arquitectura DeepestLSTM-TinyPilotnet.}
	\label{fig.deepestlstm}
\end{center}
\end{figure}

En ``Self-driving  a  Car  in  Simulation  Through  a  CNN'' \cite{self-driving} se propone una nueva arquitectura basada en la arquitectura TinyPilotNet (Figura \ref{fig.tinypilotnet}) para mejorar el rendimiento de la misma. Esta nueva red (Figura \ref{fig.deepestlstm}) está formada principalmente por 3 capas convolucionales de kernel 3x3, combinadas con capas \textit{maxpooling}, seguidas por 3 capas LSTM convolucionales 5x5 y 2 capas \textit{fully-connected}. Las capas LSTM producen un efecto de memoria, por lo que los ángulos de dirección y los valores de aceleración dados por la CNN están influenciados por los anteriores. \\


\section{Infraestructura empleada}

En esta sección se explican los ingredientes software empleados para desarrollar este proyecto. Se describirá desde el lenguaje empleado hasta diferentes herramientas software que han sido necesarias.

\subsection{Simulador Gazebo}

Como hemos visto al hablar de simuladores, Gazebo \footnote{\url{http://gazebosim.org/}} es un programa de código abierto distribuido bajo licencia Apache 2.0. Se emplea en el desarrollo de aplicaciones robóticas e inteligencia artificial. Es capaz de simular robots, objetos y sensores en entornos complejos de interior y exterior. Tiene gráficas de gran calidad y un robusto motor de físicas (masa del robot, rozamiento, inercia, amortiguamiento, etc.).\\

En este trabajo se emplea la versión 7.15.0 de Gazebo. Gracias a él se pueden incluir texturas, luces y sombras en los escenarios, así como simular la física como por ejemplo choques, empujes, gravedad, etc. Además, incluye diversos sensores, como pueden ser cámaras y láseres, los cuales podrán ser incorporados en los robots que empleemos. Todo ello hace que sea una herramienta muy potente y de gran ayuda en robótica.\\

\begin{figure}
\begin{center}
	\includegraphics[width=0.5\textwidth]{figures/Estado_arte/gazebo2.png}
   \caption{Simulador Gazebo.}
	\label{fig.gazebo1  }
\end{center}
\end{figure}

Los mundos simulados con Gazebo son mundos 3D, que se cargan a partir de ficheros con extensión ``.world''. Son ficheros \acrfull{xml} definidos en el lenguaje \acrfull{sdf}. Cada uno de estos ficheros contiene una descripción completa de todos los elementos que forman el mundo y los robots, incluyendo:


\begin{itemize}
\item Escena: Luz ambiente, propiedades del cielo, sombras, etc.
\item Mundo: Representa el mundo como un conjunto de modelos, \textit{plugins} y propiedades físicas.
\item Modelo: Articulaciones, objetos de colisión, sensores, etc.
\item Físicas: Gravedad, motor físico, paso del tiempo, colisiones, inercias, etc.
\item Plugins: Sobre un mundo, modelo o sensor.
\item Luz: Los puntos y origen de la luz.
\end{itemize}

Las etiquetas empleadas en el fichero para representar estos elementos son: Scene, World, Model, Physics, Plugin, y Light.\\

Los modelos de robots que se emplean en la simulación son creados mediante algún programa de modelado 3D (Blender, Sketchup…). Estos robots simulados necesitan ser dotados de inteligencia para lo cual se emplean los \textit{plugins}. Estos \textit{plugins} pueden dotar al robot de comportamiento u ofrecer la información de sus sensores a aplicaciones externas y recibir de éstas comandos para los actuadores de los robots.

\subsection{Entorno ROS}

\acrfull{ros} \footnote{https://www.ros.org/} \cite{middleware1} es una plataforma de software libre para el desarrollo de software de robots, que provee servicios estándar de un sistema operativo como la abstracción del harware, el control de dispositivos de bajo nivel, mecanismos de intercambio de mensajes entre procesos y un conjunto de herramientas ampliamente utilizadas en robótica. Esta plataforma es de código abierto y se distribuye bajo licencia BSD.\\

Uno de los grandes beneficios del uso de \acrshort{ros} es la integración con el simulador Gazebo. Para realizar esta comunicación se hace uso de un conjunto de paquetes de \textit{ros} llamado gazebo\_ros\_pkgs \footnote{http://wiki.ros.org/gazebo\_ros\_pkgs} \cite{ros_integration}. Gazebo se integra con \acrshort{ros} mediante \textit{ROS Messages}, servicios y reconfiguración dinámica. En la Figura \ref{fig.ros} se puede observar una visión general de la interfaz gazebo\_ros\_pkgs.\\


\begin{figure}[H]
  \begin{center}
    \includegraphics[width=0.9\textwidth]{figures/Estado_arte/ros.png}
		\caption{Interfaz del conjunto de paquetes gazebo\_ros\_pkgs}
		\label{fig.ros}
		\end{center}
\end{figure}


\acrshort{ros} está formado por una colección de nodos o procesos que se combinan en un gráfico, y se comunican entre ellos mediante \textit{topics} de transimisón, servicios \acrshort{rpc}, y el Servidor de Parámetros. El sistema de control de un robot se compone de diferentes nodos, siendo mayor el número de nodos cuanta mayor sea la funcionalidad del robot. En \acrshort{ros} existen distintos nodos que controlan un láser, cámaras, motores de ruedas, odometría, etc. El uso de nodos de \acrshort{ros} en el robot permite localizar más fácilmente los fallos que puedan surgir, ya que cada fallo se concentra únicamente en un nodo.\\

Los \textit{topics} de \acrshort{ros} \footnote{http://wiki.ros.org/Topics} \cite{ros_book} son una forma de comunicación de los nodos. Los \textit{topics} también se conocen como buses sobre los cuales los nodos intercambian mensajes. Los \textit{topics} implementan un mencanismo de comunicación de publicación y/o suscripción. La semántica de publicación y/o suscripción de los \textit{topics} es anónima, lo que desacopla la producción de información de consumo. De esta forma los nodos no saben con quien se están comunicando. Además, los nodos que desean recibir mensajes sobre un \textit{topic} se deben suscribir a él para obtener la información que publique dicho \textit{topic}. Después de suscribirse, todos los mensajes sobre el \textit{topic} se envían al nodo que realizó la solicitud. Es posible que existan varios suscriptores del mismo \textit{topic}.\\

En \acrshort{ros} existen diversos \textit{plugins} \footnote{http://wiki.ros.org/gazebo\_plugins} que aportan una gran variedad de funcionalidad para los distintos modelos de robots de Gazebo. Algunos de los \textit{plugins} más destacados son libgazebo\_ros\_camera, que permite controlar una cámara; libgazebo\_ros\_laser, que controla un sensor láser; o libgazebo\_ros\_bumper que controla un sensor \textit{bumper} (sensor de contacto). Los \textit{plugins} libgazebo\_ros\_camera y  libgazebo\_ros\_laser serán empleados por el coche utilizado en este proyecto.



\subsection{Entorno JdeRobot}

JdeRobot \footnote{\url{http://jderobot.org/Main_Page}} es un \textit{middleware} de software libre para el desarrollo de aplicaciones con robots y visión artificial. Esta plataforma fue creada por el Grupo de Robótica de la Universidad Rey Juan Carlos en 2003 y está licenciada como GPLv3 \footnote{\url{https://www.gnu.org/licenses/quick-guide-gplv3.html}}.\\

Está desarrollado en C y C++, aunque contiene componentes desarrollados en lenguajes como Python y JavaScript. El entorno que ofrece está basado en componentes, los cuales se ejecutan como procesos. Dichos componentes interoperan entre sí a través del \textit{middleware} de comunicaciones \acrshort{ice} o de \acrshort{ros} messages. Tanto \acrshort{ice} como \acrshort{ros}-messages permiten la interoperación entre los componentes incluso estando desarrollados en diferentes lenguajes.\\

En este proyecto hemos empleado el \textit{driver} del coche de Fórmula1 (Figura \ref{fig.f1}) de la organización JdeRobot que está basado en \acrshort{ros}. En el \textit{driver} del F1 se han empleado \textit{plugins} de \acrshort{ros} con el fin de dotar de movimiento al modelo y captar las imágenes de la cámara del coche. En el desarrollo del proyecto de este TFM se empleará la versión 5.6.7 de JdeRobot, ya que es la última versión estable.


\begin{figure}
  \begin{center}
    \includegraphics[width=0.5\textwidth]{figures/Infraestructura/f1.png}
		\caption{Modelo f1ROS}
		\label{fig.f1}
		\end{center}
\end{figure}


\subsection{Lenguaje Python}

Python \footnote{\url{https://www.python.org/}} es un lenguaje de programación fácil de aprender y de alto nivel. Python incluye orientación a objetos, manejo de excepciones, listas, diccionarios, etc. Incluye módulos que permiten la entrada y salida de ficheros, \textit{sockets}, llamadas al sistema e incluso interfaces gráficas como Qt. Además, permite dividir el programa en módulos reutilizables y no es necesario compilarlo, pues es interpretado.\\

La última versión ofrecida por Python Software Foundation es la 3.7.3 , pero en nuestro caso se empleará la 2.7.12 por compatibilidad con  el \textit{middleware} ROS Kinetic. El código en el que está escrito la aplicación de control visual neuronal es Python.


\subsection{Biblioteca OpenCV}

OpenCV \footnote{\url{http://opencv.org/}} es una librería de código abierto desarrollada inicialmente por Intel y publicada bajo licencia de BSD. Esta librería implementa gran variedad de herramientas para la interpretación de la imagen. Sus siglas provienen de los términos anglosajones ``Open Source Computer Vision Library'', y está orientada a aplicaciones de visión por computador en tiempo real. \\

Esta librería puede ser usada en MacOS, Windows, Android y Linux, y existen versiones para C\#, Python y Java, a pesar de que originalmente era una librería en C/C++. Además, hay interfaces en desarrollo para Ruby, Matlab y otros lenguajes.\\

\begin{figure}[H]
  \begin{center}
    \includegraphics[width=0.6\textwidth]{figures/Estado_arte/opencv.png}
		\caption{Funciones de OpenCV}
		\label{fig.opencv}
		\end{center}
\end{figure}

OpenCV está compuesto por numerosas librerías con las cuales podemos manejar estructuras de datos, detectar bordes y esquinas, escalar o rotar imágenes, modificar el espacio de color de una imagen, realizar emparejamiento, detectar líneas y círculos, tratar objetos en 3D, crear ventanas y asociar eventos a dichas ventanas, etc. Incorpora funciones básicas para modelar el fondo, sustraer dicho fondo, generar imágenes de movimiento MHI (Motion History Images), etc. Además, incluye funciones para determinar dónde hubo movimiento y en qué dirección. \\

Desde su aparición OpenCV ha sido usado en numerosas aplicaciones. Hay una gran cantidad de empresas y centros de investigación que emplean estas técnicas como IBM, Microsoft, Intel, SONY, Siemens, Google, Stanford, MIT, CMU, Cambridge e INRIA.\\

En este trabajo se ha empleado la versión 3.3.1 de OpenCV en Python. Esta librería se empleará para realizar todo lo relacionado con el tratamiento de imágenes. Con ella se extraerán datos que puedan emplearse a la hora de tomar decisiones para que el coche funcione correctamente.


\subsection{Interfaces gráficas con PyQt}

PyQt \cite{pyqt} \cite{pyqt1} es un conjunto de enlaces Python para el conjunto de herramientas Qt, las cuales se emplean para el desarrollo de interfaces gráficas. Fue desarrollado por Riverbank Computing Ltd y es soportado por Windows, Linux, Mac OS/X, iOS y Android.\\

Qt es un entorno multiplataforma orientado a objetos desarrollado en C++ que permite desarrollar interfaces gráficas e incluye \textit{sockets}, hilos, Unicode, bases de datos SQL, etc. PyQt combina todas las ventajas de Qt y Python, pues permite emplear todas las funcionalidades ofrecidas por Qt con un lenguaje de programación tan sencillo como Python.\\

En este proyecto se ha empleado la versión 5 (en concreto la versión 5.5.1) de PyQt. PyQt5 es un conjunto de enlaces Python para Qt5, disponible en Python 2.x y 3.x. Tiene más de 620 clases y 6000 funciones y métodos. PyQt5 dispone de una licencia dual, es decir, los desarrolladores pueden elegir entre una licencia GPL (General Public Licence) o una licencia comercial. \\

La interfaz gráfica de la aplicación de control visual basado en \textit{deeplearning} creado en este proyecto está escrita usando PyQt. Las clases de PyQt5 se dividen en ciertos módulos, tales como QtCore, QtGui, QtWidgets, QtXml, QtSql, etc. 


\subsection{Middleware neuronal Keras}

Keras \footnote{https://keras.io/} es un \textit{middleware} de alto nivel para redes neuronales, escrito en Python y capaz de correr sobre las plataformas TensorFlow, CNTK, o Theano. Keras fue desarrollado con el fin de que la implementación de modelos de aprendizaje profundo fuera lo  más fácil y rápido posible para la investigación y el desarrollo.\\

Este entorno se ejecuta en Python 2.7-3.6, y es posible ejecutarlo tanto en CPU como en GPU. Keras se liberó bajo la licencia permisiva del MIT \cite{Keras_license}, y fue desarrollado y mantenido por François Chollet, un ingeniero de Google que utiliza cuatro principios:

\begin{itemize}
        \item Facilidad de uso: Keras es una \acrshort{api} diseñada básandose en la experiencia del usuario, es decir, ofrece un \acrshort{api} consistente y simple, proporciona comentarios claros y procesables en caso de error del usuario.
    
    \item Modularidad: Un modelo se entiende como una secuencia o un gráfico de módulos independientes, totalmente configurables, que se pueden conectar con la menor cantidad de restricciones posible. En concreto, las capas neuronales, las funciones de coste, los optimizadores, los esquemas de inicialización, las funciones de activación y los esquemas de regularización son módulos independientes que se pueden combinar para crear nuevos modelos.
    
    \item Fácil extensibilidad: Los nuevos módulos son fáciles de agregar, y los módulos existentes proporcionan amplios ejemplos. La posibilidad de crear fácilmente nuevos módulos permite una extensibilidad total, lo que hace que Keras sea adecuado para la investigación avanzada.
    
    \item Trabajo con Python: No hay archivos de configuración de modelos separados en un formato declarativo, sino que los modelos se describen en el código de Python, facilitando la depuración de código y permitiendo la extensibilidad.
\end{itemize}

La versión principal utilizada en este proyecto es Keras 2.2.4, y se ha ejecutado sobre TensorFlow. Keras ha sido empleado para entrenar e implementar diferentes arquitecturas de redes neuronales.\\

En las próximas subsecciones, se analizan los elementos principales que forman una red neuronal convolucional y una red neuronal recurrente (\acrshort{lstm}) construida con Keras.


\subsubsection{Modelos en Keras}

En Keras cada red neuronal se define como un modelo, es una forma de organizar las capas. La clase de modelo más simple es el modelo \textit{Sequential}, que es una pila lineal de capas. Es posible construir arquitecturas más complejas, aunque se debe utilizar la API funcional de Keras, que permite crear gráficos de capas arbitrarios.\\

Los modelos \textit{Sequential} tienen diferentes métodos, y algunos son imprescindibles para el proceso de aprendizaje, como son:

\begin{itemize}
    \item \textbf{.compile()}: Configura el modelo para entrenamiento. Los principales argumentos son los siguientes:
    
        \begin{itemize}
            \item \textit{optimizer}: Nombre del optimizador que actualizará los valores de los pesos durante el entrenamiento para minimizar la función de pérdida. Existen diferentes optimizadores como Adadelta, SGD, RMSProp, Adagrad, Adamax o Adam. En las diferentes redes implementadas se ha empleado el optimizador Adam \cite{adam}.
            
            \item \textit{loss}: Nombre de la función de coste que mide la diferencia entre la predicción y la etiqueta real. En este proyecto en las redes de clasificación se ha empleado \textit{categorical cross-entropy}, también conocida como \textit{log loss}. Esta función es muy utilizada en problemas de clasificación multiclase. Esta función devuelve la entropía cruzada entre una distribución aproximada q y una distribución verdadera p, y sigue la siguiente fórmula \cite{ops_theano}:
            
            \begin{equation}\label{eq:categorical_crossentropy}
		    H(p,q)=-\Sigma_{x}p(x)\log(q(x))
		    \end{equation}	
            
            En las redes neuronales de regresión se ha empleado como función de coste \textit{\acrfull{mse}}, que da una medida de cómo de lejos están las medidas predichas de las reales, pero acentúa los errores grandes. La fórmula de \acrshort{mse} es:
            
            \begin{equation}\label{eq:mse}
                \acrshort{mse} = \frac{1}{N}\sum_{j=1}^{N}(y_j - \hat{y}_j)^2
            \end{equation}
            
            Otra posible función a emplear es \textit{\acrfull{mae}}, que nos da una medida de cuán lejos están las medidas predichas de las medidas reales.
            
            \item \textit{metrics}: Nombre de las funciones que se emplean para medir el rendimiento del modelo durante el entrenamiento y el \textit{test}. En este proyecto las métricas empleadas son \textit{accuracy}, \acrshort{mse} y \acrshort{mae}. \textit{Accuracy} es el número de predicciones correctas realizadas por el modelo sobre todo tipo de predicciones realizadas en los modelos de clasificación. La fórmula para \textit{accuracy} es la siguiente:
            
            \begin{equation}\label{eq:accuracy}
                \textit{Accuracy} = \frac{TP + TN}{TP + FP + TN + FN}
            \end{equation}
            
            donde TP es \textit{True Positive} (casos en los que la clase real del dato es 1 y la clase predicha es 1), TN es \textit{True Negative} (la clase del dato es 0 y la predicha es 0), FP es \textit{False Positive} (la clase real es 0 y la clase predicha es 1), y FN es \textit{False Negative} (la clase real es 1 y la predicha es 0).\\
            
            \acrfull{mae}, como hemos dicho anteriormente es una medida del error de la predicción, y sigue la siguiente fórmula:
            
            \begin{equation}\label{eq:mae}
                \acrshort{mae} = \frac{1}{N}\sum_{j=1}^{N}|y_j - \hat{y}_j|
            \end{equation}
        \end{itemize}
    
    \item \textbf{.fit()}: Entrena el modelo para un número dado de épocas (iteraciones en un conjunto de datos). Los siguientes argumentos son necesarios:
    
        \begin{itemize}
            \item \textit{x}: Muestras de entrenamiento. Se debe definir como un \textit{Numpy array} o una lista de \textit{Numpy arrays}.
            
            \item \textit{y}: Etiquetas de entrenamiento. Se debe definir como un \textit{Numpy array} o una lista de \textit{Numpy arrays}.
            
            \item \textit{batch\_size}: Número de muestras que serán evaluadas antes de actualizar los pesos. Si no se especifica, \textit{batch\_size} será por defecto 32.
            
            \item \textit{epochs}: Número de iteraciones sobre todo el conjunto de datos.
            
            \item \textit{callbacks}: Lista de \textit{callbacks} (ver la subsección \ref{callback}) que se aplican durante el entrenamiento y la validación.
            
            \item \textit{validation\_split o validation\_data}: En Keras hay dos posibilidades para establecer el conjunto de validación: \textit{validation\_split} o \textit{validation\_data}. \textit{validation\_split} es la fracción de los datos de entrenamiento (número entre 0 y 1) que se utilizarán como datos de validación. \textit{validation\_data} es una tupla de valores sobre la cual se debe evaluar la pérdida y cualquier métrica del modelo al final de cada época. El modelo no tendrá en cuenta el conjunto de validación al entrenar el modelo.
            
            \item \textit{shuffle}: booleano que determina si se barajan los datos de entrenamiento o no. Si los datos no son barajados durante el entrenamiento las muestras de una misma clase pueden aparecer de forma consecutiva. En este caso, el modelo tendrá que aprender las características de una determinada clase. Cuando el modelo empieza a ver muestras de la siguiente clase, se ajusta a los nuevos datos y se olvida de la característica aprendida anteriormente. Si los datos están ordenados por clases, este proceso sigue y conduce a un peor resultado.
        \end{itemize}
    
    \item \textbf{.predict()}: Genera predicciones de salida para las muestras de entrada. 
    
    \item \textbf{.evaluate()}: Devuelve el valor de \textit{loss} y los valores de \textit{metrics} para el modelo en \textit{test}.
    
    \item \textbf{.save()}: Guarda un modelo en un solo archivo \acrfull{hdf5}, que contendrá la arquitectura del modelo, los pesos del modelo, la configuración de entrenamiento, y el estado del optimizador (permite reanudar el entrenamiento por donde se quedó).
    
    \item \textbf{.load\_model()}: Carga un modelo desde un archivo \acrshort{hdf5}.
\end{itemize}


\subsubsection{Capas en Keras}

Como hemos visto anteriormente, los modelos se componen de un conjunto de capas. Estas capas se añaden al modelo empleando el método \textit{.add()} de Keras. Dentro de este método se define el tipo de capa y los parámetros de cada una. Existen diferentes tipos de capas en Keras, pero solamente veremos las empleadas en el proyecto.

\begin{itemize}
    \item \textit{Convolutional layer}: Es la capa principal de una red \acrshort{cnn}, como vimos en la Sección \ref{tipos_capas}, donde se explica con detalle su funcionamiento. Keras proporciona distintos tipos de capas convolucionales en función de las dimensiones de los datos de entrada: \textit{Conv1D}, \textit{Conv2D}, y \textit{Conv3D}. En nuestro proyecto emplearemos la capa \textit{Conv2D}, ya que nuestros datos de entrada son imágenes.\\
    
    Los argumentos principales que hay que definir en una capa convolucional en Keras son:
    
        \begin{itemize}
            \item \textit{filters}: Número de filtros. Las capas \textit{Conv2D} intermedias aprenderán más filtros que las primeras capas \textit{Conv2D}, pero menos filtros que las capas más cercanas a la salida. 
            
            \item \textit{kernel\_size}: Especifica la anchura y altura de los filtros. Puede ser un solo entero para especificar el mismo valor para todas las dimensiones espaciales, o puede ser una tupla o lista de 2 enteros.
            
            \item \textit{strides}: Entero o tupla/lista de 2 enteros, que especifica cuántos píxeles debe desplazarse el filtro antes de aplicar la siguiente convolución. El valor por defecto es 1.
            
            \item \textit{padding}: Puede ser \textit{valid} o \textit{same}. Si se emplea \textit{valid}, no se aplica relleno, dando lugar a una salida con una dimensión más pequeña que la entrada. Sin embargo, si empleamos \textit{same}, la entrada se rellenará con ceros para dar lugar a una salida que conserve las dimensiones de la entrada. El valor por defecto es \textit{valid}.

        \end{itemize}
    
    \item \textit{BatchNormalization Layer}: Normaliza las activaciones de la capa anterior en cada lote, es decir, aplica una transformación que mantenga la activación media cerca de 0 y la desviación estándar de activación cerca de 1. El argumento más importante es \textit{axis}, que indica el eje que debe normalizarse. Por ejemplo, después de una capa \textit{Conv2D} donde establecemos \textit{data\_format = ``channels\_first''}, el valor de \textit{axis} será 1. Mientras que si establecemos \textit{data\_format = ``channels\_last''}, el valor de \textit{axis} será -1.
    
    \item \textit{Pooling layer}: Como vimos en la Sección \ref{tipos_capas}, esta capa reduce las dimensiones espaciales del volumen de entrada, reduce el coste computacional, y evita el sobreajuste.\\
    
    En Keras, dependiendo de las dimensiones de entrada y la operación empleada, existen diferentes capas de \textit{pooling}: MaxPooling1D, MaxPooling2D, MaxPooling3D, AveragePooling1D, AveragePooling2D, AveragePooling3D, GlobalMaxPooling1D, GlobalMaxPooling1D, etc. En Keras, los principales argumentos necesarios para definir estas capas son:
    
    \begin{itemize}
        \item \textit{pool\_size}: Factor por el cual se reduce la escala (vertical, horizontal), donde el factor es un número entero o una tupla de 2 enteros. Si solo se especifica un número entero, se utilizará la misma longitud de ventana para ambas dimensiones. Por ejemplo, si empleamos un \textit{pool\_size} de (2, 2) se reducirá a la mitad la entrada en ambas dimensiones espaciales.
        
        \item \textit{strides}: Indica cuántos píxeles debe desplazarse la ventana antes de aplicar la siguiente operación. Su valor es un entero, o una tupla de 2 enteros, o \textit{None}.

    \end{itemize}
    
    \item \textit{Dense layer}: En Keras, las capas \textit{fully-connected} se definen como \textit{Dense layers}. El argumento principal de este tipo de capa es:
    
        \begin{itemize}
            \item \textit{units}: número de neuronas.
        \end{itemize}
    
    \item \textit{Activation layer}: En Keras, una función de activación se puede declarar como una capa en sí misma o como un argumento dentro del método \textit{.add()} de la capa anterior. Keras proporciona varias funciones de activación, como \textit{sigmoid}, \textit{linear}, \textit{ReLU} y \textit{softmax}. En este proyecto se han empleado las funciones de activación:
    
        \begin{itemize}
            \item ReLU: es una función de activación no lineal, donde la salida es igual a 0 si la entrada es menor que 0, y si la entrada es mayor que 0 la salida es igual a la entrada. La función \textit{ReLU} sigue la siguiente fórmula:
            
            \begin{equation}\label{eq:relu}
		        g(x)=max(0,x)
            \end{equation}
            
            En la Figura \ref{fig.relu} se muestra la función de activación \textit{ReLU} en el intervalo [-10, 10].
            
            \begin{figure}[H]
              \begin{center}
                \includegraphics[width=0.5\textwidth]{figures/Estado_arte/relu.png}
            		\caption{Función de activación \textit{ReLU}}
            		\label{fig.relu}
            		\end{center}
            \end{figure}
                        
            \item Softmax: Esta función de activación es muy empleada en la capa de salida de los problemas de clasificación. La función \textit{softmax} escala las salidas de cada unidad para que estén entre 0 y 1, al igual que una función sigmoide, pero también divide cada salida de tal manera que la suma total de las salidas sea igual a 1.\\
            
            La función softmax se puede expresar matemáticamente en la ecuación (\ref{eq:SoftMax}), donde z es un vector de las entradas a la capa de salida, y j indexa las unidades de salida, entonces i = 1, 2, ..., K.
            
            \begin{equation}\label{eq:SoftMax}
		        \mathrm{softmax}(z)_i=\frac{\exp(z_i)}{\Sigma_{j}{\exp(z_j)}} \quad \mathrm{for} \ j=1, ...,K.
            \end{equation}
        \end{itemize}
    
    \item \textit{Flatten layer}: Aplana la entrada, es decir, modifica sus dimensiones. Por ejemplo, convierte los elementos de una matriz de imágenes de entrada en un array plano. Esta capa no afecta al \textit{batch\_size}.
    
    \item \textit{Dropout layer}: Esta capa consiste en establecer aleatoriamente una tasa de fracción (\textit{rate}) de unidades de entrada en 0 en cada actualización durante el tiempo de entrenamiento, lo que ayuda a evitar el sobreajuste. El argumento principal de esta capa es \textit{rate}, que es la tasa de fracción mencionada anteriormente. El valor de \textit{rate} debe estar entre 0 y 1.

    
    \item \textit{LSTM layer}: Implementa una capa \acrfull{lstm}. Esta capa tiene algunos argumentos esenciales:
    
        \begin{itemize}
            \item \textit{units}: Número de celdas LSTM.
            
            \item \textit{return\_sequences}: Booleano que indica si se debe devolver la última salida en la secuencia de salida o la secuencia completa. Si se establece a \textit{True} se devuelve la secuencia completa.
        \end{itemize}
        
     \item \textit{ConvLSTM2D layer}: Es similar a una capa \acrfull{lstm}, pero las transformaciones de entrada y las transformaciones recurrentes son convolucionales. Esta capa tiene algunos argumentos esenciales:
    
        \begin{itemize}
             \item \textit{filters}: Número de filtros. 
            
            \item \textit{kernel\_size}: Especifica la anchura y altura de los filtros. Puede ser un solo entero para especificar el mismo valor para todas las dimensiones espaciales, o puede ser una tupla o lista de 2 enteros.
            
            \item \textit{strides}: Entero o tupla/lista de 2 enteros, que especifica cuántos píxeles debe desplazarse el filtro antes de aplicar la siguiente convolución.
            
            \item \textit{padding}: Puede ser \textit{valid} o \textit{same}. Si se emplea \textit{valid}, no se aplica relleno, dando lugar a una salida con una dimensión más pequeña que la entrada. Sin embargo, si empleamos \textit{same}, la entrada se rellenará con ceros para dar lugar a una salida que conserve las dimensiones de la entrada. El valor por defecto es \textit{valid}.
            
            \item \textit{return\_sequences}: Booleano que indica si se debe devolver la última salida en la secuencia de salida o la secuencia completa. Si se establece a \textit{True} se devuelve la secuencia completa.
        \end{itemize}
\end{itemize}


\subsubsection{\textit{Callbacks} en Keras} \label{callback}

Un \textit{callback} es un conjunto de funciones que se aplicarán en determinadas etapas del proceso de entrenamiento. Se puede emplear los \textit{callbacks} para obtener un vistazo de los estados internos y las estadísticas del modelo durante el entrenamiento. En este proyecto se han empleado los siguientes \textit{callbacks}:

\begin{itemize}
    \item \textit{.ModelCheckpoint()}: Guarda el modelo y sus pesos después de cada época. Es posible configurar \textit{ModelCheckpoint} para que sobreescriba el modelo solamente si una métrica que indicamos ha mejorado respecto al mejor resultado anterior. De esta forma se guarda la mejor versión del modelo.
    
    \item \textit{.TensorBoard()} \cite{tensorboard}: Es un conjunto de herramientas de visualización porporcionado por \textit{TensorFlow}, que facilita la comprensión, la depuración y la optimización de los programas. Se puede emplear TensorBoard para visualizar el gráfico proporcionado por TensorFlow, trazar métricas cuantitativas sobre la ejecución del gráfico, así como histogramas de activación para las diferentes capas en el modelo, y mostrar datos adicionales como las imágenes que pasan a través de él. 
    
    \item \textit{.CSVLogger()}: Escribe un archivo de registro CSV que contiene información sobre las épocas, el \textit{accuracy} y \textit{loss} en el disco, dando la posibilidad de inspeccionarlo más tarde. De esta forma se pueden crear gráficos a partir de estos datos o mantener un registro del proceso de entrenamiento del modelo a lo largo del tiempo.
\end{itemize}

\subsection{Formato de archivo HDF5}

\acrfull{hdf5} \cite{hdf5_1} \cite{hdf5_2} es una librería de propósito general y al mismo tiempo un formato de ficheros para el almacenamiento de datos científicos. \acrshort{hdf5} fue creado con el fin de facilitar el trabajo a los ingenieros y científicos que trabajan en entornos con altas prestaciones y con un uso masivo de datos. Keras emplea el formado de archivo \acrshort{hdf5} para guardar modelos y leer conjuntos de datos. La tecnología \acrshort{hdf5} incluye:

\begin{itemize}
    \item Un modelo de datos versátil que puede representar objetos de datos complejos y una gran variedad de metadatos.
    
    \item Un formato de archivo completamente portable sin límite en el número o tamaño de los objetos de datos de una colección.
    
    \item Una biblioteca software que se ejecuta en diversas plataformas computacionales como ordenadores portátiles o sistemas masivamente paralelos. Además, implementa una \acrshort{api} de alto nivel con interfaces C, C++, Fortran 90 y Java.
    
    \item Un gran conjunto de funciones de rendimiento que permiten optimizar el tiempo de acceso y el espacio de almacenamiento.
    
    \item Herramientas y aplicaciones para manejar, manipular, visualizar y analizar datos.
    
    \item El modelo de datos \acrshort{hdf5}, el formato de archivo, la biblioteca y las herramientas son de código libre.
\end{itemize}

En este trabajo se han empleado los archivos \acrshort{hdf5} para guardar los modelos de las diferentes redes. Para tratar con archivos \acrshort{hdf5} se emplea la biblioteca h5py \footnote{www.h5py.org} para Python.


\lhead[]{CAPÍTULO \thechapter. INFRAESTRUCTURA DESARROLLADA}
\chapter{Infraestructura desarrollada}\label{cap.infraestructura}

El propósito de este proyecto es que un coche autónomo sea capaz de conducir en diferentes circuitos mediante distintas redes neuronales que son capaces de aprender control visual. El coche dispondrá de una cámara que le proporciona información de su entorno, permiténdole tomar decisiones.\\

El vehículo, como hemos mencionado, debe ser capaz de aprender determiandas acciones. Para que el coche pueda aprender es necesario disponer de una serie de datos, por lo que se ha creado un conjunto de entrenamiento que se verá en la Sección \ref{dataset}. Con el fin de grabar este conjunto de datos se ha creado un piloto basado en visión (Sección \ref{piloto}) que es capaz de dar vueltas alrededor del circuito de forma autónoma.\\

Además se explica la infraestructura software en la que nos hemos apoyado  para desarrollar el proyecto. Se definirán los circuitos de carreras empleados tanto para el entrenamiento de las redes como para el \textit{test}. 

\section{Circuitos de carreras en Gazebo}\label{modelos_circuitos}

El objetivo de este proyecto es que nuestro Fórmula 1 sea capaz de conducir de forma autónoma por un circuito, por lo que tendremos que crear diferentes entornos (circuitos) donde se moverá. Para facilitar el algoritmo del piloto autónomo explícito y del piloto autónomo basado en redes neuronales, los circuitos tienen una línea roja pintada en el suelo.\\

Los modelos de circuitos fueron creados con una herrramienta de modelado 3D (Blender, SketchUp, etc). La mayoría de los circcuitos se corresponden con circuitos de grandes dimensiones. En estos circuitos no veremos muchas gradas alrededor, ni público u otros elementos habituales de los circuitos reales, sino que se ha simplificado su creación para que sea rápido en la ejecución del simulador. Los mundos que tienen muchos detalles son más costosos computacionalmente de simular. Lo que podremos ver en estos circuitos son elementos propios de carreteras como son rectas, curvas simples o pronunciadas, una línea de salida, paredes que evitan que el coche se salga del recorrido, y en algunos casos una grada y césped de adorno.\\

El primer modelo de circuito empleado se llama \textit{pistaSimple}. Es un circuito de carreras de gran tamaño, que consta de una carretera con una línea roja en el suelo, una línea de salida, y las paredes para evitar que el coche se salga del circuito. Este circuito no posee ningún elemento de adorno, ya que haría que la simulación fuera muy lenta. Este modelo se puede observar en la Figura \ref{fig.simple}.\\

\begin{figure}
  \begin{center}
    \includegraphics[width=0.9\textwidth]{figures/Infraestructura/circuit_Simple.png}
		\caption{Modelo pistaSimple}
		\label{fig.simple}
		\end{center}
\end{figure}

El segundo modelo de circuito se denomina \textit{monacoLine} y simula el circuito de carreras de Mónaco (Principado de Mónaco), también conocido como Montecarlo. Este modelo consta del circuito en sí mismo, así como línea de salida, paredes que rodean el circuito, césped de adorno, y una grada pequeña. El modelo \textit{monacoLine} se puede ver en la Figura \ref{fig.monaco}.\\


\begin{figure}
  \begin{center}
    \includegraphics[width=0.9\textwidth]{figures/Infraestructura/circuit_Monaco.png}
		\caption{Modelo monacoLine}
		\label{fig.monaco}
		\end{center}
\end{figure}

El tercer modelo simula el circuito de carretas Nürburgring (Alemania) acortado, llamado \textit{nurburgrinLine}. Se ha modelado el circuito con una línea de salida, la carretera, paredes para evitar que el robot se salga del recorrido, una grada pequeña y césped de adorno. Este modelo se puede observar en la Figura \ref{fig.nurburgrin}.\\

\begin{figure}
  \begin{center}
    \includegraphics[width=0.9\textwidth]{figures/Infraestructura/circuit_Nurburgrin.png}
		\caption{Modelo nurburgrinLine}
		\label{fig.nurburgrin}
		\end{center}
\end{figure}

El cuarto modelo de circuito, llamado \textit{curveGP}, simula un circuito que no tiene ninguna curva, solamente tiene curvas, tanto curvas leves como abruptas. Este circuito es de gran tamaño, y por este motivo únicamente simula el circuito, y paredes que rodean al mismo, no tiene ningún elemento de adorno. El modelo \textit{curveGP} se puede ver en la Figura \ref{fig.curveGP}.\\


\begin{figure}
  \begin{center}
    \includegraphics[width=0.9\textwidth]{figures/Infraestructura/circuit_CurveGP.png}
		\caption{Modelo curveGP}
		\label{fig.curveGP}
		\end{center}
\end{figure}


El quinto modelo de circuito se denomina \textit{ pista\_simple}. Este circuito es el más corto de todos, aunque únicamente simula el circuito, y paredes que rodean al mismo, no tiene ningún elemento de adorno. En la Figura \ref{fig.small} se puede ver el modelo \textit{pista\_simple}.\\

\begin{figure}
  \begin{center}
    \includegraphics[width=0.9\textwidth]{figures/Infraestructura/circuit_Small.png}
		\caption{Modelo pista\_simple}
		\label{fig.small}
		\end{center}
\end{figure}

Los mundos que se simulan con Gazebo son mundos 3D. Estos mundos se cargan en ficheros con extensión .world, que no son más que ficheros \acrshort{xml} definidos en el lenguaje \acrshort{sdf}. Este lenguaje contiene una descripción completa de todos los elementos que tiene el mundo y los robots.\\

Se ha creado un mundo en Gazebo para cada circuito compuesto por uno de los cinco modelos de circuito y el modelo del coche (\textit{f1ROS}). Los archivos del mundo de cada circuito son iguales en todos los casos, las únicas diferencias son que el nombre del modelo de circuito cambia, y además la posición del modelo del coche también cambia, ya que en cada circuito la línea de salida está en un lugar diferente. Por ejemplo, el archivo f1-simple-circuit.world (mundo del circuito \textit{pistaSimple}) tiene el
siguiente aspecto:

\vspace{20pt}
\begin{lstlisting}[frame=single]
<?xml version="1.0" ?>
<sdf version="1.5">
  <world name="default">
    <scene>
      <grid>false</grid>
    </scene>
    <!-- A global light source -->
    <include>
      <uri>model://sun</uri>
    </include>
    <include>
      <uri>model://pistaSimple</uri>
	<pose>-160 17 0 0 0 0</pose>
    </include>
    <include>
      <uri>model://f1ROS</uri>
      <pose>1 0 0 0 0 -1.57</pose>
    </include>

 <scene>
        <sky>
            <clouds>
                <speed>12</speed>
            </clouds>
        </sky>
     </scene>
  <light name='user_directional_light_0' type='directional'>
      <pose frame=''>0 0 10 0 -0 0</pose>
      <diffuse>0.8 0.8 0.8 1</diffuse>
      <specular>0.2 0.2 0.2 1</specular>
      <direction>0.1 0.1 -0.9</direction>
      <attenuation>
        <range>1000</range>
        <constant>0.9</constant>
        <linear>0.01</linear>
        <quadratic>0.001</quadratic>
      </attenuation>
      <cast_shadows>1</cast_shadows>
  </light>

  </world>
</sdf>

\end{lstlisting}

Además de este fichero de configuración, es necesario crear un archivo con extensión .launch, que arranca los \textit{plugins} y \textit{drivers} de ROS-Kinetic. En este fichero es necesario indicar a Gazebo algunos argumentos como el nombre del fichero de configuración con el escenario (archivo del mundo), se establece el tiempo que empleará el escenario (como por ejemplo tiempo simulado), la posibilidad de lanzar una \acrshort{gui}, y algunas opciones de depuración. Es necesario crear un archivo .launch por cada modelo de circuito o escenario. Todos estos archivos serán iguales, excepto que en el argumento ``world\_name'' se modificará el valor del archivo .world empleado. Por ejemplo, para emplear el archivo de configuración \textit{f1-simple-circuit.world}, se ha creado el archivo \textit{f1.launch} que se puede ver a continuación:\\

\vspace{20pt}
\begin{lstlisting}[frame=single]
<?xml version="1.0" encoding="UTF-8"?>
<launch>
  <!-- We resume the logic in empty_world.launch, changing only the name 
  of the world to be launched -->
  <include file="$(find gazebo_ros)/launch/empty_world.launch">
    <arg name="world_name" value="f1-simple-circuit.world"/> <!-- Note: the 
    world_name is with respect to GAZEBO_RESOURCE_PATH environmental
    variable -->
    <arg name="paused" value="false"/>
    <arg name="use_sim_time" value="true"/>
    <arg name="gui" value="true"/>
    <arg name="headless" value="false"/>
    <arg name="debug" value="false"/>
    <arg name="verbose" default="false"/>
  </include>  
</launch>
\end{lstlisting}

Los modelos de circuitos \textit{pistaSimple}, \textit{monacoLine}, \textit{nurburgrinLine} son empleados para el entrenamiento y el \textit{test} de las redes neuronales de extremo a extremo. En los modelos con dependencias temporales (como \acrshort{lstm}) se añade el modelo \textit{curveGP} tanto para entrenamiento como para \textit{test}. Este modelo es empleado para adquirir más datos de curvas, ya que en los anteriores circuitos teníamos muchos más datos de rectas que de curvas, y es necesario que las redes posean más datos de este tipo para que sean capaces de aprender ciertos comportamientos. El modelo \textit{pista\_simple} se emplea únicamente para \textit{test}.\\



\section{Piloto autónomo explícito}\label{piloto}

Como hemos mencionado el objetivo del proyecto es que un vehículo sea capaz de aprender determiandas acciones que le permitan conducir de forma autónoma. Para lograr este objetivo es necesario disponer de una serie de datos, por lo que se ha creado un conjunto de datos (Sección \ref{dataset}). Con el fin de grabar este conjunto de datos se ha creado un piloto autónomo basado en visión que es capaz de dar vueltas alrededor del circuito de forma autónoma. \\

El piloto autónomo creado debe ser capaz de navegar rápidamente por un circuito de Fórmula 1 siguiendo una línea roja pintada en el suelo. Para ello, el coche dispone de una cámara en la parte frontal izquierda y unos motores a los que se envían órdenes de velocidad (velocidad de tracción y velocidad de rotación).\\

El coche tendrá una parte perceptiva y una parte de control. En la parte perceptiva el vehícuo deberá extraer la información relevante de las imágenes: dónde está la línea roja, si está en una recta o en una curva, si ha perdido la línea roja, etc. En la parte de control, el coche deberá hacer un control reactivo PID o bien un control PD, que sea capaz de corregir la velocidad de giro para mantener al coche por encima de la línea.\\ 

Antes de explicar la solución del piloto, es necesario saber qué es un control PID. Un control PID consta de tres parámetros distintos: el proporcional, el integral y el derivativo. Para explicar mejor cada una de las partes tomaremos como ejemplo una calefacción con termómetro, que es un ejemplo clásico de control PID.  En este ejemplo el sensor es el termómetro y el objetivo es obtener una determinada temperatura, por lo que tendremos un error que será la desviación entre la temperatura observada y la temperatura deseada. El error tendrá magnitud y signo, y el controlado PID deberá dar órdenes a la calefacción o el aire acondicionado para conseguir minimizar el error y obtener la temperatura deseada.\\

La parte proporcional (P) manda a los actuadores una corrección proporcional al error, de forma que si el error es pequeño se corrige suavemente y si es grande la corrección es mayor. De no existir desviación no se modifica la temperatura. El control P tiende a conseguir que el sistema obtenga la situación deseada, pero a veces lo logra con muchas oscilaciones. Una posibilidad para suavizar las oscilaciones es emplear la parte Derivativa (D), que realiza una corrección proporcional a la derivada del error. Si el error está creciendo el control D aumenta la corrección, y si por el contrario está disminuyendo, suaviza la corrección. En algunos casos no es suficiente con un control PD, ya que puede que se estabilice en una situación no deseada a pesar de las correcciones. Para poder eliminar estos \textit{offsets} se puede emplear la parte Integral (I). Este tipo de control actúa en función de la acumulación del error, es decir, si ha pasado bastante tiempo sin que el error tienda a cero, entonces aumenta la corrección por parte del control.\\

El control Proporcional-Integral-Derivativo (PID) permite anular un determinado error de manera reactiva. Dependiendo de la aplicación concreta podremos denominar error a una cosa u otra. En nuestro caso podemos considerar error (desviación) a la diferencia horizontal en píxeles entre el centro de la línea cuando el robot está realmente recto sobre ella (este es el valor objetivo) y el centro de la línea observado en el instante actual. En el caso de que el error sea cero, es que el coche estará completamente centrado sobre la línea.\\


Una vez que ya se ha explicado de forma general la parte perceptiva y la parte de control PID, ya podemos hablar del procesado de imagen empleado en el piloto, así como el control realizado y las herramientas necesarias.\\

El coche posee una cámara situada en la parte frontal izquierda como hemos visto anteriormente. Por lo tanto, lo primero que hace el piloto es acceder a las imágenes de dicha cámara. Como hemos mencionado antes, el piloto automático debe dar una vuelta al circuito. Para ello, será necesario detectar la línea roja que está en el centro de la carretera, es decir, será necesario realizar una umbralización de esta línea. Para realizar esta umbralización, primero debemos transformar las imágenes al espacio de color HSV, ya que HSV es más robusto que RGB ante cambios de iluminación. \\

Tras realizar la transformación a HSV, se ha realizado una umbralización de la imagen empleando la función \textit{cv2.inRange} de OpenCV. Este filtrado se hace en función al rango de valores de rojo de la línea de la carretera.\\

Si mostramos la imagen tras el filtrado vemos que aparecen puntos en negro que pertenecen a la línea roja y no debería ser así. Esto ocurre en la línea de salida del circuito, ya que se puede ver que la línea parpadea y por lo tanto no tiene siempre el tono de rojo habitual. Este problema lo podemos ver en la Figura \ref{fig.color}. Para solventar este problema, se ha aplicado un cierre morfológico (primero hace una dilatación y después una erosión). Esta operación hace que los pixeles que aparecían en negro en la línea filtrada ahora pertenezcan a la línea filtrada, es decir, que aparezcan en blanco. Se puede observar en la Figura \ref{fig.cierre} cómo después de aplicar el cierre morfológico no quedan huecos en la parte segmentada al filtrar la línea roja.\\

\begin{figure}
  \begin{center}
    \includegraphics[width=0.9\textwidth]{figures/Infraestructura/filtro_color.png}
		\caption{Filtrado de color}
		\label{fig.color}
		\end{center}
\end{figure}


\begin{figure}
  \begin{center}
    \includegraphics[width=0.9\textwidth]{figures/Infraestructura/filtro_cierre.png}
		\caption{Filtrado de color con cierre}
		\label{fig.cierre}
		\end{center}
\end{figure}

Tras obtener una imagen con la línea roja en blanco, se ha tenido en cuenta que si el coche circula sobre la carretera correctamente la línea debería aparecer más o menos centrada en la imagen. Si por el contrario, la línea aparece en la parte izquierda o derecha de la imagen, es debido a que el coche debería girar puesto que está un poco desviado de la línea roja. De esta forma se puede conocer el giro que debe realizar el coche.\\

Con el fin de analizar la información proporcionada por el filtrado de la imagen es necesario analizar alguna fila de la imagen filtrada. En el caso de que solamente analicemos una única fila de la imagen, no obtendremos suficiente información para saber si nos encontramos en recta o en curva. Por este motivo, se han analizado tres filas de la imagen con el fin de saber si el coche está situado encima de una recta o una curva, o si por otro caso se ha salido el coche de la línea. En nuestro caso analizamos las filas situadas en las posiciones y1 = 260, y2 = 310, e y3 = 350. En estas líneas se calcula el centro de la línea roja, para lo cual se comprueban los valores de la imagen filtrada. Con estos centros (centro de cada fila) podremos saber si nos encontramos en recta o curva.\\

El siguiente paso es procesar esta información. El centro de la línea roja situado en la fila y1 (260) será el único que no perderemos siempre que nos encontremos cerca de la línea roja, ya que es la posición que está situada más arriba. Los centros de la línea roja situados en las filas y2 (310) e y3 (350) es posible que se pierdan al llegar a una curva grande. Si no encontramos la línea roja en la posición y1 es debido a que nos hemos salido de la línea roja. En esta situación el coche deberá ir hacia atrás y girar hacia el circuito para continuar su recorrido. Si la última vez que se ha visto la línea estaba a la izquierda de la imagen, quiere decir que el coche se ha salido hacia la derecha, y por tanto tendrá que girar a la izquierda y viceversa. Este caso se ha tenido en cuenta al elaborar el piloto.\\

La desviación o error que tenemos en cada instante se calcula respecto a la posición ideal. Esta posición ideal o de referencia puede no ser el centro de la imagen, ya que la cámara no está centrada en el vehículo y mirando exactamente la recta. En este caso, hay que analizar las imágenes cuando el coche está en recta y ver cuál es el centro de referencia (posición ideal). Haciendo algunas pruebas se ha determinado que la posición ideal es x = 326. Las desviaciones se calcularán respecto a este valor objetivo. En nuestro caso calculamos la diferencia entre la posición x de y1 (260) y la posición ideal. Esta es la desviación que tendremos en cuenta a la hora de hacer los diferentes controles PD que se llevan a cabo. Además, como hemos mencionado anteriormente, se tiene en cuenta un caso por si el coche se ha salido del circuito.\\

Otro de los casos que se han tenido en cuenta en el piloto es si el centro situado en la fila situada más abajo (y3 = 350) se ha perdido. Esto puede ocurrir si estamos en una curva grande. Si se da este caso se realizará un control PD adaptado a esta excepción. En función de la desviación que hay entre el centro situado en la fila de arriba (y1) y el centro situado en la fila de en medio (y2), se darán diferentes situaciones en las que se ajustará un PD diferente.\\

En el piloto programado se ha decidido emplear un control PD, ya que puede ser que el uso de un control P no sea suficiente, puesto que el coche puede oscilar sobre la línea roja. Por lo tanto, es mejor emplear un control PD para evitar estos vaivenes del coche. El control PD (definido por un control derivativo y un control proporcional) sigue la siguiente fórmula:\\

\vspace{10pt}
\begin{lstlisting}
Correccion = kp error + kd (error - errorAnterior)
\end{lstlisting}
\vspace{20pt}

Los valores de las constantes kp y kd se han ajustado experimentalmente. En función de la desviación obtenida se tienen diferentes controladores PD para controlar la velocidad de rotación, y se mantienen diferentes velocidades constantes en función de cada caso para la velocidad de tracción.\\

Otro caso que evaluaremos es si nos situamos en recta o curva. Si estamos en recta se aplicarán unas situaciones de control PD y si estamos en curva otras. Para diferenciar si nos encontramos en recta o en curva se empleará la diferencia entre la posición x de y3 (350) y de y1 (260). Lo que se hace exactamente es calcular la recta que pasa por el centro de y1 e y3, y después se mira la posición de x que se encuentra en esta recta para la fila y2 (310). De este modo conociendo el centro de la fila y2 y el punto x que se encuentra en la recta calculada, podemos saber si estamos en curva o recta. Esto se explica mejor a continuación. Para saber si es curva o recta se siguen los siguientes pasos:\\

\begin{enumerate}
    \item Conociendo la ecuación de una recta (y = m (x - x1) + y1) podemos calcular la pendiente de la recta del siguiente modo:
    
        \[ y = m (x - x1) + y1 \]
        \[ 260 - 350 = m (x\_arriba - x\_abajo) \]
        \[ m = -90 / (x\_arriba - x\_abajo) \]
        
     Donde x\_arriba es el centro de la fila y1 (arriba) y x\_abajo es el centro de la fila y3 (abajo).
        
    \item Una vez conocida la ecuación de la recta que pasa por estos dos puntos, se calcula el punto x de la fila de en medio (y2) que pasaría por esta recta. Este valor de x lo calculamos del siguiente modo:
        
        \[ y2 - y3 = m (x - x\_abajo) \]
        \[ (310 - 350) (x\_arriba - x\_abajo) / (-90) = (x - x\_abajo)  \]
        \[ x = (310 - 350) (x\_arriba - x\_abajo) / (-90) + x\_abajo  \]
        
    \item  Una vez tenemos el punto x que pasa por la recta calculada a partir de y1 e y3, podemos calcular la desviación de x respecto al centro de y2. Si esta desviación es mayor a un umbral es que estamos en curva y si no es que estamos en recta. Como hemos mencinado antes, aplicamos unas situaciones de PD para recta y otras situaciones para curva.
    
\end{enumerate}

A la hora de probar el piloto autónomo explícito se han hecho pruebas para ver su robustez. Una de estas pruebas consiste en modificar la posición del Fórmula 1 en la línea de salida para que no esté situado el coche justamente encima de la línea roja. En este caso el coche es capaz de volver encima de la línea roja. Otra prueba ha sido mover el vehículo mediante el teleoperador durante la ejecución del piloto programado. El coche es capaz de volver otra vez encima de la línea roja a pesar de intentar girar el coche con el teleoperador.\\

El piloto se ha ejecutado en los diferentes circuitos, tanto en sentido horario como en sentido anti-horario. Se ha medido los tiempos de simulación que tarda el vehículo en dar una vuelta al circuito en ambos sentidos. En la tabla \ref{tabla_piloto} se muestran los resultados de este piloto.\\


\begin{table}[]
\centering
\caption{Resultados del Pilotaje autónomo explícito}
\label{tabla_piloto}
\begin{tabular}{| c| c |}
\hline
Circuitos & Tiempo simulado \\
\hline \hline
pistaSimple (horario) & 1min 35 seg \\ \hline
pistaSimple (anti-horario) & 1min 33 seg \\ \hline
monacoLine (horario) & 1min 15 seg \\ \hline
monacoLine (anti-horario) & 1min 15 seg \\ \hline
nurburgrinLine (horario) & 1min 02 seg \\ \hline
nurburgrinLine (anti-horario) & 1min 02 seg \\ \hline
curveGP (horario) & 2min 13 seg \\ \hline
curveGP (anti-horario) & 2min 09 seg \\ \hline
pista\_simple (horario) & 1min 00 seg \\ \hline
pista\_simple (anti-horario) & 59 seg \\ \hline
\end{tabular}
\end{table}


\section{Creación de conjunto de datos para entrenamiento neuronal}\label{dataset}

En este proyecto el objetivo es que un vehículo sea capaz de conducir bajo diferentes circunstancias, es decir, en diferentes entornos y diferentes iluminaciones. Además, el objetivo es emplear diferentes redes neuronales con este fin. Para que las redes neuronales sean capaces de aprender es necesario crear un conjunto de datos.\\

Como se menciona en el libro ``Deep Learning, Introducción práctica con Keras'' \cite{Jordi_torres}, el conjunto de datos se debe dividir para poder configurar y evaluar el modelo de forma correcta. En \textit{Deep Learning} estos datos se dividen en tres conjuntos: datos de entrenamiento (\textit{training}), datos de validación (\textit{validation}) y datos de prueba (\textit{test}).\\

Los datos de entrenamiento son aquellos que se utilizan para que la red obtenga los parámetros del modelo. Cuando entrenamos un modelo con un conjunto de entrada lo que ocurre es que hacemos que el modelo sea capaz de aprender de forma general un concepto. De esta forma cuando le consultamos por nuevos datos el modelo será capaz de comprender estos nuevos datos y devolver un resultado fiable en función de su capacidad de generalización. Sin embargo, si este modelo no es capaz de adaptarse a los datos de entrada (por ejemplo se produce \textit{underfitting} u \textit{overfitting}), en este caso modificaremos los hiperparámetros del modelo, y después de entrenar el modelo de nuevo con los datos de entrenamiento evaluaremos el modelo con los datos de validación.\\

Los hiperparámetros se pueden ir ajustando guiados por los datos de validación hasta obtener unos resultados de validación que consideremos apropiados. Si hemos seguido este método, lo que ha sucedido es que el modelo se ha ajustado también a los datos de validación. Por este motivo, es necesario reservar un conjunto de \textit{test} que solamente emplearemos al evaluar el modelo cuando consideremos que ya hemos terminado de ajustar los hiperparámetros.\\

Como hemos visto para realizar el entrenamiento y la evaluación de las redes neuronales es necesario disponer de bases de datos donde el contenido se corresponda con el estímulo del problema. Por este motivo es necesario crear una conjunto de datos para conducción autónoma en el simulador Gazebo. Esta base de datos se ha creado a partir del piloto autónomo explícito que se ha explicado en la Sección \ref{piloto}. Como se mencionó en esta Sección, este piloto es capaz de conducir de forma autónoma mediante una solución basada en visión que calculaba las órdenes de velocidad que hay que enviar al vehículo.\\

El conjunto de datos creado es un \textit{dataset} ``casero'' donde se han almacenado las imágenes de la cámara del piloto en cada instante, así como los datos de velocidad necesarios. Los datos de velocidades se han guardado en un archivo \textit{.json}. En este fichero se han guardado diferentes datos relacionados con la velocidad. Por un lado, se han almacenado en un diccionario los datos númericos de v y w con las claves \textit{v} y \textit{w}. Por otro lado se han almacenado datos que servirán para entrenar las redes de clasificación. Para almacenar estos datos se han creado diferentes claves con sus valores en función de diferentes clasificaciones. Se han creado diferentes clasificaciones para ver el efecto que tienen en las redes neuronales. A continuación, se puede observar cómo se almacenan los datos de velocidad correspondientes a la primera imagen del \textit{dataset}:\\

\begin{lstlisting}
{"class_v_5": very_fast, "class_w_9": slight, 
"class3": "very_fast", "class2": "slight", 
"classification": "left", "w": 0.029500000000000002, "v": 13}
\end{lstlisting}
\vspace{20pt}

En este ejemplo, se puede observar que hay diferentes claves con su valor como habíamos mencionado. Las claves correspondientes a las diferentes clasificaciones son:

\begin{itemize}
    \item \textit{classification}: Esta clasificación divide los datos entre las clases ``left'' y ``right'' en función de los datos de velocidad de giro. Si la velocidad de giro es negativa el dato se corresponde con la clase ``right'', y si por el contrario es positiva se corresponde con la clase ``left''.
    
    \item \textit{class2}: Esta clasificación divide los datos de la velocidad lineal en 4 clases. Estas clases son: ``slow'' si la velocidad es menor o igual que 7; ``moderate'' si la velocidad es mayor que 7 y menor o igual que 9; ``fast'' si la velocidad es mayor que 9 y menor o igual que 11; y ``very\_fast'' si la velocidad es mayor que 11.
    
    \item \textit{class3}: En esta clasificación se dividen los datos de velocidad angular en 7 clases. Las clases son: ``radically\_left'' si la velocidad de rotación (w) es mayor o igual a 1; ``moderately\_left'' si w es menor que 1 y mayor o igual que 0.5; ``slightly\_left'' si w es menor que 0.5 y mayor o igual que 0.1; ``slight'' si w es menor que 0.1 y mayor a -0.1; ``slightly\_right'' si w es menor o igual que -0.1 y mayor que -0.5; ``moderately\_right''  si w es menor o igual que -0.5 y mayor que -1; y ``radically\_right'' si la velocidad de rotación es menor que -1.
    
    \item \textit{class\_v\_5}: Se dividen los datos de la velocidad lineal (v) en 5 clases. Estas clases son: ``negative'' si v es menor que 0, ``slow'' si la velocidad es menor o igual que 7 y mayor que 0; ``moderate'' si la v es mayor que 7 y menor o igual que 9; ``fast'' si la v es mayor que 9 y menor o igual que 11; y ``very\_fast'' si la velocidad es mayor que 11.
    
    \item \textit{class\_w\_9}: Se dividen los datos de velocidad angular en 9 clases. Las clases son: ``radically\_left'' si la velocidad de rotación (w) es mayor o igual a 2; ``strongly\_left'' si w es menor que 2 y mayor o igual que 1; ``moderately\_left'' si w es menor que 1 y mayor o igual que 0.5; ``slightly\_left'' si w es menor que 0.5 y mayor o igual que 0.1; ``slight'' si w es menor que 0.1 y mayor a -0.1; ``slightly\_right'' si w es menor o igual que -0.1 y mayor que -0.5; ``moderately\_right''  si w es menor o igual que -0.5 y mayor que -1; ``strongly\_right'' si w es menor o igual que -1 y mayor a -2; y ``radically\_right'' si la velocidad de rotación es menor que -2.
\end{itemize}

Cuando se comenzó el proyecto se creó un conjunto de datos que constaba de 5006 pares de datos, de los cuales 2803 pares eran datos de entrenamiento, 701 eran datos de validación, y 1502 eran datos de \textit{test}. Este conjunto fue grabado gracias al piloto autónomo programado que conducía de forma autónoma únicamente por el circuito \textit{pistaSimple} en un mismo sentido. Este conjunto tenía un problema y es que únicamente el vehículo poseía datos de un circuito, por lo que no era capaz de generalizar y aprender a conducir en otros entornos.\\

Con el objetivo de analizar los datos de los que disponíamos y solventar este problema se ha creado una gráfica de estadísticas de los datos. En cada imagen se han analizado dos fila y se ha calculado el centroide de la línea roja del circuito para cada una de las filas (filas 350 y 260). En el eje \textit{x} de la gráfica, se representa el centroide de la fila 350 (L2), y el eje \textit{y} representa el centroide de la fila 260 (L1) de la imagen. En la Figura \ref{fig.L1_L2_dataset1} podemos ver la representación de esta estadística del conjunto de entrenamiento (círculos rojos) contra los datos de una conducción fallida (cruces azules).\\


\begin{figure}
  \begin{center}
    \includegraphics[width=1\textwidth]{figures/Infraestructura/L1_L2_dataset1_driving.png}
		\caption{Representación pares L1-L2 (\textit{Dataset1} contra conducción)}
		\label{fig.L1_L2_dataset1}
		\end{center}
\end{figure}

En la Imagen \ref{fig.L1_L2_dataset1} se puede observar que hay bastantes casos conocidos para el coche, como pueden ser probablemente situaciones donde se encuentra en recta el coche; pero sin embargo hay zonas de la gráfica donde se representan cruces azules (conducción fallida) y no hay ningún círculo rojo (conjunto de entrenamiento) alrededor. Esto quiere decir que el coche está ante situaciones desconocidas, y por tanto, no sabe que hacer. \\

La gráfica \ref{fig.L1_L2_dataset1} nos hace pensar que es necesario entrenar el modelo con un conjunto de imágenes bastante más representativo. Por este motivo se ha creado un nuevo conjunto de datos que trata de solventar este inconveniente.\\

El nuevo conjunto de datos (denominado \textit{Dataset}) se ha grabado en tres circuitos diferentes: \textit{pistaSimple}, \textit{monacoLine}, \textit{nurburgrinLine}. El piloto ha dado varias vueltas a los 3 circuitos en ambos sentidos para poder grabar este conjunto. Este \textit{dataset} consta de 17341 pares de imágenes-datos. Se han dividido los datos obteniendo 9710 pares de datos de entrenamiento, 2428 pares de validación, y 5203 pares para \textit{test}.\\

La misma gráfica estadística que se ha empleado para evaluar el dataset ``fallido'' se ha utilizado con el fin de evaluar los nuevos datos. En la Figura \ref{fig.L1_L2_dataset3} se puede ver cómo el nuevo conjunto de datos es muy representativo, ya que consigue abarcar la gran mayoría de los casos que eran desconocidos para el coche.\\

\begin{figure}
  \begin{center}
    \includegraphics[width=1\textwidth]{figures/Infraestructura/L1_L2_Dataset3.png}
		\caption{Representación pares L1-L2 (nuevo \textit{Dataset} contra conducción)}
		\label{fig.L1_L2_dataset3}
		\end{center}
\end{figure}

En los circuitos utilizados para grabar el nuevo conjunto de datos existen más datos de rectas que de curvas, por este motivo se ha grabado un \textit{dataset} complementario al anterior de un circuito que solamente posee curvas (\textit{curveGP}). Este conjunto consta de 5268 pares de imágenes datos.\\

Este conjunto únicamente se ha añadido al entrenamiento de las redes neuronales recurrentes, ya que en las redes neuronales convolucionales ha sido suficiente con entrenar con el \textit{dataset} anterior. En la Figura \ref{fig.L1_L2_dataset_curves} se puede ver la gráfica estadística de este cojunto de datos (\textit{Dataset\_Curves}).\\

\begin{figure}
  \begin{center}
    \includegraphics[width=1\textwidth]{figures/Infraestructura/L1_L2_Dataset_Curves.png}
		\caption{Representación pares L1-L2 (\textit{Dataset\_Curves} contra conducción)}
		\label{fig.L1_L2_dataset_curves}
		\end{center}
\end{figure}

La gráfica estadística también se puede emplear para evaluar cómo se distribuyen los datos de las clases, tanto para la velocidad lineal (v) como para la velocidad de rotación (w). En la Figura \ref{fig.L1_L2_dataset_w} se representa el análisis de los pares L1-L2 (centroides de las filas 260 y 350) para 7 clases de w (``radically\_left'', ``moderately\_left'', ``slightly\_left'', ``slight'', ``slightly\_right'', ``moderately\_right'', ``radically\_right''). En la Figura \ref{fig.L1_L2_dataset_v} se representa el análisis de los pares L1-L2 para 5 clases de v (``negative'', ``slow'', ``moderate'', ``fast'', ``very\_fast''). \\

En estas imágenes (Figuras \ref{fig.L1_L2_dataset_w} y \ref{fig.L1_L2_dataset_v}) se observa cómo los ejemplos quedan más o menos agrupados por clases entorno a un rango de valores de L1 y L2. \\

\begin{figure}
  \begin{center}
    \includegraphics[width=1\textwidth]{figures/Infraestructura/L1_L2_dataset3_w.png}
		\caption{Análisis de pares L1-L2 (\textit{Dataset}) para w}
		\label{fig.L1_L2_dataset_w}
		\end{center}
\end{figure}

\begin{figure}
  \begin{center}
    \includegraphics[width=1\textwidth]{figures/Infraestructura/L1_L2_dataset3_v.png}
		\caption{Análisis de pares L1-L2 (\textit{Dataset}) para v}
		\label{fig.L1_L2_dataset_v}
		\end{center}
\end{figure}


En el entrenamiento de las redes neuronales convolucionales se ha empleado el conjunto de datos \textit{Dataset}; mientras que en el entrenamiento de las redes neuronales recurrentes se ha utilizado dicho conjunto, y adicionalmente el conjunto \textit{Dataset\_Curves}.\\

En ejecución, el piloto entrenado con las redes se ha probado en todos estos circuitos mencionados (\textit{pistaSimple}, \textit{monacoLine}, \textit{nurburgrinLine}, \textit{curveGP}) en la creación de los \textit{datasets}, y además se ha probado en un circuito que las redes no han empleado para entrenamiento. Este circuito se denomina \textit{pista\_simple} y se ha utilizado en ambos sentidos.


\section{Piloto autónomo basado en redes neuronales}

Se ha creado una aplicación de control visual con la infraestructura necesaria que se comunica con el simulador Gazebo. Además, esta aplicación ha sido creada para cargar y emplear redes neuronales de conducción, resuelve varias funcionalidades: (a) ofrece una interfaz gráfica al usuario que le ayuda a depurar el código; (b) ofrece acceso a sensores y actuadores en forma de métodos simples (oculta el \textit{middleware} de comunicaciones); (c) incluye código auxiliar para poder emplear las predicciones realizadas por las redes (bien de clasificación o de regresión). Lo deja todo atado para que el usuario sólo tenga que incluir su red y retoque unfichero donde se proporciona al vehículo las órdenes de velocidad predichas por la red. En la Figura \ref{fig.nodo_piloto} se puede observar la estructura que tiene esta aplicación.\\

\begin{figure}
  \begin{center}
    \includegraphics[width=0.8\textwidth]{figures/Infraestructura/nodo_piloto.png}
		\caption{Estructura de la aplicación de control visual basada en redes neuronales}
		\label{fig.nodo_piloto}
		\end{center}
\end{figure}

Este nodo ofrece al programador un \acrfull{api} de sensores y actuadores, y de predicciones de la red. A cotinuación se puede observar el \acrshort{api} concreto de este proyecto:

\begin{itemize}
    \item \textit{camera.getImage}: Permite obtener la imagen de la cámara del coche.
    
    \item \textit{motors.sendV}: Para establecer la velocidad lineal.
    
    \item \textit{motors.sendW}: Para establecer la velocidad de giro.
    
    \item \textit{network.prediction\_v}: Para obtener la predicción de velocidad lineal de la red.
    
    \item \textit{network.prediction\_w}: Para obtener la predicción de velocidad de giro de la red.
\end{itemize}

Es necesario emplear la bliblioteca \textit{comm} para realizar la comunicaciones entre distintos componentes. Podemos usar esta librería empleando \acrshort{ros}, o a través de un proxy de \acrshort{ice} (protocolo de comunicaciones). Para poder crear un comunicador con \textit{comm} es necesario emplear el archivo de configuración YML \footnote{YAML Ain't Markup Language format}.\\

En el archivo YML se indican los puertos de los \textit{plugins} que emplea el coche de carreras. Además, tenemos que indicar el \textit{framework} de redes neuronales que emplearemos (Keras o Tensorflow), así como los pesos de modelos de redes neuronales que ya hemos entrenado y que cargaremos para poder predecir datos de velocidad. Este fichero (driver.yml) en el proyecto tiene el siguiente aspecto:\\


\vspace{10pt}
\begin{lstlisting}

Driver:

  CameraLeft:
    Server: 2 # 0 -> Deactivate, 1 -> Ice , 2 -> ROS
    Proxy: "cam_f1_left:tcp -h localhost -p 8995"
    Format: RGB8
    Topic: "/F1ROS/CameraL/image_raw"
    Name: FollowLineF1CameraLeft

  Motors:
    Server: 0 # 0 -> Deactivate, 1 -> Ice , 2 -> ROS
    Proxy: "Motors:tcp -h localhost -p 9999"
    Topic: "/F1ROS/Motors"
    Name: FollowLineF1Motors

  robot: F1

  Network:
    Framework: Keras  # Currently supported: "Keras" or "TensorFlow"
    Model_Classification_w: models/model_smaller_vgg_7classes_biased_cropped_w.h5
    Model_Classification_v: models/model_smaller_vgg_5classes_biased_cropped_v.h5
    Model_Regression_v: models/model_controlnet_v.h5
    Model_Regression_w: models/model_controlnet_w.h5
    Dataset: Net/Dataset
NodeName: Driver
\end{lstlisting}
\vspace{20pt}

Podemos ver que los motores emplean el Puerto 9999, mientras que la cámara emplea el Puerto 8995. Además, podemos observar que en este archivo se indica que se utilizará el \textit{framework} Keras, así como los modelos (para v y w) que cargaremos en el caso de emplear una red de clasificación o una red de regresión.\\

Se han creado dos clases que permiten cargar los modelos de v y w indicados en el archivo de configuración \textit{.yml}, así como predecir los valores de v y w, y almacenarlos en las variables \textit{self.network.prediction\_v} y \textit{self.network.prediction\_w}. La clase de Python creada para las redes neuronales de clasificación es ClassificationNetwork, y para redes de regresión es RegressionNetwork. En el archivo principal (driver.py) se indicará cuál de estas dos clases queremos emplear al ejecutar el nodo Piloto. De esta forma desde un fichero podremos emplear las velocidades predichas por la red e indicar las órdenes de velocidad al vehículo.\\

Se ha dividido la aplicación piloto en diferentes partes, por lo que emplearemos hilos de ejecución para llevar a cabo diferentes tareas de forma simultánea. En este proyecto existen tres procesos diferenciados:

\begin{itemize}
    \item Hilo de percepción y control: Es el encargado de actualizar los datos de los sensores y los actuadores. El tiempo de refresco de este hilo es muy importante, y debe ser un periodo de tiempo muy corto, ya que se encarga de establecer la velocidad y la dirección del vehículo en todo momento. Si este intervalo de tiempo fuera muy grande, las decisiones que modifican la trayectoria del coche podrían ser incorrectas. Este hilo (\textit{ThreadPublisher}) se utiliza para actualizar los datos de la cámara y enviar órdenes a los motores. Se actualiza cada 80 milisegundos.
    
    \item Hilo de la interfaz gráfica de usuario (\acrshort{gui}): Se encarga de actualizar la \acrshort{gui}. El intervalo de actualización de este hilo es muy importante, ya que tenemos que mostrar la imagen que ve el coche en tiempo real. Por lo que el intervalo de tiempo debe ser pequeño. El hilo de ejecución de la \acrshort{gui} (\textit{ThreadGUI}) se actualizará cada 50 ms.
    
    \item Hilo de la red: Es el encargado de inferir valores a partir la última imagen recibida, de forma asíncrona. Cuando termina la inferencia, se almacena el valor dentro del elemento de red. Cuando el vehículo necesita los últimos datos de inferencia, solamente toma estos datos sin bloquear ningún proceso ni llamda. Este hilo (\textit{ThreadNetwork}) se debe actualizar en intervalos de tiempo pequeños para que el coche sea capaz de conducir. Este hilo se actualiza cada 50 ms.

\end{itemize}

\subsection{Interfaz gráfica}

La intergaz gráfica (\acrshort{gui}) del proyecto es una ayuda para proporcionar datos durante el pilotaje del vehículo. Esta interfaz se ha creado empleando PyQt5, dado que permite realizar interfaces con numerosos objetos gráficos (imágenes, botones, etc).\\

La \acrshort{gui} del proyecto (Figura \ref{fig.gui}) contiene la imagen que captura la cámara del vehículo. Esta imagen está situada en la parte superior izquierda de la \acrshort{gui}. Gracias a ella, el usuario puede tener una idea de la visión del coche y emplear estas imágenes como datos de entrada a las redes entrenadas.\\

\begin{figure}
  \begin{center}
    \includegraphics[width=0.7\textwidth]{figures/Infraestructura/gui.png}
		\caption{Interfaz gráfica (\acrshort{gui})}
		\label{fig.gui}
		\end{center}
\end{figure}

En la parte derecha de la imagen mostrada en la \acrshort{gui} hay 4 leds que se corresponden con diferentes rangos de velocidades lineales; mientras que en la parte inferior hay 7 leds que se corresponden con velocidades angulares del coche. En función del valor predicho por la red se encenderá un led verde de la parte inferior y un led azul de la parte derecha. De esta forma, es más fácil obtener una visualización de las órdenes de velocidad aproximadas que está recibiendo el coche. \\

Si denotamos los leds (velocidad angular) que aparecen debajo de la imagen con los números de 1 a 7 empezando por la izquierda, se encenderá cada uno de los leds en el siguiente caso:

\begin{itemize}
    \item Led 1: La velocidad angular del vehículo es mayor o igual a 1, es decir, el coche gira bruscamente a la izquierda.
    
    \item Led 2: La velocidad angular del vehículo está en el rango [0.5, 1), es decir, el coche gira moderadamente a la izquierda.
    
    \item Led 3:  La velocidad angular del coche se encuentra en el rango [0.1, 0.5), es decir, el coche gira ligeramente a la izquierda.
    
    \item Led 4: La velocidad angular del coche está en el rango (-0.1, 0.1), el coche está en recta.
    
    \item Led 5: La velocidad angular del vehículo se encuentra en el rango (-0.5, -0.1],  el coche gira ligeramente a la derecha.
    
     \item Led 6: La velocidad angular del vehículo está en el rango (-1, -0.5], el coche gira moderadamente a la derecha.
     
     \item Led 7: La velocidad angular del coche es menor o igual que -1, el coche gira bruscamente a la derecha.
\end{itemize}

Si denotamos los leds (velocidad lineal) que a la derecha de la imagen con los números de 1 a 4 empezando por abajo, se encenderá cada uno de estos leds en el siguiente caso:

\begin{itemize}
    \item Led 1: La velocidad es menor o igual que 7. Esta situación se corresponde con los casos donde el coche se encuentra en una curva y necesita reducir la velocidad o incluso dar marcha atrás en algún caso.
    
    \item Led 2: La velocidad se encuentra en el rango (7, 9], es decir, o bien estamos en recta y nos encontramos algo desviados o estamos en una curva muy leve.
    
    \item Led 3: La velocidad se encuentra en el rango (9, 11], es decir, estamos en recta y nos encontramos un ``pelín'' desviados del centro de la línea roja.
    
    \item Led 4: La velocidad es mayor que 11. En esta situación nos encontramos en una recta y el coche va a mucha velocidad.
\end{itemize}


Además, para añadir más información de velocidad que le permita al usuario depurar fallos, a la derecha de los leds que indican la velocidad lineal se han añadido las órdenes de velocidades que se envían a los motores del coche. Por un lado, tenemos el valor de la velocidad lineal, que aparece indicado con una v; mientras que el valor de la velocidad angular se indica con una letra w.\\

Esta interfaz gráfica además muestra un teleoperador justo debajo del botón ``Play Code''. Mediante este teleoperador se puede mover manualmente el coche en el mundo de Gazebo si se desea. En la esquina superior derecha, donde se muestra las órdenes de velocidad del coche, aparecerán los valores de velocidad que tiene el mismo cuando se teledirige. La velocidad lineal del coche se puede controlar moviendo el \textit{joystick} en sentido vertical. Cuanto más subamos el \textit{joystick} más velocidad tendrá el coche hacia delante, y si lo bajamos del todo más velocidad lineal tendrá hacia atrás. La velocidad angular del vehículo se controla moviendo el \textit{joystick} en sentido horizontal, según lo movamos a izquierda o a la derecha, el robot girará en un sentido u otro.\\

En la aplicación gráfica también hay cuatro botones importantes, dos que sirven para controlar lo que  sucede con el algoritmo que controla al coche, y otros dos que se emplean en el manejo del un \textit{dataset}. El botón inferior  izquierdo, en el que aparece un símbolo de STOP, es el que emplearemos cuando teledirigimos el coche y queremos que pare en un punto y no siga navegando. El botón que se encuentra entre la imagen y el teleoperador, en el cual pone ``Play Code'', es el botón con el que le ordenamos al componente que comience a ejecutar el código del fichero donde le damos órdenes a los motores del coche. Este botón cambia de color al pulsarlo. Si queremos que este código pare en un determinado momento, pulsaremos el mismo botón haciendo que pare; y si queremos reanudar su comportamiento lo volveremos a pulsar.\\

Los otros dos botones se emplean para el manejo de un \textit{dataset}. Por un lado, tenemos el botón de la parte superior derecha, en el cual pone ``Save Dataset'', que sirve para crear un nuevo \textit{dataset} que guarde los datos tanto de velocidad como las imágenes que ve el coche durante la ejecución de su algoritmo. Este conjunto de datos se crea con la misma estructura que el conjunto de datos mencionado en la Sección \ref{dataset}, y se almacenará en el directorio ``Dataset''. Este conjunto de datos únicamente se crea si pulsamos este botón, de no ser así simplemente se ejecuta el algoritmo que permite navegar al vehículo. Por otra parte, el botón que se sitúa en la parte inferior derecha, permite borrar el \textit{dataset} creado.\\



\subsection{Tiempo de ejecución}

En el proyecto es muy importante el tiempo de ejecución, ya que este tiempo influye en las decisiones que toma el vehículo, y cuanto más rápido sea el algoritmo mejor. En el tiempo que tarda en tomar decisiones el coche influirá el ordenador que empleemos. Este es un inconveniente, ya que quien posea mejor ordenador obtendrá tiempos de ejecución menores que quien tenga un ordenador sin tantas capacidades. La ejecución de Gazebo consume muchos recursos del ordenador haciendo que el coche sea más lento.\\

En la parte inferior de Gazebo se puede ver el \textit{Real Time}, el \textit{Sim. Time} (tiempo simulado) y el \textit{Real Time Factor}, los cuales tienen mucho que ver en el tiempo de ejecución de Gazebo. El parámetro \textit{Real Time} expresa el tiempo real en ejecución. El factor \textit{Sim. Time} expresa el tiempo simulado. Si empleáramos un ordenador con grandes capacidades entonces el \textit{Sim. Time} debería estar próximo al \textit{Real Time}. Mientras que si utilizamos un ordenador con menos capacidades veremos que el \textit{Sim. Time} es mucho menor que el \textit{Real Time}. Por su parte, el factor \textit{Real Time Factor} es un producto de la tasa de actualización y el tamaño del paso. Si queremos obtener un tiempo de simulación bajo, este factor tendrá que tener un valor entorno a 1. Si este parámetro es menor que 1 veremos que la ejecución es más lenta, y cuando se aproxima a 0.2 o menos es demasiado lenta.\\

En el caso del ordenador que se ha empleado el \textit{Real Time Factor} es muy bajo en algunas ocasiones durante el pilotaje, lo que hace que el \textit{Real Time} sea mucho mayor que el \textit{Sim. Time}. En este ordenador el \textit{Real Time Factor} normalmente oscila entre 0.15 y 0.6, siendo en grandes ocasiones cercano a 0.15.

\lhead[]{CAPÍTULO \thechapter. REDES DE CLASIFICACIÓN}
\chapter{Redes de clasificación}\label{cap.clasificacion}

Una vez explicado el contexto, los objetivos y las herramientas empleadas en este proyecto, en este capítulo se detallarán las arquitecturas de redes de clasificación empleadas, así como la experimentación tanto con las dimensiones de la imagen como en el uso de diferentes conjuntos de datos y número de clases, y los resultados de los mismos.


\section{Contexto}

Las redes neuronales de extremo a extremo se han empleado ampliamente en problemas de clasificación, es decir, en problemas en los que el objetivo es determinar la clase a la que pertenece un elemento. En este proyecto se emplea este tipo de red con el fin de predecir las acciones de dirección y velocidad de tracción de un vehículo.\\

En las redes de clasificación se han cuantificado tanto las medidas de velocidad lineal como las medidas del ángulo de dirección en valores discretos, que representan las etiquetas de las clases. Se ha estudiado la influencia de las especificaciones de cuantización de clase en la conducción del vehículo. Las especificaciones son tanto el número de clases, como el rango de valores de estas clases.\\

Las diferentes clasificaciones empleadas en función de los valores de los ángulos y las velocidades lineales se pueden ver en la creación del dataset (Sección \ref{dataset}), donde se especifican los rangos de valores que toma cada clase en cada clasificación. En los experimentos realizados se verán las combinaciones empleadas de cierto número de clases para la velocidad de tracción (v) con cierto número de clases para la velocidad de rotación (w). En nuestro caso, se han probado combinaciones de 4 clases de v con 7 u 9 clases de w (con mayor o menor rango de w), y la combinación de 5 clases de v con 7 clases de w.\\

En la conducción, la red predecirá una determinada clase tanto para v como para w, pero esta clase se debe traducir a órdenes de velocidad que se envía al coche. Debido a que las clases que predecimos se encuentran en un determinado rango lo que se hace es mandar como orden de velocidad a los motores del coche el resultado de la media del mínimo y el máximo valor de ese rango, o un valor próximo a dicha media ajustado experimentalmente. Es decir, si por ejemplo, la red predice la clase ``moderately\_right'' de w, que para una clasificación de 7 clases tiene un rango de valores entre -0.5 y -1, entonces le indicaremos al vehículo que tome una velocidad de rotación igual a -0.75.\\

En las próximas secciones se detallarán las arquitecturas de red probadas, y además se detallará la influencia de emplear una imagen completa o una imagen recortada, así como la influencia que tiene el conjunto de entrenamiento en el rendimiento de las redes.


\section{Arquitecturas de red}

En esta sección se explicarán las arquitecturas de red que se han estudiado en redes de clasificación y las experiencias que se han obtenido de cada arquitectura. Las redes neuronales empleadas para clasificación son \acrshort{cnn}.

\subsection{LeNet-5}

En el momento inicial se empleó la arquitentura de red LeNet-5 \cite{LeCunGradient} (Figura \ref{fig.Lenet}), propuesta por Yann LeCun, Leon Bottou, Yosuha Bengio y Patrick Haffner para el reconocimiento de carácteres.\\

LeNet-5 es una red muy simple, que consta únicamente de 7 capas. Tres de estas capas son capas convolucionales (C1, C3 y C5), las cuales emplean un filtro de tamaño 5x5 y un \textit{stride} de 1. Entre las capas convolucionales se aplica una capas de submuestreo (\textit{pooling}), es decir, en total dos capas de \textit{pooling} (S2 y S4) con un tamaño de filtro de 2x2. La capa 6 es una \textit{fully-connected} (F6), seguida por la capa de salida.\\

\begin{figure}
\begin{center}
	\includegraphics[width=1\textwidth]{figures/Clasificacion/model_lenet.png}
   \caption{Arquitectura Lenet-5.}
	\label{fig.Lenet}
\end{center}
\end{figure}

Al comienzo del proyecto se empleó LeNet-5 para entrenar las redes de clasificación, pero pronto se vió que era un modelo muy simple y no era suficiente para que el coche fuera capaz de aprender a conducir de forma autónoma. Por este motivo se introdujo la siguiente arquitectura de red.

\subsection{SmallerVGGNet}

La arquitectura de red empleada \cite{smallerVggNet} (Figura \ref{fig.SmallerVGGNet}) es una versión reducida del modelo VGGNet, que fue propuesto por Simonyan y Zisserman en el artículo ``Very Deep Convolutional Networks for Large Scale Image Recognition''. La arquitectura SmallerVGGNet empleada está diseñada para problemas multiclase, es decir, una clasificación con diversas clases. \\

\begin{figure}
\begin{center}
	\includegraphics[width=0.28\textwidth]{figures/Clasificacion/model_smallervgg.png}
   \caption{Arquitectura SmallerVGGNet.}
	\label{fig.SmallerVGGNet}
\end{center}
\end{figure}

En esta red inicialmente tenemos un bloque compuesto por una capa convolucional de 32 filtros de tamaño 3x3 y activación \textit{ReLU}, seguida de una capa de normalización del lote (\textit{BatchNormalization}), una capa de sumuestreo (\textit{pooling}), y una capa de \textit{dropout} del 25\%. A continuación, hay dos bloques compuestos por una capa convolucional seguida de una capa \textit{BatchNormalization}, una capa una capa convolucional, una capa \textit{BatchNormalization}, una capa de \textit{pooling}, y una capa de \textit{dropout} del 25\%. En el primero de estos dos bloques en las capas convolucionales se emplean 64 filtros de tamaño 3x3 y activación \textit{ReLU}; mientras que en el segundo bloque en las capas convolucionales se usan 128 filtros de tamaño 3x3 y activación \textit{ReLU}. Al final de la red tenemos un bloque de capas \textit{Fully connected}, donde en la última de estas capas se utiliza una función de activación sigmoide para la clasificación de múltiples etiquetas.\\


Esta arquitectura es bastante más compleja que la arquitectura LeNet-5, permitiendo de esta forma a la red aprender situaciones más complejas. Este modelo se ha empleado en los diferentes experimentos realizados que veremos en la siguiente sección.


\section{Experimentos}

En esta sección se datallarán todos los experimentos realizados durante el entrenamiento de redes de clasificación, ya sean experimentos acerca de las dimensiones de las imágenes, el número de clases, etc. \\


\subsection{Dimensiones de imagen}

Las imágenes capturadas por la cámara del vehículo tienen unas dimensiones de 640 x 480 píxeles. Algunas pruebas consisten en emplear las imágenes completas (Figura \ref{fig.completa}) y entrenar las redes de clasificación con las mismas en formato BGR. Antes de entrenar las redes con estas imágenes, se reducen las dimensiones de las mismas por un factor de escala de 1/4 en horizontal y 1/4 en vertical en total para aliviar la carga del entrenamiento. Por lo que las imágenes a la entrada de la red tienen unas dimensiones de (160, 120, 3).\\

\begin{figure}
\begin{center}
	\includegraphics[width=0.5\textwidth]{figures/Clasificacion/img_normal.png}
   \caption{Imagen completa}
	\label{fig.completa}
\end{center}
\end{figure}

\begin{figure}
\begin{center}
	\includegraphics[width=0.5\textwidth]{figures/Clasificacion/img_cropped_class.png}
   \caption{Imagen recortada}
	\label{fig.recortada}
\end{center}
\end{figure}

Además se han realizado pruebas empleando un recorte de imagen (\textit{image cropping}), que consiste en extraer una zona concreta de la imagen donde se considera que se almacena la parte relevante de información. Es decir, esta imagen (Figura \ref{fig.recortada}) contiene información acerca de la carretera, eliminando de esta forma la parte del cielo de la imagen. Esta imagen tiene unas dimensiones de 240 x 640 píxeles, aunque antes de entrenar la red se reducen las dimensiones 1/4 en horizontal y 1/4 en vertical, siendo las dimensiones de la imagen de (160, 60, 3) al entrar a la red.



\subsection{Número de clases}

Como hemos mencionado antes, el número de clases, así como el rango de valores de velocidad de cada clase influirá en el rendimiento de la conducción. Es necesario entrenar una red para la velocidad lineal (v) y una red para la velocidad de rotación, siendo ambas empleadas durante la conducción.\\

En el proyecto se han estudiado varias combinaciones de clases de v y clases de w, sabiendo que los datos de velocidad de rotación se encuentran en un rango de (-2.9269; 3.1138) y los datos de velocidad de tracción se encuentran en el rango (-0.6; 13). En un primer momento se ha estudiado la combinación de 7 clases para w y 4 para v. Como segundo experimento se han empleado 9 clases para w y 4 clases para v. Finalmente, se ha experimentado con 7 clases de velocidad de rotación y 5 clases de velocidad de tracción.\\

En la clasificación de 7 clases de velocidad de rotación (w), las clases se dividen según los siguientes rangos:

\vspace{10pt}
\begin{lstlisting}
"radically_left": w >= 1
"moderately_left": 0.5 <= w < 1
"slightly_left": 0.1 <= w <= 0.5
"slight": - 0.1 < w < 0.1
"slightly_right": - 0.5 < w <= -0.1
"moderately_right": - 1 < w <= -0.5
"radically_right": w <= -1
\end{lstlisting}
\vspace{20pt}


En la clasificación de 9 clases de velocidad de rotación, las clases se dividen según los siguientes rangos:

\vspace{10pt}
\begin{lstlisting}
"radically_left": w >= 2
"strongly_left": 1 <= w < 2
"moderately_left": 0.5 <= w < 1
"slightly_left": 0.1 <= w <= 0.5
"slight": - 0.1 < w < 0.1
"slightly_right": - 0.5 < w <= -0.1
"moderately_right": - 1 < w <= -0.5
"strongly_right": -2 < w <= -1
"radically_right": w <= -2
\end{lstlisting}
\vspace{20pt}


En la clasificación de 4 clases de velocidad de tracción (v), las clases se dividen en función a los siguientes rangos:

\vspace{10pt}
\begin{lstlisting}
"slow": v <= 7
"moderate": 7 < v <= 9
"fast": 9 < v <= 11
"very_fast": v > 11
\end{lstlisting}
\vspace{20pt}

En la clasificación de 5 clases de velocidad de tracción, las clases se dividen en función a los siguientes rangos:

\vspace{10pt}
\begin{lstlisting}
"negative": v <= 0
"slow": 0 < v <= 7
"moderate": 7 < v <= 9
"fast": 9 < v <= 11
"very_fast": v > 11
\end{lstlisting}
\vspace{20pt}


Los resultados de estas combinaciones de las clasificaciones se describirán con más detalle en las Secciones \ref{metrica_clasificacion} y \ref{resultados_clasificacion}.


\subsection{Diferentes bases datos}

Los resultados de las redes neuronales no solamente tienen que ver con la arquitectura de red empleada o con la forma de entrenar, sino que el conjunto de entrenamiento tiene una gran influencia sobre el mismo.\\

En el entrenamiento se ha empleado la base de datos \textit{Dataset} (Sección \ref{dataset}), donde se tienen datos de todas las clasificacions mencionadas en la sección anterior. Aunque uno de los problemas de los datos en las redes de clasificación es que normalmente no se dispone del mismo número de datos por cada clase. Este problema implica que en algunas ocasiones haya muchos datos de una determinada clase y muy pocos de otra, haciendo que la red aprenda las situaciones donde hay muchos datos y no aprenda las clases donde tenemos menos datos.\\

En el conjunto de datos empleado sucede este inconveniente, ya que hay algunas clases de las que disponemos de muchos datos y de otras de muy pocos, desequilibrando de esta forma el aprendizaje de la red.\\

En este conjunto de datos disponemos de un total de 17341 pares de imágenes-datos, de los cuales se emplean para entrenamiento únicamente 12138 pares. Si empleamos el conjunto de datos entero para ver de cuántos datos disponemos en función de las clases nos encontramos con lo siguiente:\\

\begin{itemize}
    \item En la clasificación de 7 clases de velocidad de rotación (w), el número de datos por cada clase es:

    \vspace{10pt}
    \begin{lstlisting}
    Numero de datos de "radically_left": 825
    Numero de datos de "moderately_left": 3054
    Numero de datos de "slightly_left": 2882
    Numero de datos de "slight": 4030
    Numero de datos de "slightly_right": 2606
    Numero de datos de "moderately_right": 2907
    Numero de datos de "radically_right": 1037
    \end{lstlisting}
    \vspace{20pt}


    \item En la clasificación de 9 clases de velocidad de rotación,  el número de datos por cada clase es:

    \vspace{10pt}
    \begin{lstlisting}
    Numero de datos de "radically_left": 80 
    Numero de datos de "strongly_left": 745
    Numero de datos de "moderately_left": 2054
    Numero de datos de "slightly\left": 2882
    Numero de datos de "slight": 4030
    Numero de datos de "slightly_right": 2606
    Numero de datos de "moderately_right": 2907
    Numero de datos de "strongly_right": 953
    Numero de datos de "radically_right": 84
    \end{lstlisting}
    \vspace{20pt}


    \item En la clasificación de 4 clases de velocidad de tracción (v),  el número de datos por cada clase es:

    \vspace{10pt}
    \begin{lstlisting}
    Numero de datos de "slow": 9885
    Numero de datos de "moderate": 3251
    Numero de datos de "fast": 2535
    Numero de datos de "very_fast": 1670
    \end{lstlisting}
    \vspace{20pt}

    \item En la clasificación de 5 clases de velocidad de tracción,  el número de datos por cada clase es:

    \vspace{10pt}
    \begin{lstlisting}
    Numero de datos de "negative": 197
    Numero de datos de "slow": 9688
    Numero de datos de "moderate": 3251
    Numero de datos de "fast": 2535
    Numero de datos de "very_fast": 1670
    \end{lstlisting}
    \vspace{20pt}

\end{itemize}


Si evaluamos el conjunto de entrenamiento (12138 pares de datos) con el fin de saber de cuántos datos disponemos en función de las clases, obtenemos lo siguiente lo siguiente:\\

\begin{itemize}
    \item En la clasificación de 7 clases de velocidad de rotación (w), el número de datos por cada clase es:

    \vspace{10pt}
    \begin{lstlisting}
    Numero de datos de "radically_left": 590
    Numero de datos de "moderately_left": 2160
    Numero de datos de "slightly_left": 1996
    Numero de datos de "slight": 2800
    Numero de datos de "slightly_right": 1826
    Numero de datos de "moderately_right": 2013
    Numero de datos de "radically_right": 753
    \end{lstlisting}
    \vspace{20pt}


    \item En la clasificación de 9 clases de velocidad de rotación,  el número de datos por cada clase es:

    \vspace{10pt}
    \begin{lstlisting}
    Numero de datos de "radically_left": 572
    Numero de datos de "strongly_left": 18
    Numero de datos de "moderately_left": 2160
    Numero de datos de "slightly_left": 1996
    Numero de datos de "slight": 2800
    Numero de datos de "slightly_right": 1826
    Numero de datos de "moderately_right": 2013
    Numero de datos de "strongly_right": 20
    Numero de datos de "radically_right": 733
    \end{lstlisting}
    \vspace{20pt}


    \item En la clasificación de 4 clases de velocidad de tracción (v),  el número de datos por cada clase es:

    \vspace{10pt}
    \begin{lstlisting}
    Numero de datos de "slow": 6945
    Numero de datos de "moderate": 2306
    Numero de datos de "fast": 1725
    Numero de datos de "very_fast": 1162
    \end{lstlisting}
    \vspace{20pt}

    \item En la clasificación de 5 clases de velocidad de tracción,  el número de datos por cada clase es:

    \vspace{10pt}
    \begin{lstlisting}
    Numero de datos de "negative": 139
    Numero de datos de "slow": 6806
    Numero de datos de "moderate": 2306
    Numero de datos de "fast": 1725
    Numero de datos de "very_fast": 1162 
    \end{lstlisting}
    \vspace{20pt}

\end{itemize}


Como se puede ver tanto en el conjunto de datos completo como en el conjunto de entrenamiento existe un desbalanceo de los datos, lo que influirá en el entrenamiento. Por este motivo, se han realizado 3 experimentos basándonos en la base de datos:\\

\begin{itemize}
    \item El primer experimento consiste en entrenar la red con el conjunto de entrenamiento sin ninguna modificación. A las redes neuronales entrenadas de esta forma se les llamará desbalanceadas.
    
    \item El segundo experimento consiste en crear una nueva base de datos de entrenamiento balancedada, es decir, donde exista el mismo número de ejemplos por cada clase. Para lograr este objetivo se parte de la clase con el menor número de ejemplos, y se selecciona el mismo número de datos aleatoriamente para cada una de las otras clases. Por ejemplo, en la base de datos balanceada de 4 clases de velocidad lineal tendremos 1162 ejemplos por cada clase. A las redes entrenadas con estos conjuntos las llamaremos balanceadas.
    
    \item El tercer experimento consiste en entrenar las redes con el conjunto de entrenamiento por completo, pero al entrenar se emplea el parámetro \textit{class\_weight} de Keras que nos permite dar pesos a cada clase. Este parámetro es un diccionario donde se indican las etiquetas (clases) y los pesos que le damos a cada etiqueta. De esta forma aunque dispongamos de menos datos para alguna de las clases le daremos más peso durante el entrenamiento. A las redes entrenadas de esta manera les llamaremos sesgadas.
\end{itemize}

Los resultados de estos tres experimentos realizados para cada una de las combinaciones de velocidades se describirán en las Secciones \ref{metrica_clasificacion} y \ref{resultados_clasificacion}.



\subsection{Métricas de evaluación}\label{metrica_clasificacion}

Con el fin de evaluar los resultados obtenidos tras el entrenamiento se calculan ciertas métricas de evaluación en el conjunto de \textit{test}. En las redes de clasificación las métricas que se han evaluado son: \textit{Accuracy}, \textit{Accuracy Top 2}, \textit{Precision}, \textit{Recall}, y \textit{F1-Score}.\\

En los problemas de clasificación, \textit{Accuracy} es el número de predicciones correctas realizadas por el modelo sobre todo tipo de predicciones realizadas. El \textit{Accuracy} se puede calcular con la siguiente fórmula:

$$\frac{1}{N} \sum\limits_{n=1}^N \delta\{ \hat{p}_n = p_n \}$$
\vspace{10pt}

Donde $\hat{p_n}$ es la etiqueta que la red predice en la clasificación, $p_n$ son las etiquetas reales, y $N$ es el número de muestras. Por último, la función $\delta\{x\}$ se define como:

$$\delta\{ \textup{condición}\} = \left\{ \begin{array}{lr} 1 &  \textup{si condición} \\ 0 & \mbox{resto} \end{array} \right.$$
\vspace{10pt}

El término \textit{Accuracy Top 2} lo calculamos como la métrica \textit{Accuracy}, pero en este caso si la clase predicha es una de las clases adyacentes o la clase real se considera que es correcta la predicción.\\

La métrica \textit{Precision} se define como la relación entre los positivos verdaderos (TP) y el número total de positivos predichos por un modelo (TP y FP). En la clasificación multiclase se debe tener en cuenta que existen varias clases y se emplea la siguiente fórmula:

 
\[ Precision = \frac{TP_X}{TP_X + FP_X} \]
\vspace{10pt}

Donde \(TP_X\) es el número de verdaderos positivos para la clase X, es decir, el número de aciertos correspondiente para dicha clase. Mientras que \(FP_X\) es el número de falsos positivos para la clase X, es decir, el número de veces que se ha predicho dicha clase sin ser así.\\


El parámetro \textit{Recall} es la relación entre los positivos verdaderos (TP) y el número total de positivos que se producen. En los problemas multiclase se puede definir como:\\

\[ Recall = \frac{TP_X}{TP_X + FN_X} \]
\vspace{10pt}

Donde \(FN_X\) son los falsos negativos para la clase X, es decir, el número de veces que se predijo
errónamente otra clase habiéndose producido X.\\

La métrica \textit{F1-Score} es una puntuación única que representa tanto a Precision (P) como a Recall (R). Se puede calcular mediante la siguiente fórmula:

\[F1-Score =  \frac{2 * Precision * Recall}{Precision + Recall} \]
\vspace{10pt}


Estas métricas calculadas en el conjunto de \textit{test} nos dan una idea de cómo de bien ha ido el entrenamiento. Cada una de las medidas se calculará para cada una de las redes entrenadas para v y w.\\

En la Tabla \ref{metricas_classificacion_recortada_w} se pueden ver los resultados de las métricas promedio para las redes de velocidad de rotación (w) con imágenes recortadas. En este caso vemos 3 redes de 7 clases de w en función de cómo entrenamos según los datos, y 3 redes con 9 clases para w.\\


\begin{table}[H]
\centering
\caption{Métricas de test de redes de clasificación (w, imagen recortada)}
\label{metricas_classificacion_recortada_w}
\begin{tabular}{c|c|c|c|c|c|}
\cline{1-6}
                        \multicolumn{1}{|c|}{Red}    & Acurracy       & Accuracy top 2      & Precision       & Recall        & F1-Score        \\ \hline
\multicolumn{1}{|c|}{7w sesgada}    & 94 \%             & 99 \%         & 95 \%            & 95 \%          & 95 \%       \\ \hline
\multicolumn{1}{|c|}{7w balanceada}     & 93 \%             & 99 \%          &  94 \%              &  94 \%            &  94 \%             \\ \hline
\multicolumn{1}{|c|}{7w desbalanceada}      &  95 \%             & 99 \%           &  95 \%            & 95 \%        &  95 \%            \\ \hline
\multicolumn{1}{|c|}{9w sesgada}       &  93 \%     &  99 \%      &  94 \%           &  94 \%            &  94 \%               \\ \hline
\multicolumn{1}{|c|}{9w balanceada}      &  93 \%         &  99 \%        &  94 \%         &  94 \%      &  94 \%          \\ \hline
\multicolumn{1}{|c|}{9w desbalanceada}     & 95 \%          & 99 \%        & 96 \%           & 96 \%         & 96 \%                \\ \hline
\end{tabular}
\end{table}

En la Tabla \ref{metricas_classificacion_recortada_v} se pueden observar los resultados de las métricas para las redes de velocidad de tracción (v) con imágenes recortadas. En esta tabla tenemos 3 redes para 4 clases de v y 3 redes para 5 clases de v.\\

\begin{table}[H]
\centering
\caption{Métricas de test de redes de clasificación (v, imagen recortada)}
\label{metricas_classificacion_recortada_v}
\begin{tabular}{c|c|c|c|c|c|}
\cline{1-6}
                        \multicolumn{1}{|c|}{Red}    & Acurracy       & Accuracy top 2      & Precision       & Recall        & F1-Score        \\ \hline
\multicolumn{1}{|c|}{4v sesgada}    & 95 \%     & 98 \%         & 95 \%            & 95 \%          & 95 \%       \\ \hline
\multicolumn{1}{|c|}{4v balanceada}     & 92 \%        & 96 \%          &  94 \%              &  93 \%            &  93 \%             \\ \hline
\multicolumn{1}{|c|}{4v desbalanceada}      &  95 \%        & 97 \%           &  95 \%            & 95 \%        &  95 \%            \\ \hline
\multicolumn{1}{|c|}{5v sesgada}       & 93 \%         & 96 \%     & 95 \%            & 93 \%           & 94 \%              \\ \hline
\multicolumn{1}{|c|}{5v balanceada}      & 92 \%        & 95 \%         & 93 \%           & 92 \%     & 93 \%            \\ \hline
\multicolumn{1}{|c|}{5v desbalanceada}       & 93 \%          & 96 \%           & 95 \%          & 94 \%        & 94 \%               \\ \hline
\end{tabular}
\end{table}


En la Tabla \ref{metricas_classificacion_completa_w} se pueden observar los resultados de las métricas para las redes de velocidad de rotación con imágenes completas. En esta tabla tenemos 3 (sesgada, balanceada y desbalanceada) redes para 7 clases de w y 3 redes para 9 clases de w.\\

\begin{table}[H]
\centering
\caption{Métricas de test de redes de clasificación (w, imagen completa)}
\label{metricas_classificacion_completa_w}
\begin{tabular}{c|c|c|c|c|c|}
\cline{1-6}
                        \multicolumn{1}{|c|}{Red}    & Acurracy       & Accuracy top 2      & Precision       & Recall        & F1-Score        \\ \hline
\multicolumn{1}{|c|}{7w sesgada}    & 95 \%             & 99 \%         & 95 \%            & 95 \%          & 95 \%       \\ \hline
\multicolumn{1}{|c|}{7w balanceada}     & 93 \%             & 99 \%          &  94 \%              &  93 \%            &  93 \%             \\ \hline
\multicolumn{1}{|c|}{7w desbalanceada}      &  95 \%             & 99 \%           &  95 \%            & 95 \%        &  95 \%            \\ \hline
\multicolumn{1}{|c|}{9w sesgada}       & 94 \%     & 99 \%       & 95 \%      & 94 \%           & 94 \%               \\ \hline
\multicolumn{1}{|c|}{9w balanceada}      & 93 \%          & 99 \%         & 94 \%       & 93 \%       & 93 \%             \\ \hline
\multicolumn{1}{|c|}{9w desbalanceada}       & 95 \%        & 99 \%    & 95 \%        & 95 \%        & 95 \%                 \\ \hline
\end{tabular}
\end{table}

En la Tabla \ref{metricas_classificacion_completa_v} se pueden ver los resultados de las métricas de las redes de velocidad lineal con imágenes completas. En la tabla vemos 3 redes para 4 clases de v y 3 redes para 5 clases de v.\\

\begin{table}[H]
\centering
\caption{Métricas de test de redes de clasificación (v, imagen completa)}
\label{metricas_classificacion_completa_v}
\begin{tabular}{c|c|c|c|c|c|}
\cline{1-6}
                        \multicolumn{1}{|c|}{Red}    & Acurracy       & Accuracy top 2      & Precision       & Recall        & F1-Score        \\ \hline
\multicolumn{1}{|c|}{4v sesgada}    & 94 \%    & 97 \%         & 95 \%            & 95 \%          & 95 \%       \\ \hline
\multicolumn{1}{|c|}{4v balanceada}     & 89 \%       & 95 \%          &  91 \%              &  89 \%            &  90 \%             \\ \hline
\multicolumn{1}{|c|}{4v desbalanceada}      &  94 \%      & 97 \%           &  95 \%            & 95 \%        &  95 \%            \\ \hline
\multicolumn{1}{|c|}{5v sesgada}       &   93 \%      &  96 \%      &  95 \%             &   94 \%           &  94 \%               \\ \hline
\multicolumn{1}{|c|}{5v balanceada}      &  90 \%       &  95 \%         &  93 \%         &  91 \%       &  91 \%            \\ \hline
\multicolumn{1}{|c|}{5v desbalanceada}  &  93 \%      &  96 \%           &  95 \%         &  94 \%         &  94 \%    \\ \hline
\end{tabular}
\end{table}

En estas tablas se pueden ver que los resultados en el conjunto de prueba en la mayoría de casos superan el 90\%, pero esto no implica que la conducción vaya a tener éxito como veremos en la Sección \ref{resultados_clasificacion}, ya que el resultado de las métricas es un promedio de todas las clases. Esto quiere decir que por ejemplo en una clase nos puede dar un resultado de \textit{Accuracy} del 100\% mientras que en otra clase nos da un resultado mucho menor, pero al hacer un promedio nos hace intuir que los resultados serán buenos. Aún así nos pueden dar una idea de cómo ajustar los parámetros durante el entrenamiento.


\subsection{Resultados}\label{resultados_clasificacion}

El objetivo principal era explorar las diferentes combinaciones de redes de clasificación y su empleo para que el vehículo sea capaz de conducir solo. Por este motivo, el vehículo se ha probado en cada uno de los entornos mencionados en la Sección \ref{modelos_circuitos} con cada una de las combinaciones. Se han creado unas tablas con los resultados de cada combinación, donde se indican el porcentaje de circuito recorrido y el tiempo que ha tardado el vehículo en recorrer el circuito completo si se da el caso.\\

En un primer momento se contempló la combinación del empleo de 4 clases de velocidad lineal y 7 clases de velocidad de rotación. Estos resultados se muestran en las Tablas \ref{resultados_classificacion_recortada_4v_7w}, \ref{resultados_classificacion_normal_4v_7w}. En estas tablas se muestran los resultados para una combinación de redes sesgadas, otra combinación de redes balanceadas y otra de redes desbalanceadas (según el entrenamiento realizado). En la Tabla \ref{resultados_classificacion_recortada_4v_7w} se muestran los resultados de las redes entrenadas con imágenes de entrada recortadas. En la Tabla \ref{resultados_classificacion_normal_4v_7w} las redes han sido entrenadas con las imágenes de entrada completas (sin recortar).\\

\begin{table}[H]
\centering
\caption{Resultados de conducción con redes de clasificación (4 clases de v y 7 de w, imagen recortada)}
\label{resultados_classificacion_recortada_4v_7w}
\begin{tabular}{c|c|c|c|c|c|c|c|}
\cline{2-8}
                          & \multicolumn{1}{c|}{Manual} & \multicolumn{2}{c|}{4v+7w sesgada} & \multicolumn{2}{c|}{4v+7w balanceada} & \multicolumn{2}{c|}{4v+7w desbalanceada} \\ \cline{1-8} 
                        \multicolumn{1}{|c|}{Circuitos}    & Tiempo       & \%       & Tiempo       & \%        & Tiempo       & \%      & Tiempo     \\ \hline
\multicolumn{1}{|c|}{pistaSimple (h)}    & 1' 35''           & 100 \%         & 1' 38''           & 98 \%          &            & 100 \%       & 1' 42''      \\ \hline
\multicolumn{1}{|c|}{pistaSimple (ah)}     & 1' 33''           & 100 \%          & 1' 38''            & 100 \%           & 1' 41''           & 100 \%       & 1' 39''       \\ \hline
\multicolumn{1}{|c|}{monacoLine (h)}      & 1' 15''           & 5 \%           &             & 5 \%       &             & 5 \%       &           \\ \hline
\multicolumn{1}{|c|}{monacoLine (ah)}       & 1' 15''            & 5 \%       &             & 5 \%           &             & 5 \%          &      \\ \hline
\multicolumn{1}{|c|}{nurburgrinLine (h)}      & 1' 02''            & 8 \%          &            & 8 \%        &           & 8 \%       &     \\ \hline
\multicolumn{1}{|c|}{nurburgrinLine (ah)}       & 1' 02''           & 90 \%           &           & 80 \%        &            & 80 \%       &       \\ \hline
\multicolumn{1}{|c|}{curveGP (h)}     & 2' 13''           & 100 \%           & 2' 19''            & 100 \%        & 2' 03''           & 100 \%       & 2' 20''      \\ \hline
\multicolumn{1}{|c|}{curveGP (ah)}       & 2' 09''            & 100 \%         & 2' 12''            & 3 \%        &           & 100 \%      & 2' 21''     \\ \hline
\multicolumn{1}{|c|}{pista\_simple (h)}       & 1' 00''           & 100 \%          & 1' 04''            & 100 \%        & 1' 01''             & 100 \%      & 1' 04''       \\ \hline
\multicolumn{1}{|c|}{pista\_simple (ah)}     & 59''            & 100 \%          & 1' 04''          & 100 \%        & 59''             & 100 \%      & 1' 05''        \\ \hline
\end{tabular}
\end{table}


\begin{table}[H]
\centering
\caption{Resultados de conducción con redes de clasificación (4 clases de v y 7 de w, imagen completa)}
\label{resultados_classificacion_normal_4v_7w}
\begin{tabular}{c|c|c|c|c|c|c|c|}
\cline{2-8}
                          & \multicolumn{1}{c|}{Manual} & \multicolumn{2}{c|}{4v+7w sesgada} & \multicolumn{2}{c|}{4v+7w balanceada} & \multicolumn{2}{c|}{4v+7w desbalanceada} \\ \cline{1-8} 
                        \multicolumn{1}{|c|}{Circuitos}    & Tiempo       & \%       & Tiempo       & \%        & Tiempo       & \%      & Tiempo     \\ \hline
\multicolumn{1}{|c|}{pistaSimple (h)}    & 1' 35''           & 70 \%         &            & 7 \%          &            & 40 \%       &       \\ \hline
\multicolumn{1}{|c|}{pistaSimple (ah)}     & 1' 33''           & 10 \%          &             & 7 \%           &           & 10 \%       &       \\ \hline
\multicolumn{1}{|c|}{monacoLine (h)}      & 1' 15''           & 5 \%           &            & 5 \%       &             & 5 \%       &           \\ \hline
\multicolumn{1}{|c|}{monacoLine (ah)}       & 1' 15''            & 5 \%       &             & 5 \%           &             & 5 \%          &        \\ \hline
\multicolumn{1}{|c|}{nurburgrinLine (h)}      & 1' 02''            & 8 \%          &            & 8 \%        &             & 8 \%       &       \\ \hline
\multicolumn{1}{|c|}{nurburgrinLine (ah)}       & 1' 02''           & 8 \%           &            & 8 \%        &            & 8 \%       &        \\ \hline
\multicolumn{1}{|c|}{curveGP (h)}     & 2' 13''           & 100 \%           & 2' 12''           & 25 \%        &            & 85 \%       &      \\ \hline
\multicolumn{1}{|c|}{curveGP (ah)}       & 2' 09''            & 100\%         & 2' 11''            & 8 \%        &            & 75 \%      &      \\ \hline
\multicolumn{1}{|c|}{pista\_simple (h)}       & 1' 00''           & 12 \%          &           & 10 \%        &             & 100 \%      & 1' 07''       \\ \hline
\multicolumn{1}{|c|}{pista\_simple (ah)}     & 59''            & 25 \%          &          & 12 \%        &             & 100 \%      & 1' 09''        \\ \hline
\end{tabular}
\end{table}

En los resultados de estas tablas se puede observar que los resultados obtenidos para las imágenes de entrada recortadas son bastante mejores que para las imágenes completas para cada una de las redes. Aún así vemos que ninguna de estas redes es capaz de completar todos los circuitos, aunque la red sesgada es la que mejor resultado obtiene. En el caso de la red 4v+7w sesgada de imagen recortada, se puede observar que completa 3 de los circuitos en ambos sentidos, pero los otros 2 no es capaz de completarlos. Esto se debe a que estos dos circuitos poseen curvas más abruptas, donde o bien el vehículo deberá aplicar una velocidad de rotación mayor o una velocidad de tracción menor.\\

Al estudiar los resultados anteriores se contempló realizar una combinación de 4 clases de velocidad lineal y 9 clases de velocidad de rotación. Los resultados se muestran en las Tablas \ref{resultados_classificacion_recortada_4v_9w} (imagen recortada) y \ref{resultados_classificacion_completa_4v_9w} (imagen completa) para cada una de las redes entrenadas.\\


\begin{table}[H]
\centering
\caption{Resultados de conducción con redes de clasificación (4 clases de v y 9 de w, imagen recortada)}
\label{resultados_classificacion_recortada_4v_9w}
\begin{tabular}{c|c|c|c|c|c|c|c|}
\cline{2-8}
                          & \multicolumn{1}{c|}{Manual} & \multicolumn{2}{c|}{4v+9w sesgada} & \multicolumn{2}{c|}{4v+9w balanceada} & \multicolumn{2}{c|}{4v+9w desbalanceada} \\ \cline{1-8} 
                        \multicolumn{1}{|c|}{Circuitos}    & Tiempo       & \%       & Tiempo       & \%        & Tiempo       & \%      & Tiempo     \\ \hline
\multicolumn{1}{|c|}{pistaSimple (h)}    & 1' 35''           & 100 \%   & 1' 42''           & 95 \%         &        & 100 \%       & 1' 45''       \\ \hline
\multicolumn{1}{|c|}{pistaSimple (ah)}     & 1' 33''           & 100 \%       & 1' 39''            & 100 \%           & 1' 40''    & 100 \%     & 1' 41''\\ \hline
\multicolumn{1}{|c|}{monacoLine (h)}      & 1' 15''           & 5 \%      &             & 5 \%    &         & 5 \%      &           \\ \hline
\multicolumn{1}{|c|}{monacoLine (ah)}       & 1' 15''            & 12 \%     &             & 5 \%          &        & 5 \%         &      \\ \hline
\multicolumn{1}{|c|}{nurburgrinLine (h)}      & 1' 02''            & 8 \%         &         & 8 \%       &    & 8 \%      &     \\ \hline
\multicolumn{1}{|c|}{nurburgrinLine (ah)}       & 1' 02''           & 80 \%          &           & 80 \%      &            & 80 \%       &       \\ \hline
\multicolumn{1}{|c|}{curveGP (h)}     & 2' 13''      & 100 \%      & 2' 17''          & 100 \%        & 2' 03''          & 100 \%      & 2' 17''   \\ \hline
\multicolumn{1}{|c|}{curveGP (ah)}       & 2' 09''    & 100 \%        & 2' 13''        & 100 \%    & 2' 02''    & 100 \%       & 2' 15''     \\ \hline
\multicolumn{1}{|c|}{pista\_simple (h)}       & 1' 00''           & 100 \%        & 1' 04''          & 100 \%       & 59''            & 100 \%       & 1' 04''   \\ \hline
\multicolumn{1}{|c|}{pista\_simple (ah)}     & 59''            & 100 \%    & 1' 02''   & 100 \%        & 1' 00''     & 100 \%     & 1' 08''    \\ \hline
\end{tabular}
\end{table}


\begin{table}[H]
\centering
\caption{Resultados de conducción con redes de clasificación (4 clases de v y 9 de w, imagen completa)}
\label{resultados_classificacion_completa_4v_9w}
\begin{tabular}{c|c|c|c|c|c|c|c|}
\cline{2-8}
                          & \multicolumn{1}{c|}{Manual} & \multicolumn{2}{c|}{4v+9w sesgada} & \multicolumn{2}{c|}{4v+9w balanceada} & \multicolumn{2}{c|}{4v+9w desbalanceada} \\ \cline{1-8} 
                        \multicolumn{1}{|c|}{Circuitos}    & Tiempo       & \%       & Tiempo       & \%        & Tiempo       & \%      & Tiempo     \\ \hline
\multicolumn{1}{|c|}{pistaSimple (h)}    & 1' 35''        & 5 \%         &           & 5 \%         &            & 5 \%        &       \\ \hline
\multicolumn{1}{|c|}{pistaSimple (ah)}     & 1' 33''           &  5 \%          &             & 5 \%           &           & 5 \%      & \\ \hline
\multicolumn{1}{|c|}{monacoLine (h)}      & 1' 15''           & 3 \%         &       & 2 \%     &         & 2 \%       &           \\ \hline
\multicolumn{1}{|c|}{monacoLine (ah)}       & 1' 15''            & 3 \%     &             & 2 \%           &       & 2 \%          &      \\ \hline
\multicolumn{1}{|c|}{nurburgrinLine (h)}      & 1' 02''            &  3 \%        &         & 2 \%        &       & 2 \%       &     \\ \hline
\multicolumn{1}{|c|}{nurburgrinLine (ah)}       & 1' 02''           & 2 \%         &           & 2 \%        &         & 2 \%       &       \\ \hline
\multicolumn{1}{|c|}{curveGP (h)}     & 2' 13''           & 4 \%      &            & 3 \%         &       & 2 \%       &       \\ \hline
\multicolumn{1}{|c|}{curveGP (ah)}       & 2' 09''            &  3 \%        &      & 2 \%    &        & 2 \%       &     \\ \hline
\multicolumn{1}{|c|}{pista\_simple (h)}       & 1' 00''           & 4 \%           &        & 3 \%      &     & 3 \%       &   \\ \hline
\multicolumn{1}{|c|}{pista\_simple (ah)}     & 59''            & 4 \%       &         & 3 \%         &       & 5 \%      &         \\ \hline
\end{tabular}
\end{table}

En los resultados obtenidos para las redes 4v+9w se puede observar de nuevo que los resultados son mucho mejores para las redes entrenadas con la imagen recortada que con la imagen completa. Además, se puede ver que los mejores resultados se logran con la red 4v+9w sesgada, aunque aún así no se ha conseguido completar todos los circuitos en ambos sentidos. Llegamos a la conclusión de que aunque empleemos un mayor número de clases para la velocidad de rotación, no es suficiente para un buen rendimiento en la conducción.\\

La conclusión obtenida de los resultados anteriores es que este fallo se produce debido a que en los datos hay casos donde la velocidad es negativa, es decir, el vehículo da marcha atrás para poder girar sin chocar con la valla. Por este motivo se ha estudiado la combinación de 7 clases de velocidad de rotación y 5 clases de velocidad de tracción, donde se contemplan valores negativos.\\

Los resultados de las redes 5v+7w se pueden ver en las Tablas \ref{resultados_classificacion_recortada_5v_7w} y \ref{resultados_classificacion_normal_5v_7w}, donde la primera se corresponde con las redes entrenadas con la imagen recortada y la segunda con las redes entrenadas con la imagen completa.\\


\begin{table}[H]
\centering
\caption{Resultados de conducción con redes de clasificación (5 clases de v y 7 de w, imagen recortada)}
\label{resultados_classificacion_recortada_5v_7w}
\begin{tabular}{c|c|c|c|c|c|c|c|}
\cline{2-8}
                          & \multicolumn{1}{c|}{Manual} & \multicolumn{2}{c|}{5v+7w sesgada} & \multicolumn{2}{c|}{5v+7w balanceada} & \multicolumn{2}{c|}{5v+7w desbalanceada} \\ \cline{1-8} 
                        \multicolumn{1}{|c|}{Circuitos}    & Tiempo       & \%       & Tiempo       & \%        & Tiempo       & \%      & Tiempo     \\ \hline
\multicolumn{1}{|c|}{pistaSimple (h)}    & 1' 35''           & 100 \%         & 1' 41''           & 75 \%          &            & 100 \%       & 1' 42''      \\ \hline
\multicolumn{1}{|c|}{pistaSimple (ah)}     & 1' 33''           & 100 \%          & 1' 39''            & 100 \%           & 1' 39''           & 100 \%       & 1' 43''       \\ \hline
\multicolumn{1}{|c|}{monacoLine (h)}      & 1' 15''           & 100 \%           & 1' 20''            & 70 \%       &             & 85 \%       &           \\ \hline
\multicolumn{1}{|c|}{monacoLine (ah)}       & 1' 15''            & 100 \%       & 1' 18''            & 8 \%           &             & 100 \%          & 1' 20''       \\ \hline
\multicolumn{1}{|c|}{nurburgrinLine (h)}      & 1' 02''            & 100 \%          & 1' 03''            & 100 \%        & 1' 03''            & 100 \%       & 1' 05''      \\ \hline
\multicolumn{1}{|c|}{nurburgrinLine (ah)}       & 1' 02''           & 100 \%           & 1' 05''           & 80 \%        &            & 80 \%       &       \\ \hline
\multicolumn{1}{|c|}{curveGP (h)}     & 2' 13''           & 100 \%           & 2' 06''            & 97 \%        &            & 100 \%       & 2' 15''      \\ \hline
\multicolumn{1}{|c|}{curveGP (ah)}       & 2' 09''            & 100 \%         & 2' 11''            & 100 \%        & 2' 05''           & 100 \%      & 2' 15''     \\ \hline
\multicolumn{1}{|c|}{pista\_simple (h)}       & 1' 00''           & 100 \%          & 1' 02''            & 100 \%        & 1' 02''             & 100 \%      & 1' 01''       \\ \hline
\multicolumn{1}{|c|}{pista\_simple (ah)}     & 59''            & 100 \%          & 1' 03''          & 100 \%        & 1' 03''             & 100 \%      & 1' 04''        \\ \hline
\end{tabular}
\end{table}


\begin{table}[H]
\centering
\caption{Resultados de conducción con redes de clasificación (5 clases de v y 7 de w, imagen completa)}
\label{resultados_classificacion_normal_5v_7w}
\begin{tabular}{c|c|c|c|c|c|c|c|}
\cline{2-8}
                          & \multicolumn{1}{c|}{Manual} & \multicolumn{2}{c|}{5v+7w sesgada} & \multicolumn{2}{c|}{5v+7w balanceada} & \multicolumn{2}{c|}{5v+7w desbalanceada} \\ \cline{1-8} 
                        \multicolumn{1}{|c|}{Circuitos}    & Tiempo       & \%       & Tiempo       & \%        & Tiempo       & \%      & Tiempo     \\ \hline
\multicolumn{1}{|c|}{pistaSimple (h)}    & 1' 35''           & 35 \%         &            & 10 \%          &            & 90 \%       &       \\ \hline
\multicolumn{1}{|c|}{pistaSimple (ah)}     & 1' 33''           & 100 \%          & 1' 49''            & 100 \%           & 1' 46''           & 90 \%       &       \\ \hline
\multicolumn{1}{|c|}{monacoLine (h)}      & 1' 15''           & 100 \%           & 1' 24''            & 5 \%       &             & 100 \%       &  1' 23''          \\ \hline
\multicolumn{1}{|c|}{monacoLine (ah)}       & 1' 15''            & 100 \%       & 1' 29''            & 8 \%           &             & 100 \%          & 1' 24''       \\ \hline
\multicolumn{1}{|c|}{nurburgrinLine (h)}      & 1' 02''            & 100 \%          & 1' 10''            & 8 \%        &             & 90 \%       &       \\ \hline
\multicolumn{1}{|c|}{nurburgrinLine (ah)}       & 1' 02''           & 100 \%           & 1' 07''           & 8 \%        &            & 100 \%       & 1' 09''       \\ \hline
\multicolumn{1}{|c|}{curveGP (h)}     & 2' 13''           & 95 \%           &            & 80 \%        &            & 25 \%       &      \\ \hline
\multicolumn{1}{|c|}{curveGP (ah)}       & 2' 09''            & 7 \%         &             & 3 \%        &            & 20 \%      &      \\ \hline
\multicolumn{1}{|c|}{pista\_simple (h)}       & 1' 00''           & 8 \%          &           & 8 \%        &             & 100 \%      & 1' 08''       \\ \hline
\multicolumn{1}{|c|}{pista\_simple (ah)}     & 59''            & 12 \%          &          & 12 \%        &             & 100 \%      & 1' 08''        \\ \hline
\end{tabular}
\end{table}

En las tablas anteriores se puede concluir de nuevo que los resultados obtenidos con imágenes de entrada son mejores que con las imágenes completas. Además, las redes sesgadas también obtienen mejor resultado que las balanceadas y las desbalanceadas. En la Tabla \ref{resultados_classificacion_recortada_5v_7w} es posible observar una red con la cual somos capaces de completar el circuito completo para todo los circuitos de los que disponemos. Esta red es la que hemos denominado 5v+7w sesgada (imagen recortada). \\

\begin{figure}
\begin{center}
	\includegraphics[width=0.8\textwidth]{figures/Clasificacion/nurburgrin_class.png}
   \caption{Pilotaje del coche en el circuito nurburgrinLine}
	\label{fig.nurburgrin_class}
\end{center}
\end{figure}

\begin{figure}
\begin{center}
	\includegraphics[width=0.8\textwidth]{figures/Clasificacion/monaco_class.png}
   \caption{Pilotaje del coche en el circuito monacoLine}
	\label{fig.monaco_class}
\end{center}
\end{figure}

Un dato a tener en cuenta es que si nos fijamos en la columna ``Manual'' se pueden ver los tiempos realizados por el piloto manual, mientras que si nos fijamos en la columna de tiempo de la red 5v+7w sesgada se ven los tiempos logrados con esta red. Los tiempos obtenidos del pilotaje mediante esta red no se encuentran muy lejanos a los resultados del piloto manual. Esto permite concluir que esta red aprende de forma correcta a conducir de forma autónoma. El resultado de la conducción del vehículo de forma satisfactoria se puede ver en las Figuras \ref{fig.nurburgrin_class} y \ref{fig.monaco_class}.\\



\section{Conclusiones}

En este proyecto se ha logrado que un vehículo sea capaz de conducir de forma autónoma mediante redes de clasificación. Este es el caso de la red 5v+7w sesgada (imagen recortada).\\

Gracias a los diferentes experimentos realizados y los resultados obtenidos, se pueden sacar algunas conclsiones acerca del entrenamiento de estas redes y de la cuantización de las clases.\\

En primer lugar, se ha llegado a la conclusión de que tanto el número de clases como el rango de valores que abarca cada una de las clases tiene una gran influencia en el rendimiento del problema planteado. Se establece que con 7 clases de velocidad de rotación es suficiente para lograr una buena conducción, y que no es necesario emplear 9 clases para conseguir el objetivo. Sin embargo, en el caso de las clases de velocidad lineal, no es suficiente con emplear 4 clases, hay que añadir una quinta clase para que el coche sea capaz de tomar la velocidad necesaria en cada caso.\\

En segundo lugar, los resultados muestran que es mejor emplear una imagen de entrada recortada a una imagen completa. Esto se debe a que la red posiblemente se distraiga con la parte superior de la imagen (el cielo) y no consiga centrar su atención en la parte más importante de la imagen (la carretera).\\

En tercer lugar, se concluye que son muy importantes los datos de entrenamiento para conseguir una red efectiva. En los resultados se ha podido observar que el mejor entrenamiento es con las redes sesgadas, es decir, donde le damos pesos a las clases. De esta forma podemos darle más importancia a las clases pertenecientes a curvas más abruptas que a las rectas. Así nuestro vehículo aprende mejor las relaciones visuales con las clases. Por otra parte con las redes entrenadas con el conjunto de datos desbalanceado es normal que no consigamos el mejor resultado debido a que tenemos muchos más datos para algunas clases que para otras. Sin embargo, aunque podría parecer que las redes entrenadas con el \textit{dataset} balanceado podrían dar el mejor resultado, esto no sucede así. La razón de este problema es que aunque poseeamos el mismo número de datos para cada una de las clases, no tenemos un gran número de imágenes por cada clase, lo que hace que el vehículo no sea capaz de aprender a conducir. Es decir, es necesario tener un amplio conjunto de entrenamiento que nos permita aprender las relaciones que deseamos.\\

En cuarto lugar, se ha comprobado que aunque obtengamos buenos resultados en las métricas de evaluación para cada una de las clases, es posible que en algunos casos con estas redes el vehículo choque contra la valla, y en cambio en otros casos donde se obtengan peores resultados el vehículo será capaz de conducir. Esto se debe a que cuando estamos pilotando el coche, puede ser que si predecimos mal un valor no implique mucha desviación del coche de la línea roja. Pero, sin embargo, si la red en un instante dado predice 3 o 4 valores seguidos mal, el coche se irá desviando cada vez más y no será capaz de retificar para volver a la recta. Esto implica que el entrenaminto de redes neuronales para conducción autónoma sea algo complejo y necesite mucha experimentación antes de lograr un buen resultado.
\lhead[]{CAPÍTULO \thechapter. REDES DE REGRESIÓN}
\chapter{Redes de regresión}\label{cap.regresion}

\section{Contexto}

\section{Arquitecturas de red}

\subsection{PilotNet}

\begin{figure}
\begin{center}
	\includegraphics[width=0.5\textwidth]{figures/Regresion/pilotnet.png}
   \caption{Arquitectura PilotNet.}
	\label{fig.Pilotnet}
\end{center}
\end{figure}


\subsection{TinyPilotNet}

\begin{figure}
\begin{center}
	\includegraphics[width=0.5\textwidth]{figures/Regresion/tinypilotnet.png}
   \caption{Arquitectura TinyPilotNet.}
	\label{fig.TinyPilotNet}
\end{center}
\end{figure}

\subsection{ControlNet}

\section{Experimentos}

A continuación se explicaán todos los experimentos realizados durante el entrenamiento de redes de regresión, tanto los relacionados con las dimensiones y los tipos de las imágenes, preprocesado de los datos, etc. \\

\subsection{Preprocesado de los datos}

Los datos de entrenamiento son una parte fundamental del aprendizaje por parte de las redes neuronales. En algunas ocasiones es díficil extrar una gran cantidad de datos, por este motivo se suele hacer un aumentado de datos. El aumentado de datos consiste en aumentar la información de la red a partir del conjunto de datos.\\

En el entrenamiento se ha realizado un preprocesado para aumentar los datos de los que disponemos, y de esta forma tener los mismos datos de velocidad de rotación a la izquierda que a la derecha. Es decir, lo que se ha hecho exactamente es emplear la operación \textit{cv2.flip} de OpenCV, que consiste en hacer un espejo de nuestra imagen. De esta forma para una imagen en la que giramos a la izquierda (velocidad de rotación 0.5), ahora dispondríamos de la imagen correspondiente pero a la derecha (velocidad de rotación -0.5).\\

En la Figura \ref{fig.image_camera} se puede ver una imagen obtenida por la cámara del coche, mientras que en la Figura \ref{fig.image_flip} se puede observar la misma imagen tras aplicar la operación \textit{flip} de OpenCV.\\


\begin{figure}
\begin{center}
	\includegraphics[width=0.5\textwidth]{figures/Regresion/img_normal.png}
   \caption{Imagen de la cámara}
	\label{fig.image_camera}
\end{center}
\end{figure}

\begin{figure}
\begin{center}
	\includegraphics[width=0.5\textwidth]{figures/Regresion/img_flip.png}
   \caption{Imagen tras realizar la operación \textit{flip}}
	\label{fig.image_flip}
\end{center}
\end{figure}


\subsection{Dimensiones imagen y tipo de imagen}

\begin{figure}
\begin{center}
	\includegraphics[width=0.5\textwidth]{figures/Regresion/img_normal.png}
   \caption{Imagen completa}
	\label{fig.completa_reg}
\end{center}
\end{figure}

\begin{figure}
\begin{center}
	\includegraphics[width=0.5\textwidth]{figures/Regresion/img_cropped_reg.png}
   \caption{Imagen recortada}
	\label{fig.recortada_reg}
\end{center}
\end{figure}

\begin{figure}
\begin{center}
	\includegraphics[width=0.5\textwidth]{figures/Regresion/dif_gray_128.png}
   \caption{Imagen diferencia}
	\label{fig.diferencia_reg}
\end{center}
\end{figure}


\subsection{Métricas de evaluación}

\begin{table}[H]
\centering
\caption{Métricas de test de redes de regresión (v, imagen recortada)}
\label{metricas_regresion_recortada_v}
\begin{tabular}{c|c|c|}
\cline{1-3}
                        \multicolumn{1}{|c|}{Red}    & Mean squared error       & Mean absolute error             \\ \hline
\multicolumn{1}{|c|}{PilotNet}    & 0.162276   & 0.104023   \\ \hline
\multicolumn{1}{|c|}{TinyPilotNet}     & 0.653815      & 0.424191   \\ \hline
\multicolumn{1}{|c|}{PilotNet (stacked)}   & 1.081069    & 0.489701   \\ \hline
\multicolumn{1}{|c|}{PilotNet (stacked, dif)}     & 1.292020    & 0.551575        \\ \hline
\end{tabular}
\end{table}


\begin{table}[H]
\centering
\caption{Métricas de test de redes de regresión (w, imagen recortada)}
\label{metricas_regresion_recortada_w}
\begin{tabular}{c|c|c|}
\cline{1-3}
                        \multicolumn{1}{|c|}{Red}    & Mean squared error       & Mean absolute error             \\ \hline
\multicolumn{1}{|c|}{PilotNet}    & 0.000312   &  0.006651   \\ \hline
\multicolumn{1}{|c|}{TinyPilotNet}     & 0.001290      & 0.023038   \\ \hline
\multicolumn{1}{|c|}{PilotNet (stacked)}   & 0.006833    & 0.043313  \\ \hline
\multicolumn{1}{|c|}{PilotNet (stacked, dif)}     & 0.002911    & 0.022799        \\ \hline
\end{tabular}
\end{table}




\begin{table}[H]
\centering
\caption{Métricas de test de redes de regresión (v, imagen completa)}
\label{metricas_regresion_completa_v}
\begin{tabular}{c|c|c|}
\cline{1-3}
                        \multicolumn{1}{|c|}{Red}    & Mean squared error       & Mean absolute error             \\ \hline
\multicolumn{1}{|c|}{PilotNet}    & 0.205252    & 0.142394    \\ \hline
\multicolumn{1}{|c|}{TinyPilotNet}     & 0.459055      & 0.296804   \\ \hline
\multicolumn{1}{|c|}{PilotNet (stacked)}   & 1.084940    & 0.469273  \\ \hline
\multicolumn{1}{|c|}{PilotNet (stacked, dif)}     & 1.889947    & 0.468693        \\ \hline
\multicolumn{1}{|c|}{Temporal (dif)}      & 0.442861     & 0.316337   \\ \hline
\end{tabular}
\end{table}

\begin{table}[H]
\centering
\caption{Métricas de test de redes de regresión (w, imagen completa)}
\label{metricas_regresion_completa_w}
\begin{tabular}{c|c|c|}
\cline{1-3}
                        \multicolumn{1}{|c|}{Red}    & Mean squared error       & Mean absolute error             \\ \hline
\multicolumn{1}{|c|}{PilotNet}    & 0.000316   & 0.007184    \\ \hline
\multicolumn{1}{|c|}{TinyPilotNet}     & 0.001366      & 0.022281   \\ \hline
\multicolumn{1}{|c|}{PilotNet (stacked)}   & 0.004471    & 0.034673   \\ \hline
\multicolumn{1}{|c|}{PilotNet (stacked, dif)}     & 0.005292    & 0.031134        \\ \hline
\multicolumn{1}{|c|}{Temporal (dif)}      & 0.002155     & 0.029771   \\ \hline
\end{tabular}
\end{table}




\subsection{Resultados}

\begin{table}[H]
\centering
\caption{Resultados de conducción con redes neuronales de regresión (imagen recortada)}
\label{resultados_regresion_recortada}
\begin{tabular}{c|c|c|c|c|c|}
\cline{2-6}
                          & \multicolumn{1}{c|}{Manual} & \multicolumn{2}{c|}{PilotNet} & \multicolumn{2}{c|}{TinyPilotNet} \\ \cline{1-6} 
                        \multicolumn{1}{|c|}{Circuitos}    & Tiempo       & \%       & Tiempo       & \%        & Tiempo         \\ \hline
\multicolumn{1}{|c|}{pistaSimple (h)}    & 1' 35''           & 100 \%         & 1' 37''       &  100 \%        & 1' 41''               \\ \hline
\multicolumn{1}{|c|}{pistaSimple (ah)}     & 1' 33''           & 100 \%          & 1' 38''           & 100 \%        & 1' 41''      \\ \hline
\multicolumn{1}{|c|}{monacoLine (h)}      & 1' 15''           & 100 \%            & 1' 20''            & 100 \%         & 1' 19''                \\ \hline
\multicolumn{1}{|c|}{monacoLine (ah)}       & 1' 15''       &  100 \%      & 1' 19''         & 100 \%          & 1' 18''         \\ \hline
\multicolumn{1}{|c|}{nurburgrinLine (h)}      & 1' 02''       &  100 \%         & 1' 04''           & 100 \%        & 1' 04''       \\ \hline
\multicolumn{1}{|c|}{nurburgrinLine (ah)}       & 1' 02''     & 100 \%         & 1' 06''          & 100 \%     & 1' 05''            \\ \hline
\multicolumn{1}{|c|}{curveGP (h)}     & 2' 13''           & 100 \%         & 2' 16''            & 25 \%        &              \\ \hline
\multicolumn{1}{|c|}{curveGP (ah)}       & 2' 09''            & 100 \%         & 2' 12''        & 75 \%        &         \\ \hline
\multicolumn{1}{|c|}{pista\_simple (h)}       & 1' 00''           & 100 \%       & 1' 04''            & 100 \%         & 59''        \\ \hline
\multicolumn{1}{|c|}{pista\_simple (ah)}     & 59''          & 100 \%       & 1' 05''         & 100 \%        & 1' 00'                 \\ \hline
\end{tabular}
\end{table}


\begin{table}[H]
\centering
\caption{Resultados de conducción con redes neuronales de regresión introduciendo temporalidad (imagen recortada)}
\label{resultados_regresion_temporal_recortada}
\begin{tabular}{c|c|c|c|c|c|c|c|}
\cline{2-6}
                          & \multicolumn{1}{c|}{Manual} & \multicolumn{2}{c|}{PilotNet (stacked)} & \multicolumn{2}{c|}{PilotNet (stacked, dif)} \\ \cline{1-6} 
                        \multicolumn{1}{|c|}{Circuitos}    & Tiempo       & \%       & Tiempo       & \%        & Tiempo      \\ \hline
\multicolumn{1}{|c|}{pistaSimple (h)}    & 1' 35''     & 100 \%   & 1' 41''    & 100 \%   & 1' 39''          \\ \hline
\multicolumn{1}{|c|}{pistaSimple (ah)}     & 1' 33''           & 10 \%        &        & 100 \%     & 1' 38''   \\ \hline
\multicolumn{1}{|c|}{monacoLine (h)}      & 1' 15''           & 85 \%         &        & 45 \%       &            \\ \hline
\multicolumn{1}{|c|}{monacoLine (ah)}       & 1' 15''       & 15 \%       &         & 5 \%           &         \\ \hline
\multicolumn{1}{|c|}{nurburgrinLine (h)}      & 1' 02''       & 8 \%      &       & 8 \%        &        \\ \hline
\multicolumn{1}{|c|}{nurburgrinLine (ah)}       & 1' 02''     & 80 \%     &        & 50 \%     &          \\ \hline
\multicolumn{1}{|c|}{curveGP (h)}     & 2' 13''     & 25 \%       &     & 25 \%     &              \\ \hline
\multicolumn{1}{|c|}{curveGP (ah)}       & 2' 09''            & 75 \%    &     & 75 \%       &       \\ \hline
\multicolumn{1}{|c|}{pista\_simple (h)}       & 1' 00''           & 30 \%       &       & 100 \%      & 1' 06''    \\ \hline
\multicolumn{1}{|c|}{pista\_simple (ah)}     & 59''    & 10 \%    &          & 100 \%      & 1' 05''          \\ \hline
\end{tabular}
\end{table}





\begin{table}[H]
\centering
\caption{Resultados de conducción con redes neuronales de regresión (imagen completa)}
\label{resultados_regresion_completa}
\begin{tabular}{c|c|c|c|c|c|}
\cline{2-6}
                          & \multicolumn{1}{c|}{Manual} & \multicolumn{2}{c|}{PilotNet} & \multicolumn{2}{c|}{TinyPilotNet} \\ \cline{1-6} 
                        \multicolumn{1}{|c|}{Circuitos}    & Tiempo       & \%       & Tiempo       & \%        & Tiempo         \\ \hline
\multicolumn{1}{|c|}{pistaSimple (h)}    & 1' 35''           & 100 \%         & 1' 41''       &  100 \%        & 1' 39''               \\ \hline
\multicolumn{1}{|c|}{pistaSimple (ah)}     & 1' 33''           & 100 \%          & 1' 39''           & 100 \%        & 1' 38''      \\ \hline
\multicolumn{1}{|c|}{monacoLine (h)}      & 1' 15''           & 100 \%            & 1' 21''            & 100 \%         & 1' 19''                \\ \hline
\multicolumn{1}{|c|}{monacoLine (ah)}       & 1' 15''       &  100 \%      & 1' 23''         & 100 \%          & 1' 20''         \\ \hline
\multicolumn{1}{|c|}{nurburgrinLine (h)}      & 1' 02''       &  100 \%         & 1' 03''           & 100 \%        & 1' 05''       \\ \hline
\multicolumn{1}{|c|}{nurburgrinLine (ah)}       & 1' 02''     & 100 \%         & 1' 06''          & 100 \%     & 1' 06''            \\ \hline
\multicolumn{1}{|c|}{curveGP (h)}     & 2' 13''           & 100 \%         & 2' 20''            & 100 \%        & 2' 11''             \\ \hline
\multicolumn{1}{|c|}{curveGP (ah)}       & 2' 09''            & 100 \%         & 2' 16''        & 100 \%        & 2' 06''        \\ \hline
\multicolumn{1}{|c|}{pista\_simple (h)}       & 1' 00''           & 100 \%       & 1' 07''            & 100 \%         & 1' 02''        \\ \hline
\multicolumn{1}{|c|}{pista\_simple (ah)}     & 59''          & 100 \%       & 1' 09''         & 100 \%        & 1' 02'                 \\ \hline
\end{tabular}
\end{table}


\begin{table}[H]
\centering
\caption{Resultados de conducción con redes neuronales de regresión introduciendo temporalidad (imagen completa)}
\label{resultados_regresion_temporal_completa}
\begin{tabular}{c|c|c|c|c|c|c|c|}
\cline{2-8}
                          & \multicolumn{1}{c|}{Manual} & \multicolumn{2}{c|}{PilotNet (stacked)} & \multicolumn{2}{c|}{PilotNet (stacked, dif)} & \multicolumn{2}{c|}{Temporal (dif)} \\ \cline{1-8} 
                        \multicolumn{1}{|c|}{Circuitos}    & Tiempo       & \%       & Tiempo       & \%        & Tiempo  & \%        & Tiempo          \\ \hline
\multicolumn{1}{|c|}{pistaSimple (h)}    & 1' 35''      & 100 \%     & 1' 40''  & 100 \%     & 1' 43''   & 35 \%       &     \\ \hline
\multicolumn{1}{|c|}{pistaSimple (ah)}     & 1' 33''           & 100 \%   & 1' 46''    & 10 \%        &   & 10 \%       &  \\ \hline
\multicolumn{1}{|c|}{monacoLine (h)}      & 1' 15''           & 50 \%         &         & 5 \%           &     & 3 \%        &                 \\ \hline
\multicolumn{1}{|c|}{monacoLine (ah)}       & 1' 15''    & 7\%      &         & 5 \%          &    & 3 \%       &        \\ \hline
\multicolumn{1}{|c|}{nurburgrinLine (h)}      & 1' 02''       & 50 \%       &          & 8 \%        &     & 8 \%        &     \\ \hline
\multicolumn{1}{|c|}{nurburgrinLine (ah)}       & 1' 02''     & 80 \%    &      & 50 \%     &   & 3 \%        &            \\ \hline
\multicolumn{1}{|c|}{curveGP (h)}     & 2' 13''      & 25 \%       &      & 25 \%     &     &       &           \\ \hline
\multicolumn{1}{|c|}{curveGP (ah)}       & 2' 09''            & 100 \%     & 2' 07''     &  75 \%     &   &       &        \\ \hline
\multicolumn{1}{|c|}{pista\_simple (h)}       & 1' 00''           & 100 \%     & 1' 11'      & 100 \%       & 1' 03''    &       &         \\ \hline
\multicolumn{1}{|c|}{pista\_simple (ah)}     & 59''      & 100 \%     & 1' 08''     & 100 \%       & 1' 02''    &       &                \\ \hline
\end{tabular}
\end{table}


\section{Conclusiones}
\lhead[]{CAPÍTULO \thechapter. CONCLUSIONES}
\chapter{Conclusiones}\label{cap.conclusiones}

En este capítulo se exponen las conclusiones finales obtenidas, así como los posibles trabajos futuros, tras analizar las diferentes redes neuronales empleadas (redes de regresión y clasificación), y algunos experimentos realizados con las mismas.

\section{Conclusiones}

En este proyecto se ha alcanzado satisfactoriamente el objetivo global de explorar diferentes redes neuronales de clasificación y regresión para controlar la conducción de un vehículo mediante control visual, así como realizar diferentes experimentos con las mismas. El objetivo era lograr que el coche fuera capaz de conducir en diferentes entornos en el simulador Gazebo. Para ello se ha creado un nodo Piloto que nos ha permitido evaluar las redes en el control del vehículo.\\

En primer lugar, se ha experimentado con redes de clasificación, evaluando la influencia de la cuantización de las clases, es decir, tanto el número de clases como el rango de valores que abarca cada una de estas clases. Estudiando el efecto de estas cuantizaciones se ha conseguido que la red de clasificación 5v+7w sesgada (imagen recortada) fuera capaz de recorrer todos los circuitos de manera satisfactoria. Se han sacado algunas conclusiones acerca del entrenamiento de estas redes y de la cuantización de las clases:\\

\begin{itemize}
    \item El número de las clases y el rango de los valores que abarca cada clase tiene una gran influencia en el rendimiento de la conducción. El uso de 7 clases de velocidad de rotación y 5 clases de velocidad lineal es suficiente para lograr una buena conducción. En caso de tener menos clases el rendimiento sería deficiente, además de que es necesario ajustar el rango de valores de velocidad que abarca cada clase adecuadamente.
    
    \item El uso de una imagen recortada en este tipo de redes mejora el rendimiento, ya que la red no se distrae con información de la imagen que no es importante para la conducción.
    
    \item El entrenamiento más adecuado se realiza con redes sesgadas, es decir, ponderando las clases para dar más influencia a aquellas clases donde tenemos menor número de datos o clases que se corresponden con situaciones más difíciles para el coche.
\end{itemize}

El segundo subobjetivo era la experimentación con redes de regresión, evaluando el uso de diferentes arquitecturas, así como la influencia que tiene la imagen de entrada empleada. Experimentando con estos aspectos se ha logrado que las siguientes redes de regresión completen todos los circuitos: PilotNet (imagen recortada e imagen completa), TinyPilotNet (imagen completa) y DeepestLSTM-TinyPilotNet (imagen completa). Se han sacado algunas conclusiones acerca del entrenamiento de las redes con las que se ha experimentado, así como de la influencia de la imagen de entrada:\\

\begin{itemize}
    \item El uso de una imagen completa mejora el rendimiento de las redes de regresión, ya que estas imágenes disponen de información de las vallas que hay alrededor del circuito y que pueden dar idea de si el coche se está aproximando a ellas o no, por lo que debería disminuir la velocidad y girar en el sentido adecuado. En las imágenes recortadas no disponemos de esta información, aunque la red PilotNet si consigue lograr un resultado efectivo con este tipo de imágenes.
    
    \item El empleo de redes más profundas mejora el resultado de la conducción. Este es el caso de PilotNet y DeepestLSTM-TinyPilotNet frente a TinyPilotNet y LSTM-TinyPilotNet.
    
    \item Al introducir capas \textit{ConvLSTM2D} en las redes se observa un pilotaje más suave. Es el caso de la red DeepestLSTM-TinyPilotNet, donde vemos que el coche en casi todo momento intenta volver a situarse encima de la línea roja. Esto se debe al efecto de memoria introducido por la red, que permite que el vehículo tenga un mayor conocimiento acerca de la situación.
    
    \item Es complejo conseguir una imagen apilada perfecta para la conducción, ya que es necesario alcanzar un número de imágenes a apilar adecuado, así como un margen adecuado entre las mismas.
    
    \item Proporcionar una imagen diferencia con información relevante acerca de cómo ha variado la situación del vehículo o no ha cambiado (sigues en recta, entras en una curva, etc) es bastante complejo.
    
    \item Es difícil proporcionar información temporal a una red neuronal en una imagen, ya sea la imagen diferencia o la imagen apilada. El motivo es que no tenemos una idea de cómo interpreta esta información la red de forma interna, es decir, no sabemos qué partes de la imagen temporal considera relevantes y cuáles no. 
\end{itemize}


Además, a lo largo del proyecto se han extraído algunas conclusiones que sirven tanto para las redes de clasificación como para las redes de regresión empleadas para la conducción autónoma. Algunas de estas conclusiones son:\\

\begin{itemize}
    \item Los datos de entrenamiento que empleemos influirán ampliamente en la calidad de los resultados. Por este motivo es necesario disponer de un amplio conjunto de entreamiento que nos permita aprender de todas las situaciones a las que se pueda enfrentar el vehículo. En ocasiones es difícil disponer de muchos datos de situaciones más difíciles, por eso será necesario realizar algún procesado de los datos que nos permita balancear un poco los mismos. De esta forma las redes no aprenderán únicamente a conducir en situaciones simples.
    
    \item Buenos resultados en las métricas de evaluación no implican una buena conducción. Es posible que el empleo de redes con buenas métricas de como resultado que el vehículo choque contra la valla, y en cambio en otros casos donde se obtengan peores resultados el vehículo sea capaz de conducir.
    
    \item El entrenamiento de las redes neuronales de conducción autónoma es complejo y necesita mucha experimentación. Esto se debe a que debemos emplear una red de velocidad lineal y una red de velocidad angular para el pilotaje, y en algunas ocasiones esta relación puede fallar ya sea por parte de una red u otra. Además, en el pilotaje si predecimos un valor mal puede ser que no afecte mucho a la calidad de la conducción, pero si predecimos varios valores erroneamente de forma continua el coche se desviará de la línea roja chocando con la valla.
    
    \item Tanto en redes de clasificación como en redes de regresión se ha logrado que alguna red sea capaz de efectuar una conducción autónoma en todos los circuitos. Los tiempos de conducción conseguidos por todas las redes satisfactorias no se encuentran muy lejanos a los tiempos del piloto manual. Esto nos da una idea de que las redes están aprendiendo adecuadamente de los datos que poseen. Debido a esto se puede establecer que si el piloto manual en una situación difícil no actúa de forma exacta, la red tampoco predecirá un valor perfecto de conducción. Esta situación se puede dar en el proyecto, ya que el conjunto de datos está basado en un piloto manual que realiza una conducción autónoma basándose en visión, pero no es una ciencia exacta.
    
    \item Entrenar dos redes (velocidad de rotación y velocidad lineal) que conjuntamente sean efectivas en todos los entornos de los que se dispone es difícil, y por ello necesita bastante tiempo de experimentación. 
\end{itemize}


Además de alcanzar los dos subobjetivos, se han conseguido realizar otros requisitos implícitos en el uso de cada red, como emplear el simulador Gazebo y la creación del nodo Piloto, que permiten evaluar la conducción autónoma del vehículo mediante redes neuronales.\\

En cuanto a los aportes personales, hemos aprendido a entrenar diferentes redes neuronales para conducción autónoma basada en información visual. Además, se ha aprendido a emplear la información obtenida por los resultados para la mejora de próximos entrenamientos. Este proyecto además ha servido para comprender las diferentes fases en las que se divide un trabajo de esta envergadura. Gracias a ello se ha aprendido a dividir un gran objetivo en subobjetivos de ingeniería, permitiendo facilitar una solución más rápida a los mismos. Además, durante el proyecto han surgido problemas habituales de ingeniería, más complejos o mas sencillos, pero ha sido necesario solventarlos bien mediante más pruebas y experimentos, bien cambiando la técnica que se estaba empleando al entrenar, o bien refinando los modelos de redes neuronales empleados hasta obtener los objetivos deseados.\\


\section{Trabajos futuros}

Debido a que éste es un Trabajo de Fin de Máster no ha sido posible alargarlo ad in-finitum para realizar más mejoras sobre el mismo. A continuación, se detallarán posibles líneas concretas en las que se puede mejorar cada una de las redes.\\

En el proyecto se podrían llevar a cabo algunas mejoras, por eso se proponen las siguientes ideas:

\begin{itemize}
    \item Grabar una base de datos adicional de situaciones bastante complejas o situaciones que desconoce el vehículo. Algunos ejemplos podrían ser el vehículo ha perdido de vista la línea roja o se encuentra en una curva bastante abrupta que no ha vista con anterioridad. 
    
    \item Realizar un estudio más amplio de la cuantización de las clases (número de clases y rango de clase) permitiendo ajustar de forma más precisa el número de clases y el rango a los valores de los que disponemos en el conjunto de datos. Una posibilidad sería incluir un mayor número de clases para los ángulos de giro (de mayor valor) que se dan en las situaciones complejas (pérdida de carretera o curvas muy abruptas), ajustando de esta  forma más la conducción a este tipo de situaciones.
    
    \item Una dificultad encontrada en el proyecto ha sido cómo incorporar información temporal con imágenes apiladas. Una posible mejora es realizar un estudio más amplio que nos lleve a ajustar de manera más adecuada el número de imágenes que apilamos y el margen entre las mismas.
    
    \item Realizar un estudio más amplio de la imagen diferencia que nos permita introducir información temporal en una única imagen. El objetivo es buscar qué margen de tiempo entre las dos imágenes de las que se realiza la diferencia es el adecuado. Además, una posible mejora consiste en el estudio de la posibilidad de incluir algún preprocesado de imagen que nos permita resaltar la información que nos interasa en las imágenes. De esta forma las redes aprenderían la información que desde nuestro punto de vista es relevante.
    
    \item Crear un algoritmo que intente explicar lo que las redes aprenden y cómo toman sus decisiones. Este método fue desarrollado en el artículo ``Explaining  how  a  deep  neural network trained with end-to-end learning steers a car'' \cite{explaining-end2end} \cite{visual}, donde se intenta conocer cómo toma sus decisiones PilotNet. En este atículo se crea un método que determina qué elementos en la imagen influyen más en las decisiones de las velocidades de la red. A las secciones de la imagen con esta propiedad las denominan objetos salientes. La idea sería emplear este conocimiento de los objetos salientes de la imagen para poder ajustar parámetros de las redes y mejorar así la conducción. Además, esta idea nos podría ayudar a crear una imagen diferencia adecuada, ya que seríamos capaces de saber cuál es la información relevante para las redes.
\end{itemize}


%%%%%%%%%%%%%%% Bibliograí­a %%%%%%%%%%%%%%%
\nocite{*}
\bibliographystyle{unsrt}
\bibliography{Bibliografia.bib}
\addcontentsline{toc}{chapter}{Bibliografía}

\end{document}