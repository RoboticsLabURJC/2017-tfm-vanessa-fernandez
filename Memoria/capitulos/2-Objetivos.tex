\chapter{Objetivos}\label{cap.objetivos}

Una vez explicado el contexto de este proyecto, se describirán en este capítulo los objetivos, los requisitos y la metodología empleados.

\section{Objetivos}

El propósito principal de este proyecto es el estudio de diferentes redes neuronales basadas en información visual que permitan a un vehículo ser capaz de conducir de forma autónoma. El coche deberá ser capaz de conducir en diferentes circuitos en el simulador Gazebo. Los entornos de entrenamiento y de prueba serán diferentes para que el vehículo sea capaz de aprender diferentes estímulos que le permitan conducir en situaciones variadas.\\

Este objetivo genérico se ha articulado en cuatro subobjetivos concretos:

\begin{enumerate}
    \item Se creará una aplicación de control visual con la infraestructura necesaria que se comunica con el simulador Gazebo, donde se podrá ver el resultado de la predicción de las redes neuronales. Esta aplicación tendrá diferentes ingredientes: 

    \begin{itemize}
        \item Infraestructura de conexión en el simulador, tanto para recoger imágenes de la cámara a bordo del coche como para enviar órdenes al acelerador y al volante.
        \item Permitirá cargar y emplear diferentes redes neuronales, además de incluir una GUI.
        \item En un fichero específico se guarda el código que genera las órdenes de velocidad predichas por la red neuronal y se envían al coche simulado.
    \end{itemize}
    
    \item Se crearán varias bases de datos propias de conducción autónoma con un algoritmo de pilotaje basado en visión sencillo, que permitan el entrenamiento supervisado de diferentes redes neuronales.
    
    \item Se realizará un estudio y se llevarán a cabo pruebas de diferentes modelos de redes neuronales convolucionales de clasificación que se pueden emplear para la conducción autónoma. 
    
    \item Se realizará un estudio y se llevarán a cabo pruebas de diferentes modelos de redes neuronales convolucionales y redes neuronales recurrentes de regresión que se pueden emplear para la conducción autónoma. 

\end{enumerate}


\section{Requisitos}

El proyecto se desarrollará basándose en los subobjetivos mencionados anteriormente y tendrá que ajustarse a los requisitos de partida del proyecto: 

\begin{enumerate}
    \item La simulación se realizará en el simulador Gazebo, en concreto en la versión 7.15.0. El modelo de coche empleado es el modelo f1ROS (posee una cámara como sensor) creado por la organización JdeRobot \footnote{\url{https://jderobot.org/Main_Page}}. Este modelo se encuentra disponible en el repositorio de Github JdeRobot-assets \footnote{\url{https://github.com/JdeRobot/assets}}.

    \item Se empleará el \textit{middleware} robótico \acrshort{ros}, en concreto en la versión \textit{ROS Kinetic}. Este \textit{middleware} que simplifica el desarrollo de software robótico se explicará en mayor detalle en el Capítulo~\ref{cap.estado}. 

    \item El sistema operativo que se empleará en este proyecto será Ubuntu 16.04.
    
    \item El lenguaje de desarollo empleado será Python. Debido a la compatilibidad con el \textit{middleware} ROS Kinetic no se ha empleado Python-3.X, sino que se utiliza Python-2.7.
    
    \item Se hará uso de la API de redes neuronales Keras, escrita en Python y capaz de ejecutarse sobre TensorFlow, CNTK o Theano. En este proyecto se ejecutará sobre TensorFlow y se empleará la versión 2.2.4.
    
    \item Las soluciones deben ser ágiles. Los algoritmos propuestos no pueden detenerse demasiado tiempo a pensar cuál será el próximo movimiento del vehículo, porque debe reaccionar rápido, en tiempo real y con movimientos suaves.
    
\end{enumerate}


\section{Metodología}

El desarrollo del proyecto se ha realizado siguiendo una metodología iterativa, donde cada iteración está compuesta por varias fases: determinar objetivos, planificación, diseño e implementación, análisis de riesgos, además de reuniones periódicas con los tutores.\\

Se ha decidido seguir el modelo de desarrollo en espiral, creado por Barry Boehm~\cite{modelo_espiral}~\cite{modelo_espiral1}~\cite{modelo_espiral2}. Este modelo se adapta perfectamente a este tipo de proyectos, ya que permite separar el comportamiento final en varias subtareas más sencillas y después juntarlas. Además, el modelo permite una gran flexibilidad ante cambios en los requisitos, algo muy común en estos proyectos.\\

Este modelo de ciclo de vida permite obtener prototipos funcionales poco a poco, a la vez que se realiza el desarrollo del producto de forma incremental. El modelo consta de diferentes iteraciones, también conocidas como ciclos. En cada ciclo existen cuatro fases bien diferenciadas: (1) Se concretan los objetivos específicos que deben cumplirse para que el ciclo actual se considere terminado en función de los objetivos finales; (2) se realiza un análisis detallado de cada posible riesgo que pueda tener el objetivo definido y se planean estrategias alternativas; (3) se desarrolla el producto y se realizan las pruebas necesarias; (4) se analizan los resultados obtenidos a través de las pruebas, y se planifica la siguiente iteración.\\


\begin{figure}[H]
  \begin{center}
    \includegraphics[width=0.4\textwidth]{figures/Objetivos/espiral.png}
		\caption{Modelo en espiral}
		\label{fig.espiral}
		\end{center}
\end{figure}

Esta metodología se ha llevado a cabo mediante reuniones semanales con los tutores en las que se analizaban los resultados de cada iteración, y en función de los resultados se fijaban nuevos objetivos. Además, en estas reuniones se analizaban los posibles fallos y se resolvían las dudas que iban surgiendo.\\


El código desarrollado semanalmente se ha subido al repositorio propio público de Github \footnote{\url{https://github.com/RoboticsURJC-students/2017-tfm-vanessa-fernandez}}, que emplea el sistema de control de versiones. Además, se ha desarrollado una bitácora en la página de JdeRobot \footnote{\url{https://jderobot.org/Vmartinezf-tfm}}, donde semanalmente se han explicado los avances y se han mostrado los resultados mediante imágenes y vídeos.\\

El resultado del TFM, las diferentes redes neuronales desarrolladas, se encuentran disponibles en el repositorio Github como software libre.



\section{Plan de trabajo}

Las etapas en las que se divide el proyecto, que se corresponden con el modelo en espiral, son:

\begin{itemize}
    \item Familiarización con la API de redes neuronales Keras y estudio de diferentes soluciones de aprendizaje extremo a extremo para conducción autónoma. En esta etapa se ha descargado e instalado Keras, así como todo el software necesario para desarrollar el proyecto. Además, se ha estudiado la creación de redes neuronales convolucionales en Keras, y su uso en algunos proyectos de la organización JdeRobot.
    
    \item Desarrollo de la infraestructura necesaria en Gazebo, y de una aplicación de control visual que permita la conducción del coche integrando una red entrenada en Keras.
    
    \item Creación de una base de datos que permita entrenar una red neuronal con la información visual del coche y los datos de velocidad.
    
    \item Estudio y mejora de redes neuronales convolucionales de clasificación aplicadas a la conducción autónoma. Se realizarán múltiples pruebas para tratar de conseguir la red más robusta posible y emplearla en la aplicación desarrollada.
    
    \item Estudio y mejora de redes neuronales convolucionales y recurrentes de regresión aplicadas a la conducción autónoma. Se realizarán diversas pruebas para intentar conseguir la red más robusta posible y emplearla en la aplicación desarrollada.
\end{itemize}

