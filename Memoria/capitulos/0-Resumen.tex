\chapter*{Resumen}

En la última década, en \acrfull{va} se está estudiando ampliamente la conducción autónoma. Los humanos somos capaces de mirar a la carretera y saber al instante que acción llevar a cabo. En función a la situación en la que nos encontramos sabemos qué acciones llevar a cabo para lograr una buena conducción.  Sin embargo, este procedimiento es más complicado para los ordenadores. En la actualidad se está investigando ampliamente cómo emplear las \acrfull{rna} para predecir comportamiento autónomo en vehículos. En este proyecto se estudia la conducción autónoma en simuación mediante redes neuronales basadas en información visual.\\

Las decisiones tomadas por la red neuronal están determinadas por los datos empleados durante el entrenamiento de la red. Por lo tanto, cuanto más representativo sea el conjunto de datos, mejor rendimiento se espera que tenga la red. Además, el coche deberá ser capaz de conducir en diferentes entornos. Por ello, se creará un conjunto de datos a partir de un piloto manual que conduce de forma autónoma a través de un algoritmo basado en visión.\\


El objetivo principal es proporcionar una comparativa de diferentes modelos de redes neuronales que se pueden emplear para la conducción autónoma. Por este motivo, se estudiarán y se llevarán a cabo pruebas con diferentes arquitecturas de redes. Asimismo se establecerán conclusiones sobre los datos de entrada a la red, los diferentes experimentos realizados y los resultados obtenidos. \\

Se ha estudiado el empleo de redes neuronales convolucionales de clasificación en conducción autónoma, realizando múltiples pruebas para tratar de conseguir la red más robusta posible y emplearla en el pilotaje del vehículo.\\

Se ha estudiado el empleo de redes neuronales convolucionales y redes neuronales recurrentes de regresión en conducción autónoma. Se han llevado a cabo diversas pruebas para intentar conseguir la red más robusta posible y emplearla en el pilotaje del coche.